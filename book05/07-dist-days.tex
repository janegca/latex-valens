\section{The Operative and the Inauspicious Day. This is the Real Distribution of the Days}

\textbf{/215K/} To find the operative and the inauspicious day, calculate as follows: multiply the full years of the nativity by 5 1/4. Determine the days from the birth date to the day in question. Add this figure to the previous result and divide by 12. Count the remainder from the Ascendant, giving one to each sign. (A few astrologers count from the sign just following the \Moon.) We examine the sign where the count stops to see if it is operative or not. 

It \mndl  is necessary to note how the Moon and its inclination are configured with respect to the sign. If on the day in question the Moon is in aspect with the sign of its inclination, and if they are in operative signs, the day will be fine, noteworthy, profitable. <If they are in inoperative signs, the day will be average.> I

If the day and the inclination of the Moon are found in the same sign, results will be even better. If on that day the Moon is turned away from the inclination, but both are in operative signs, the day will be average, not entirely inauspicious. If they are in the other signs, the day will be miserable, hurtful, and dangerous.

It \mndl is necessary to examine how the ruler of the day is configured, which stars it has in aspect, whether it is in the same sign, an operative sign, \textbf{/205P/} or a turned-away sign. <It is also necessary to examine> how the stars in transit are situated with respect to the day and its ruler. The quality of the day will be evident from the nature of each sign and star. If a nativity has the day in a given sign, and there is a transit or aspect of a star with that day, the day will be operative for good or bad depending on the stars in conjunction. It is the same as the results indicated by the year: the day will be effective for those results when it comes to the places of the transmissions and receptions and to the points in square or opposition to these places.

An example: Hadrian year 4, Mechir 13, the first hour of the night. I am investigating Antoninus year 20, Phaophi 10. <36 full years multiplied by 5 1/4 gives 189.> There are 243 <days from Mechir 13 to
Phaophi 10>, and altogether they total 432. I subtract 360 for a result of 72. I count this from the Ascendant in \Virgo\xspace and stop at \Leo. The day just precedes an angle <Ascendant>. The \Sun, the ruler at the nativity, was in opposition to the \textbf{/216K/} day (because the \Sun\xspace at the nativity was in \Aquarius), \Mars\xspace was in transit, and the \Moon\xspace in \Capricorn\xspace was turned away. The day was precarious. In addition it was in the <XII> Place having to do with slaves; the native became enraged at a slave.

\mndl <It is necessary to note how the day and its ruler are configured together, what stars are in aspect with this place and its ruler, and at what phase they are (i.e. morning or evening rising, whether eastern,
western, or acronychal, whether they are at the first or second stationary point, or are proceeding with their proper motions) and whether they are in their houses, triangles, or exaltations.>
…This too is a scientific method: the signs in which the Sun is located foretell the outcome of the month…


\newpage