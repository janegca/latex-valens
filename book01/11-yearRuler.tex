\section{The House Ruler of the Year (11K, 10P)}

\mnm[0.3cm]
If you want to know the houseruler of the year, calculate in the same way. To continue with the previous example: the full years of the Augustan era are 148, the leap years are 36, plus the one day of Thoth 1, for a total of 185. I divide by 7 for a result of 26, remainder 3. Count this $<$3$>$ from the \Sun’s
$<$day$>$. The year goes to \Mars. 

Now that you have found the ruler of the year, you can find the ruler of the month as follows, applying the arrangement of the spheres in ascending order: Thoth $<$1$>$ is \Mars’. Since
Thoth 29 goes to \Mars\xspace again, the 30th is \Mercury’s. Phaophi 1 is \Jupiter’s, Phaophi 30 is \Venus’s, Athyr
1 is \Saturn’s, Choiak 1 is the \Moon’s, Tybi $<$1$>$ is \Mercury’s, and Mechir $<$1$>$ is \Venus’s. Since the
ruler of the year is \Mars, of the month, \Venus, of the day, \Mercury, and of the hour, the \Sun, it will be necessary to examine how these stars are situated at the nativity. 

\mndl[0.2cm]
If they are in their proper places and proper sect, they indicate activity/occupation, especially when the ruler of the year happens to be transiting the current year, the ruler of the month transiting the current month, and the ruler of the day transiting the current day. If however they are unfavorably situated and have malefics in aspect, they indicate reversals
and upsets.

To me it seems more scientific to take the full years of the Augustan era plus the leap years (as was just stated), plus the days from Thoth 1 to the birth date, then to divide by 7 and count the remainder from the \Sun$<$’s day$>$. Then consider that $<$day’s star$>$, where the count stops, the ruler of the year. The first day of the month of each nativity will control the birth day. It does not seem reasonable for everyone born in the same year to have the same houseruler $<$=ruler of the year$>$. In general, the old astrologers took the ruler of the year and of the universal rotation from the first day of Thoth (where they put the start of the new year), but it is more scientific to take it from the rising of Sirius.\textbf{/27P/}
\newpage