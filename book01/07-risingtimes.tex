\section{The Rising Times of the Signs (7K, 6P)}
\index{signs!rising times}
How many hours each sign takes to rise can be figured from the rising times of each sign. For example: \Aries\, rises in 20 <equatorial times>; now an hour has 15 equatorial times. If you take 15 from 20, the result is 5, which is 1/3 of 15. Therefore \Aries\, will rise in 1 1/3 hour.

You can discover how long each degree takes to rise thus: double the rising time of each sign; multiply this by six—the result <for \Aries> is 240. The degree is 8 “months” <=8/12 of an equinoctial time>\footnote{240/360 = 2/3. 12 x 2/3 = 8.}.

For each sign the amount its rising time is more or less <than another sign’s> can be found as follows: \Aries\, rises in 20; \Libra\, in 40, for a total of 60. The rising time of a sign plus the rising time of the sign in opposition will total 60. The hours of a sign plus the hours of the sign in opposition will total 4 hours. The “days” and “months” of each sign plus those of the sign in opposition will total two “years.” By however much one sign exceeds the half, by so much the sign in opposition will fall short, and vice–versa. 

So—in the previous example—subtract the lesser \textbf{/23P/} from the greater, 20 from 40; the remainder is 20. One-fifth of this is 4, so the addition/subtraction factor for each sign is 4. If to the 20 of \Aries\, we add 4, the result is 24. In this time \textbf{/24K/} \Taurus\, will rise. Then \Gemini\, in 28, \Cancer\, in 32, \Leo\, in 36, \Virgo\, in 40, \Libra\, in 40. From \Scorpio\, to \Pisces\, subtract in the same manner. By investigating in this way, you will find <the rising times> for each klima.

Another method: assume \Leo\, rises in 36; the same for \Scorpio, but \Taurus\, and \Aquarius\, in 24. <When subtracted> the result is 12, of which the third part is 4. This is the addition/subtraction factor. And so by investigating in this way, you will find the rising times for each klima.

The difference between klimata and the progressive increase <of the rising times> are calculated as follows: in the first klima the rising times from \Cancer\, to \Sagittarius\, total 210; 1/6 of this is 35. In this amount \Leo\, rises. Continuing with the procedure at hand, if you subtract the 25 of \Aquarius\, and take one third of the remainder, you will know the rising times of the signs.

Given that there are 7 klimata, in the seventh, from \Cancer\, to \Sagittarius, the rising times total 234. If
you subtract the 210 of the first klima from 234, 24 are left. One-sixth (since there are 6 klimata between) of this is 4. Thus 4 is the increase needed for each klima in the construction of the table of rising times. So in the first klima the rising time from \Cancer\, to \Sagittarius\, is 210. In the second klima, 214; in the
third, 218; in the fourth, 222; in the fifth, 226; in the sixth, 230; in the seventh, 234.

\newpage