\section{A Handy Method for the Seven Zone System [or the Sabbatical Day] (10K, 9P)}

For the week [and the Sabbatical day] proceed as follows: take the full years of the Augustan era and the leap years, and add to that sum the days from Thoth 1 to the birth date. Then subtract as many 7’s as possible $<$=divide by 7$>$. Count the result off from the \Sun’s day, and the birth date will belong to the star at which the count stops. 

The order of the stars with respect to the days is \Sun, \Moon, \Mars, \Mercury, \Jupiter, \Venus, \Saturn.

The arrangement of their spheres is \Saturn, \Jupiter, \Mars, \Sun, \Venus, \Mercury, \Moon.

\mnm[0.3cm]
It is from this latter arrangement that the hours are named, and from the hours, the day of the next star in sequence. For example: Hadrian year 4, Mechir 13 (in the Alexandrian calendar), the first hour of the night. The full years of the Augustan era are 148, the leap years are 36, and from Thoth 1 to Mechir 13 are 163 days. The total is 347. I divide by 7 for a result of 49, remainder 4. Starting from the \Sun’s day, the count $<$4$>$
comes to \Mercury’s day. The first hour of that day belongs to \Mercury.

\begin{longtable}[c]{|r|c|r|c|}
\hline
\multicolumn{2}{|c|}{Day Hours} & 
\multicolumn{2}{|c|}{Night Hours} \\
\hline
\endhead
1  & \Mercury	& 1  & \Sun		\\
2  & \Moon 		& 2  & \Venus	\\
3  & \Saturn 	& 3  & \Mercury	\\
4  & \Jupiter	& 4  & \Moon		\\
5  & \Mars		& 5  & \Saturn	\\
6  & \Sun			& 6  & \Jupiter	\\
7  & \Venus		& 7  & \Mars		\\
8  & \Mercury	& 8  & \Sun		\\
9  & \Moon		& 9  & \Venus	\\
10 & \Saturn		& 10 & \Mercury	\\
11 & \Jupiter	& 11 & \Moon\footnote{\textbf{/26P/}}	\\
12 & \Mars		& 12 & \Saturn	\\
\hline
\end{longtable}

The next day, Mechir 14, continues in this pattern: the first hour belongs to Jupiter.

\newpage