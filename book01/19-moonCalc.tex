\section{A Hipparcheion Concerning the Calculation of the Sign of the Moon (19K,17P)}
\index{planets!\Moon!finding}
I handily find the sign of the \Moon\, as follows: add the <correct> factor for the year in question from the table of kings below. Divide the factor by three, not discarding the remainder, but keeping it. If the remainder is one, add 10 to the number; if the remainder is 2, add 20; if the remainder is 3, add nothing—the number divides evenly. 

Next take one-half of the months from Thoth until the birth date, and add the number of days <in the month of birth> to the first number. Divide by 30 (if possible) and count off the remainder from the \Sun’s sign. If it was in the beginning <of the sign>, give 2 1/2 <to each sign>; if it is towards the end, give the appropriate amount. The Moon is wherever the count stops.

Use the same method to find the date of a given nativity: add the factor to the year in question and divide (as explained) by 3. Then add one-half of the months, note the number. Next estimate the distance from the sun to the moon \textbf{/31P/} by assigning 2 1/2 <days> to each sign. Now determine which is the larger number. If <the number derived from> the distance from the \Sun\, to the \Moon\, is larger, \textbf{/32K/} subtract from it the previously calculated number and the result will show the date. If the distance is less, add 30 to it, then subtract the previously calculated number. If the two numbers are both divisible by 30, the \Moon\, is in \Conjunction with the \Sun.

For example: Hadrian year 3, Athyr 28. I add 2 (the customary factor for this king) to year 3, for a total of 5. I divide by 3, the remainder is 2; therefore I add 20, for a total of 25. One-half of the months <from Thoth to Athyr> is 1 1/2, plus the 28 <days in Athyr> make the total so far 54 1/2. I divide by 30, for an answer of 1, remainder 24 1/2. The \Moon\, will be this many days from conjunction with the \Sun. This number I count off from the \Sun’s position in \Sagittarius, giving 2 1/2 to each sign. The \Moon\, is in \Virgo\, on the aforesaid day.

To find the date as follows: again to year 3 I add 2, then divide by 3, for a remainder of 2. Therefore I add 20, for a total of 25, then one-half of the months, 1 1/2, to get the total 26 1/2. Then I estimate the distance from the \Sun\, to the \Moon\, (i.e. from \Sagittarius\, to \Virgo) to be 24 1/2 days. Since it is not possible to subtract 26 1/2, the previous total, from 24 1/2, I add 30 to it and get 54 1/2. Now from this I subtract 26 1/2, with 28 as the result. This indicates the date of birth. The customarily added factors for each king is appended, in chronological order as follows:


\begin{footnotesize}
\begin{longtable}{lcccc}
\caption{Years of the Roman Kings}
\label{Table 1.1} \\
\hline
\multirow{2}{*}
	{\textbf{King}}     & 
	\textbf{Years}   & 
	\textbf{<Running} & 
	\textbf{Subtract}   & 
	\textbf{Remainder} \\
 				   & 
	\textbf{of}  & 
	\textbf{Total>} & & \\ 
\hline
\endfirsthead
% use empty optional title [] to suppress additional pages in 
%   list of tables
%\caption[]{Inclinations of the Moon} \\
\hline
\multirow{2}{*}
	{\textbf{King}}     & 
	\textbf{Years}   & 
	\textbf{<Running} & 
	\textbf{Subtract}   & 
	\textbf{Remainder} \\
 				   & 
	\textbf{of}  & 
	\textbf{Total>} & & \\ 
\hline
\endhead
Augustus 1 & 43 & 44 & 30 & 14 \\
\multicolumn{5}{l}{\parbox{9cm}{\hspace{1em}I add this figure [14] to Tiberius. The years of Tiberius are:}}\\ \\
Tiberius & 22 & [36] & [30] & [6] \\ 
\multicolumn{5}{l}{\parbox{9cm}{\hspace{1em}for a total of 36 [22+14]. I subtract 30 with a remainder of 6:}} \\ \\
Gaius & 4 & 10 & & 10 \\ 
Claudius & 14 & 24 & 19 & 5 \\
Nero & 14 & 19 \\
\multicolumn{5}{l}{\parbox{9cm}{\hspace{2em}The 19-year period is full. Since this period is operative, \textbf{/32P/} we add (in order to complete 30) 11 years to Vespasian's reign:}} \\ 
\\
Vespasian & 10 & 21 & 19 & 2 \textbf{/33K/} \\ 
Titus & 3 & 5 & & 5 \\ 
Domitian & 15 & 20 & 19 & 1 \\ 
Nerva & 1 & 2 &  & 2 \\ 
Trajan & 19 & 21 & 19 & 2 \\ 
Hadrian & 21 & 23 & 19 & 4 \\ 
Antoninus & 23 & 27 & 19 & 8 \\ 
\parbox{2cm}{\tiny{Antoninus \& \\ Lucius Commodus}}
	 & 32 & 40 & 30 & 10 \\ 
\parbox{2cm}{\tiny{Severus \& \\ Antoninus}}
	 & 25 & 35 & 30 & 5 \\ 
Antoninus &4 & 9 & & 9 \\
Alexander & 13 & 22 & 19 & 4 \\ 
Maximianus & 3 & 7 &  & 7 \\ 
Gordianus & 6 & &  & \\ 
Philip & 6 & 
	\multicolumn{3}{l}{The 19-year period is full.}\\
\bottomrule
\end{longtable}
\end{footnotesize}

\newpage