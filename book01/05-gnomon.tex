\section{The Gnomon of the Ascendant (5K)}

Take the degree-position of the \Sun\xspace with reference to its “Ascending \textbf{/20P/} time,” and multiply it by ten. \textbf{/21K/} (Do this for day births; for night births take the point in opposition <to the \Sun>.) Then multiply the result by the given hour/time <of birth>, whether day or night, whether given in whole hours or including fractions. Then divide by 360, and treat the remainder as the “gnomon of the Ascendant.”

For example: klima <2>, second hour of the day; \Sun\xspace in \Cancer\xspace 21\deg, \Moon\xspace in \Aries\xspace 22\deg. The “Ascending time” of the \Sun’s degree-position is 22;24. Multiply this by ten for a result of 224. This figure multiplied by two, then divided by 360\deg, gives 88. This is the gnomon of the Ascendant.

Another example: the \Sun\xspace in \Capricorn\xspace 19\deg. The birth was in the third hour of the night. The \Moon’s motion was 12 17/30\deg. I enter the column of the table under the third hour, where I find at 12 of motion, 41 1/2, and at 13 of motion 44 1/2. The difference between 44 1/2 and 41 1/2 is 3, and 17/30\deg\xspace times 3 equals 1 7/10\deg. I add this figure to 41 1/2 because the \Moon’s motion was 12 17/30\deg. All
together the degrees total 43 1/5. Now add to this figure the \Sun’s degree-position, 19\deg. The total is 62 1/5\deg. I count this off from \Cancer\xspace, since the birth was at night, and the Ascendant is in \Virgo\xspace 2\deg 12'.
According to the table, the Ascendant was \Virgo\xspace 3\deg.

For new-moon births, it will be necessary to look carefully at the term of the new moon and the ruler of the sign. Whichever of them controls the degree which just precedes the hour, that degree will be the Ascendant. For full–moon births, it will be necessary to determine the term of the full moon and the ruler
of its sign.

For day births, it is necessary to take the \Sun’s <degree-position> and the remaining degrees of the \Moon and to divide by 30. Find the remainder in the table of rising times and multiply the figure entered there at the \Sun’s sign by the degrees of the sign. Then, having added the \Sun’s degree-position, divide by 30. Whatever is left will be the solar gnomon. We note this figure carefully and make it the lunar gnomon as follows. Double the \Moon’s degree-position; divide by 30; multiply the remainder by 12 and add the \textbf{/21P/} \Moon’s degree-position. Then divide by 30 and the remainder will be \textbf{/22K/} the lunar gnomon.

For night births, add the remaining degrees of the \Moon\xspace to the \Sun’s degree-position, divide by 30 in the table of rising times. We add the remainder to the \Sun’s sign and note the “horary magnitude.” We multiply the sun’s degree-position. We add the \Sun’s degree-position and divide by 30. The remainder will be the solar gnomon. If the solar gnomon is greater than the lunar, then subtract from the Ascendant.
If the lunar is greater, [then] add whatever the excess is. If they are equal, do not add or subtract. Likewise if the remainder is 15 or less, there will be addition or subtraction.

Having determined by sign the sign in the Ascendant, we will find the degree in this way: note the year of the quadrennium as it is given below. Add the hours entered there to the hour/time <of birth>. Calculate the \Moon’s degree-position <for the new time>. We will consider the Ascendant to have that position.

\renewcommand{\arraystretch}{1.2}
\begin{table}[h!]
\begin{center}
\begin{tabular}{| p{.3\linewidth} | p{.3\linewidth} |}
\hline
First Year & 1 Hour \\
Second Year & 6 Hours \\
Third Year & 12 Hours \\
Fourth Year & 6 Hours \\
\hline
\end{tabular}
\end{center}
\end{table}

The year of the quadrennium is associated with the rising of the Dog Star <Sirius>:

\begin{table}
\begin{center}
\begin{tabularx}{\textwidth}	{| l | X |}
\hline
\multicolumn{2}{|c|}{The Quadrennium} \\
\hline
First Year
	& Sirius rises with Cancer in the first day hour \\
Second Year 
	& It rises with Libra in the sixth day hour \\
Third Year 
	& It rises with Capricorn in the twelfth day hour \\
Fourth Year 
	& It rises with Aries in the sixth night hour \\
\hline	
\end{tabularx}
\end{center}
\end{table}
\mnt[0.2cm]
Calculated in this way, the Ascendant is useful in casting horoscopes in later years <after birth>, the hours from the quadrennium table being added (depending on the year in question), then counted from the hour of birth. Put the Ascendant in whichever hemisphere of the sky—day or night—the count ends, and interpret the nativity with respect the the stars which are occupying an angle at that time.

\newpage