\section{Listening and Beholding Signs (8K, 7P)}
\index{signs!listening and beholding}
\index{signs!beholding | see{listening and beholding}}
Similarly the listening and the beholding signs (the sextile signs) must be calculated from their rising times as follows: \Pisces\, beholds \Taurus\,; in the second klima the rising times of the six signs from \Pisces\, <to Leo> total 160 and from \Taurus\, to \Libra\, total 200. \Pisces\, is less than \Taurus\, and therefore listens to it. The rising times of the two groups total 360. 

Likewise from \Gemini\, to \Scorpio\, there are 212 and from \Leo\, to \Capricorn\, 212; therefore \Gemini\, and \Leo\, are of equal rising time and listen to each other. 

Again from \Virgo\, to \Aquarius\, is 200, from \Scorpio\, to \Aries\, 160. They behold each other \textbf{/24P/} and <\Scorpio\, listens to \Virgo\,. From \Leo\, to \Capricorn\, is 212>, and from \Libra\, to \Pisces\, is 180\ldots 

From \Sagittarius\, to \Taurus\, is 148, and from \Aquarius\, to \Cancer\, is 148. They listen to each other and are of equal rising times. Similarly for the rest <of the signs>.

Some astrologers consider the sympathy of the sextile signs to be as follows: they add the rising times of the two <sextile> signs \textbf{/25K/} and divide the sum in half. Then they see if the intervening sign actually rises in that time. For example: \Aries\, 20 plus \Gemini\, 28 totals 48, half of which is 24. \Taurus\, actually does rise in that time. Therefore \Aries\, will have sympathy with \Gemini\,. Likewise \Taurus\, with \Cancer, since their rising times total 56, half of which is 28. In this time \Gemini\, actually does rise. 

Likewise \Gemini\, with \Leo\, and \Cancer\, with \Virgo. \Leo\, however does not have sympathy with \Libra\, because their rising times total 78, half of which is 39—but Virgo actually rises in 40. Likewise for the rest of the signs\footnote{These conditions act as mitigations for averse signs and are more fully explained by Paulus Alexandrinus in sections 12 and 13. Also see Firmicus Maternus, Book 1\S 29 and Ptolemy's \textsl{Tetrabiblios} Chapters XVII and XVIII, Chris Brennan's \textsl{Hellenistic Astrology} p315-17, or Demetra George's \textsl{Ancient Astrology, Vol 1} p166.}.

\newpage