\section{Listening and Beholding Signs (8K, 7P)}

Similarly the listening and the beholding signs (the sextile signs) must be calculated from their rising times as follows: \Pisces\xspace beholds \Taurus\xspace; in the second klima the rising times of the six signs from \Pisces\xspace <to Leo> total 160 and from \Taurus\xspace to \Libra\xspace total 200. \Pisces\xspace is less than \Taurus\xspace and therefore listens to it. The rising times of the two groups total 360. 

Likewise from \Gemini\xspace to \Scorpio\xspace there are 212 and from \Leo\xspace to \Capricorn\xspace 212; therefore \Gemini\xspace and \Leo\xspace are of equal rising time and listen to each other. 

Again from \Virgo\xspace to \Aquarius\xspace is 200, from \Scorpio\xspace to \Aries\xspace 160. They behold each other \textbf{/24P/} and <\Scorpio\xspace listens to \Virgo\xspace. From \Leo\xspace to \Capricorn\xspace is 212>, and from \Libra\xspace to \Pisces\xspace is 180\ldots 

From \Sagittarius\xspace to \Taurus\xspace is 148, and from \Aquarius\xspace to \Cancer\xspace is 148. They listen to each other and are of equal rising times. Similarly for the rest <of the signs>.

Some astrologers consider the sympathy of the sextile signs to be as follows: they add the rising times of the two <sextile> signs \textbf{/25K/} and divide the sum in half. Then they see if the intervening sign actually rises in that time. For example: \Aries\xspace 20 plus \Gemini\xspace 28 totals 48, half of which is 24. \Taurus\xspace actually does rise in that time. Therefore \Aries\xspace will have sympathy with \Gemini\xspace. Likewise \Taurus\xspace with \Cancer, since their rising times total 56, half of which is 28. In this time \Gemini\xspace actually does rise. 

Likewise \Gemini\xspace with \Leo\xspace and \Cancer\xspace with \Virgo. \Leo\xspace however does not have sympathy with \Libra\xspace because their rising times total 78, half of which is 39—but Virgo actually rises in 40. Likewise for the rest of the signs.

\newpage