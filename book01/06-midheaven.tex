\section{The Midheaven (6K, 5P)}

Midheaven $<$MC$>$ can be handily found in this way: using the rising times for the $<$appropriate$>$ klima, add the rising times from the Descendant to the point in opposition, then take half of the sum.
Count this off from the Descendant. MC will be where the count stops. \textbf{/22P/} 

For example: the Ascendant is \Capricorn\xspace 15\deg\xspace in the second klima. I take the rising times from the \textbf{/23K/} Descendant, \Cancer\xspace 15\deg, to \Capricorn\xspace 15\deg; the total is 214. Half of this is 107. Adding to this the 15\deg\xspace of \Cancer, I count from that same point. The count stops at \Scorpio\xspace 2\deg, which is MC. Similarly for the other $<$degree-positions$>$.

If you wish to know the length of the hours of the day, in all cases add the rising times from the \Sun’s degree-position to the point in opposition. Take 1/15 of that and you will know the length of the hour.

For example: assume the previous Descendant, \Cancer\xspace 15\deg, is the \Sun’s position. The rising times from there to the point in opposition total 214; 1/15 of 214 equals 14 [remainder 4] with 4/15 parts of an hour left over. Therefore the day in the klima of Syria, with the \Sun\xspace in \Cancer\xspace 15\deg, will be 14 4/15 hours. 

If you want to know the length of the night, work out the calculation by adding the rising times from the point opposite the \Sun\xspace to $<$the \Sun’s$>$ position. Similarly with the rest of the signs.

\newpage