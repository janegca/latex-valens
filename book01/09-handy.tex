\section{A Handy Method for New and Full Moons (9K, 8P)}

To find new and full moons handily: take the distance from the \Sun’s degree-position to the \Moon’s, and determine how many dodekatemoria there are between. Count this amount off from the \Sun’s degree position and you will find the new moon there. The \Moon\xspace will be as many degrees from conjunction as there are dodekatemoria which have been determined. 

For full-moon nativities, take the distance from the point opposite the \Sun\xspace to the \Moon, and determine how many dodekatemoria there are $<$[in] this distance$>$. Subtract that amount from the position of the point opposite the \Sun. The full moon will be there. 

Also, if you add 15\deg\xspace to the degree-position of the full moon, you will find the position of the next new moon. If you add 15\deg\xspace to the position of the new moon, you will find the next full moon. For example: Mesore 2, the \Sun\xspace in \Leo\xspace 5\deg, the \Moon\xspace in \Libra\xspace 26\deg. The distance from the \Sun\xspace to the \Moon\xspace is 81\deg, which is very nearly 7 dodekatemoria. Therefore the moon is seven days past the conjunction. Next I deduct the 7 from \textbf{/25P/} the \Sun’s position and arrive at \Cancer\xspace 28\deg. The previous new moon occurred there. From Mesore 2 I subtract 7; the result is Epiphi 25. If we add 15 to \Cancer\xspace 28\deg
the result is \Leo\xspace 13\deg. The full moon will be at \Aquarius\xspace 13\deg. \textbf{/26K/} 

Calculate the full moon as follows: assume Mechir 13, the \Sun\xspace in \Aquarius\xspace 22\deg, the \Moon\xspace in \Scorpio\xspace 7\deg. I take the distance from the point opposite the \Sun, \Leo\xspace 22\deg, to the \Moon’s position; this is
75\deg, which equals 6 dodekatemoria. I subtract this from \Leo\xspace 22\deg. The result is \Leo\xspace 16\deg, where the full moon occurred. Again I subtract the 6 dodekatemoria from Mechir 13, for a result of Mechir 7. Since from the conjunction to the full moon there are 15 days, I add the 8 $<$days from Mechir 7 to Mechir 13$>$ to this 13, and get 21. Therefore the \Moon\xspace is that many days $<$21$>$ from new.

\newpage