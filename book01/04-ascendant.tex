\section{Finding the Ascendant}
\index{Ascendant!how to find}
Having determined accurately the \Sun’s degree-position at the nativity, note where the \textsl{dodekatemorion} falls. The sign in trine to the left of this position will be the Ascendant, or the equivalent sign (i.e. either masculine or feminine), providing you take into account the distinction between night and day births. For example: let the \Sun\, be in \Aquarius\, 22\deg. The \textsl{dodekatemorion} of this point is in \Scorpio; the sign in trine to the left is \Pisces. If the birth was in the day, either \Pisces\, or \Taurus\, or \Cancer\, must be the Ascendant. If the birth was at night, one of the diametrically opposite signs <must be>. \Virgo\, would be in the Ascendant in the first hour <of the night>.

Having determined accurately the degree-position of the \Sun, for day births add to this position the rising time of the sign in which the \Sun\, is; then begin to count from the \Moon’s position at the nativity, giving each sign one degree. The Ascendant will be <in the sign> where the count stops, or (as mentioned above) in the equivalent sign. For night births add the rising time of the\Moon’s sign and count from the \Sun’s position at the nativity. Using the previous example again: the \Sun\, in \Aquarius\, <22\deg>, the \Moon\, in
\Scorpio. I add the rising time <of \Scorpio>, 37, to 22\deg <the \Sun’s position>, for a result of 59. I count this off from the \Sun\, and stop at \Virgo. The Ascendant is there.

\textbf{/19P/} Find the number <of days> from Thoth to the day of birth; multiply the hour/time <of birth> by 15 and add the result to the first number. For day births count from \Virgo, giving 30 to each sign. For night births, count from \Pisces.
Alternatively, multiply the hour/time <of birth> by 15 [and add the degree-position of the\Sun]. Then for day births, count from the \Sun\, with reference to the rising time <of the sign> in the klima of birth; for night births, count from the the point opposite the \Sun\, with reference to the rising time. In this way, the mystical, compelling Ascendant will be found. 

For day births \textbf{/20K/} the point of conception will be trine or sextile to the Sun and in the Ascendant; for night births the signs in opposition <to these places> will be the point of conception. As a result, for whatever hour you observe, night or day, you will find the Ascending sign.

To find the Ascendant precisely to the degree, do this: multiply the hour/time of birth by the motion of the \Moon. For day births count from the \Sun’s degree-position; for night births count from the point in opposition <to the \Sun>. The degree where the count stops will be considered the Ascendant. For example: Hadrian year 4, Mechir 13, the first hour of the night. The \Sun\, was in \Aquarius\, 22\deg, the \Moon\, is
\Scorpio\, 7\deg, the motion of the moon in its <204th> day from epoch was 13;52\deg. I consulted the appended
table under 14 in the first row and I found below in the first column of hours, 16. I then counted from the degree in opposition to the \Sun, \Leo\, 22\deg. I stopped in \Virgo 8\deg. If more or fewer degrees are found in the table of rising times, it can be ascertained from the aforementioned procedure whether the hour requires an added or a subtracted factor.

For those born during the day, add the remaining degrees in the \Sun’s <sign> to the \Moon’s degree position and divide by 30. The remainder will be the <degree in> the Ascendant. For those born at night: add the remaining degrees in the \Moon’s <sign> to the \Sun’s degree-position. If the resulting number is
greater than the calculated hour/time <of birth>, the amount by which it exceeds either 30 or the number of the hour will be the Ascendant.

Count the days (including the intercalary days) from Epiphi 25 to the day of birth, and add 22 to this number. Count the result off by 30’s, starting at \Cancer\, for day births, at \Capricorn\, for night births. The Ascendant will be where the count stops, and the degree thus determined will be the degree in the Ascendant.

\newpage