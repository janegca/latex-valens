\section{The Visibility Periods of the Moon (13K, 12P)}
\index{planets!Moon!visibility periods}
The visibility periods of the Moon are as follows: in its first day it appears 4/5 of an hour. In its second day it appears 1 3/5 of an hour. Forecast the time <of its visibility> by multiplying the days <since new moon> by 4, then dividing by 5. For example: it is 15 days since new moon; 4 times this equals 60, of which 1/5 is 12; the Moon, being full, will be visible 12 hours.

\begin{table}[ht]
\begin{center}
\begin{tabularx}{\textwidth}	{| r | l | r | X |}
\hline
Day & Visibility & Day & Visibility \\
\hline
1	& 4/5 hours	& 9		& 7 1/5	\\
2	& 1 3/5	  	& 10	& 8			\\	
3	& 2 1/4		& 11	& 8 4/5	\\
4	& 3 1/5		& 12	& <9 3/5>	\\
5	& 4				& 13	& <10 2/5> \\
6	& 4 4/5		& 14	& 11 1/5	\\
7	& 5 3/5		& 15 	& 8			\\
8	& 5 2/5		& 16	& similarly from <16 to 30>, as
							  from 1 to 15, but subtracting \\				\hline
\end{tabularx}
\end{center}
\end{table}

The month is 29 1/2 days; the year 354\footnote{Schmidt has 359 days.} days.

\newpage
\section{The Invisibility Period of the Moon (14K, 13P)}
\index{planets!Moon!invisibility}
The Moon becomes invisible as it approaches conjunction with the Sun. The calculation of this in each sign is as follows: take one-half of the rising time of the sign in which the \Sun\xspace is located, and at that point the moon will be invisible. For example: the \Sun\xspace in \Aries\xspace in the second klima. The rising time of this sign is 20, half of which is 10. Subtract 10 from 30\deg\xspace <\Aries\xspace 1\deg\xspace = \Pisces\xspace 30\deg>. The \Moon\xspace will become invisible at \Pisces\xspace 20\deg.

\begin{table}[ht]
\begin{center}
\begin{tabularx}{\textwidth}{| l | l | X |}
\hline
\Sun\xspace in: 		& 1/2 Rising Time	
	& \Moon\xspace Invisible in: \\
\hline
\Taurus	& 12 & \Aries\xspace 18\deg \\
\Gemini	& 14 & \Taurus\xspace 16\deg\xspace \textbf{/28P/} \\
\Cancer	& 16 & \Gemini\xspace 14\deg \\
\Leo		& 18 \textbf{/29K/} & \Cancer\xspace 12\deg \\
\Virgo		& 20 & \Leo\xspace 10\deg \\
\hline
\end{tabularx}
\end{center}
\end{table}

Similarly for the rest of the signs.
\newpage
