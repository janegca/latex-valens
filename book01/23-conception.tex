\section{Conception (23K,21P)}
\index{conception}
Having established all this, now we must speak about conception, putting aside complications and rejecting envy.

There are three factors: minimum, mean, maximum, and the difference between each factor is 15 days. If we add or subtract 15 to or from any factor, the next one will be reached. 

The minimum factor is 258, which will apply when the Moon just follows the Descendant (in the Place just following the Descendant). 

The mean factor is 273, which will apply when the Moon is in the Ascendant. 

The maximum factor is 288, which applies when the Moon is in the Descendant. 

If we measure the 15 days of difference in the celestial hemisphere from the Ascendant to the Descendant, we find that 2 1/2 fall to each sign. Let the Ascendant be \Cancer, the Descendant \Capricorn:

\begin{longtable}{cr}
\toprule
\textbf{\Moon's Position} & \textbf{Gestation Period} \\
\hline
\endhead
Just following the Descendant & 258 \\
\Aquarius					& 260 1/2	\\
\Pisces 					& 263		\\
\Aries 					& 265 1/2	\\
\Taurus 					& 268		\\
\Gemini 					& 270 1/2	\\
\Cancer (Ascendant)		& 273		\\
\Leo 						& 275 1/2	\\
\Virgo 					& 278		\\
\textbf{/50P/} \Libra 	& 280 1/2	\\
\Scorpio 					& 283		\\
\Sagittarius 			& 285 1/2	\\
\Capricorn 				& 288		\\
\bottomrule
\end{longtable}

For example: Nero year 8, Mesori 6/7, hour 11 <of the night>; the \Moon\xspace in \Libra, the Ascendant in
\Cancer. Since the \Moon\xspace is at an angle <IC>, the nativity will occur in 280 days 12 hours. We must subtract these days from the 365 days of the year. The result is 84 days 12 hours. Now if we add this 84 to Mesori 6, we come to Phaophi 27, the 23rd hour, which is the time of conception. In other words, if we go from Phaophi 27 to Mesori 6, the total is 280 days. We will now demonstrate this using many \textbf{/51K/} methods, all leading straight to the answer. 

Given the birth date, let us determine the time <from conception> to birth. If the \Moon\xspace is found to be in the hemisphere above the earth, calculate the degrees from the Descendant to the \Moon’s degree-position and assign 2 1/2 to each 30\deg\xspace of arc. Then add this sum to the minimum factor (258), and you will find the conception to have been that many days previous. Count this amount back from the birth date, and you will find the date of the conception to be where the count stops.

If you want another method, calculate the degrees from the \Moon’s degree-position to the Ascendant and assign 2 1/2 <days> to each 30\deg; then subtract this from the mean factor (273). The date of conception will have been that many days previous. Likewise if the \Moon\xspace is in the hemisphere below the earth, calculate the degrees from the Ascendant to the \Moon, then assign 2 1/2 to each 30\deg\xspace division. Summing up, add this to the mean factor (273). The date of conception will have been that many days previous. Or, calculate the degrees from the \Moon\xspace to the Descendant and figure the total number of days by adding 2 1/2 for each 30\deg\xspace division and subtracting the result from 288. The date of conception will have been that many days previous.

For example, so that my readers may understand the determination: Hadrian year 4, Mechir 13/14, hour 1 of the night; the moon in \Scorpio\xspace 7\deg, the Ascendant in \Virgo\xspace 7\deg. Since the \Moon\xspace is found to be in
the hemisphere beneath the earth, I take the degrees from the Ascendant to the \Moon; this is 60\deg. To each 30\deg\xspace I assign 2 1/2, for a result of 5 days. I add this to the mean factor (273), and the result is 278. The conception was that many days ago. \textbf{/51P/} I count back <278> days from the hour of birth; the conception day is Pachon 11.

Alternatively, I subtract the 5 days of the distance <from the Ascendant to the \Moon> from 92, for a result of 87. (The mean factor is 273, which leaves a remainder of 92 when subtracted from 365 days.) If we add 87 to Mechir 14 and count this off from the birth date, we come to Pachon 11. If you calculate the degrees from the \Moon\xspace to the Descendant, i.e. \Pisces\xspace 7\deg, the total is 120\deg. Take 2
1/2 for each 30\deg, for a total of 10 days. Now if I subtract this from the maximum factor (288), the result is 278. If you count this amount back from the day of birth and calculate the \Moon, you will find it to have been in the Ascendant at the delivery.

\textbf{/52K/} If the \Moon\xspace is in the hemisphere above the earth, take the distance in degrees from it to the
Ascendant, assign 2 1/2 to each 30\deg division, and find the total number of days. If you wish, add 92 to this and count off the sum from the birth date forwards; the date of conception will be where the count stops. Vice-versa, calculate from the date which you determined <to be the date of conception> forward to the date of birth, and you will know the number of days. 

If the \Moon\xspace is in the hemisphere beneath the earth, you will calculate from the Ascendant to the \Moon: determining the distance in degrees, assign 2 1/2 days to each 30\deg. Subtract this from 92, and add the result to the birth date. Count from there forwards. That will be the date of conception. Count back from the birth date the amount which you added to the <mean> factor (273).

Another example: Trajan year 17, Mesori 2, hour 11 1/2; the sun in \Leo\xspace 5\deg, the \Moon\xspace in \Libra\xspace 26\deg, the Ascendant in \Capricorn\xspace 24\deg. Since the moon is in the hemisphere above the earth, I take the distance
from it to the Ascendant, which is very nearly 96\deg. To each 30\deg I assign 2 1/2 for a result of 7 1/2 days. I add this to 92, and the sum is 99 1/2. \textbf{/52P/} I count from the birth date forwards and arrive at Athyr 6. Vice-versa <I can count> from Athyr 6 to the birth date as done earlier; the total days are 266. The conception was that number of days previous. If I do not want to add the 7 to 92, I subtract it from 273, for a result of 266. I count this number back from the birth date. Calculating the \Moon, I find it in
\Capricorn, in the Ascendant.

The Moon at the nativity will indicate in which sign the Ascendant of the conception was located. The Moon’s degree-position at the nativity will also be the position of the Ascendant at the conception. Other astrologers calculate the Moon by doubling the degree-position of the nativity’s Ascendant.

Alternatively they take one-fourth of the \Sun’s degree-position at the birth and consider the \textbf{/53K/} sign in
trine to the right of that point to be the Ascendant of the conception. Therefore we will not go astray if we henceforth seek the answers for all nativities using the system given above. Let this be the divinely compelling manner of our future method for solving nativities.

\newpage