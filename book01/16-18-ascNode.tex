\section{A Handy Method for Finding the Ascending Node (16K,15P)}

A handy method for finding the ascending node: take the full years of the Augustan era and multiply them by 19 1/3. Add for each Egyptian month 1\deg\xspace 35' and for each day 3'. Divide by 360\deg\xspace circles. Now count off the remainder <of the division> from \Cancer\xspace in the direction of diurnal motion <=East to West>, \textbf{/29P/} giving 30 to each sign. The ascending node will be where the count stops. For example: Hadrian year 4, \textbf{/30K/} Phamenoth 19. The full years from Augustus are 148; this figure times 19 1/3 equals 2862. From Thoth to Phamenoth there are 10\deg, for a total of 2872. I divide this by 360\deg\xspace for a result of 7; a remainder of 352 is left. This remainder is counted in the direction of diurnal motion from \Cancer\xspace and comes to \Leo 8\deg. The desired ecliptic point will be there, the descending node at the point in opposition.

\mndl[0.2cm]
\index{planets!Moon!nodes}
It will be necessary to examine if benefics are in aspect with these positions, especially with the ascending node. If so, the nativity will be prosperous and effective. Even if the nativity is found to be average or inclined toward diminution, the native will ascend and rise to a high rank. Malefics portend
upsets and accusation.

From the <tables of> lunar epochs and daily motions the ascending node and the sign of its latitude will be found as follows: for example, take the previous nativity, Hadrian year 4, Phamenoth 19. From the epoch to the nativity date is 204. Next to the epoch is entered 12;18 of latitude. Next to 204 is entered 11;37 of latitude. The total is 23;55. Multiply this times 15\deg\xspace and the result is 358\deg\xspace 45'. This is counted from \Leo\xspace in the direction of proper motion <=West to East> and comes to \Cancer\xspace 28\deg\xspace 45'.

\textbf{17K.}\footnote{This appears to be the start of a new section in Kroll without a heading.}
\setcounter{section}{17}

[Another more concise method: the 23;55 is counted from \Leo, 2 given to each sign. The count stops at \Gemini, having allotted 22, with 1;55 remaining. This I multiply by 15\deg, and the result is 28\deg\xspace 45' of \Cancer.)

Next in every case I take the degrees from \Taurus\xspace to the previously determined degree; the distance is very nearly 89\deg. I subtract this amount from the \Moon’s degree-position (which is \Scorpio\xspace 7\deg), and come to \Leo\xspace 8\deg, the ascending node. It will be necessary to do the same calculation for the rest of the nativities.

If I wish to know the sign of the latitude, I will calculate as follows: the latitude entered next to the epoch is 12;18. I multiply only this by 15\deg\xspace and the result is 184\deg\xspace 30'. I count this off from \Leo\xspace and stop
at \Aquarius 4\deg\xspace 30'. Next the “degrees of latitude” entered next to 204 is 11;37. I multiply this figure by
15\deg, and the result is 174\deg\xspace 15'. I add \Aquarius\xspace 4\deg\xspace 30' to this and count the sum off from the same place. The result is \Cancer 28\deg\xspace 45'. \textbf{/30P/} By using this method for the rest of the epochs we will find the sign of the latitude.]\textbf{/31K/}

\newpage
\section{The Determination of the Steps and the Winds of the Moon (18K, 16P)}
\index{winds!of the Moon}
We will find the step and wind as follows: from \Leo\xspace to \Libra\xspace the Moon declines northwards; from \Scorpio\xspace to \Capricorn\xspace it declines southwards; from \Aquarius\xspace to \Aries\xspace it ascends southwards; and from \Taurus\xspace to \Cancer\xspace it ascends northwards\footnote{This description does not agree with the description of the winds given in Book III.4 where the directions are related to the planets sign of exaltation and the right and left squares to that sign.}

\index{steps!of the Moon}
The steps are found as follows: since each step is 15\deg, and since a sign contains 30\deg, each sign comprises two steps. We can find the step of the latitude by starting at \Leo. Since the latitude in the previous nativity was found to be 23;55, I count this off from \Leo, giving 2 to each sign. The count stops at \Cancer\xspace 1;55 <step>. We now know that the Moon is ascending northwards at the sixth step of this wind.

\newpage