\section{Reckoning the Sun and the Five Planets (20K,18P)}
\index{planets!finding the planets}
You will discover the Sun’s degree-position as follows: in every case, to a birth date which falls in the months from Thoth to Phamenoth add 8\deg; you will find the total to be the \Sun’s position. To <birth dates in> Pharmouthi add 7\deg, to Pachon 6\deg, to Payni 5\deg, to Epiphi 4\deg, to Mesore 3\deg. For example: in Phaophi 6, I add 8\deg, total 14; the \Sun\, will be in \Libra\, 14\deg. In Pachon 6, I add 6\deg, total 12. The \Sun will be at \Taurus\, 12\deg.

Since some students have become very enthusiastic about the derivation of numerical data, for them I must append the \textbf{/33P/} handy methods for the rest of the stars, so that through such studies they may gain delightful and precise-to-the-degree methods. They can now make an examination of the more important procedures with the greatest enthusiasm.

Now then, Saturn is to be calculated as follows: take the full years since Augustus and divide by 30, if possible. Multiply the remainder <of the division> \textbf{/34K/} by 12\deg. Multiply the result of the division by 30 (=the synodic period <of \Saturn>) by 5\deg. For each month from Thoth <to the date of birth> add 1\deg, and for each day 2'. Having totaled all this, count from \Cancer\, in the direction of proper motion, giving 30\deg\, to each sign. The star will be where the count stops.

Jupiter as follows: divide the full years from Caesar by 12. Multiply the remainder by 12\deg\, and add
this to the result of the previous division by 12 (=the synodic period <of Jupiter>). Total this, plus 1\deg\, for each month and 2' for each day. Having added, count the sum from \Taurus, giving 12 to each sign.

Mars as follows: take the number of years from Augustus to the year in question, divide by 30, and note whether the remainder is odd or even. If it is even, start counting from \Aries; if it is odd, start from \Libra. Having found this number, double it and add to it 2 1/2 for each month <after Thoth>. If the result is more than 60, count off the amount over 60 from \Libra\, or \Aries, giving 5 to each sign. Wherever the count stops, make note of the sign and examine which sign the \Sun\, is in. If the \Sun\, is found to be west of the star, the star will be behind <=to the west> its calculated position; if the \Sun\, is found to be east of the star, the star will be ahead <=to the east> of its calculated position. In other words, in each case, place the
star nearer the \Sun\, than the sign in which you have calculated it to be. The rest of the stars, especially \Venus, show the same peculiarity when they are moving near the mean position of the \Sun.

Venus as follows: take the years from Augustus to the year in question and divide by 8. Examine the remainder (which will be less than 8) to see if Venus is at a point of maximum eastern elongation <during that year>. If it is, use this point and add the \textbf{/34P/} number of days from that point to the day in question; if not, use the number right above it <in the table>, just as with the \Moon. In other words, if the point of maximum eastern elongation is found to be before the nativity, use it; if it is after the nativity, use the number right above it. Add together the days, then subtract the elongation factor of the sign. Subtract 120\deg\, [for each sign\footnote{[for the sign of the elongation] - marginal note [Riley]}]. Count off the remaining degrees from the adjoining sign [from the sign of the elongation], giving each sign 25\deg. \Venus\, will be where the count stops. \textbf{/35K/} The point of maximum eastern elongation will be clear from the remainders in the calculation of years above. If the remainders in our first calculation are 1, 3, 4, 6, or 7, then \Venus\, is at maximum eastern elongation <during that year>. If the remainders are 2 or 5, it is in motion <during that year>.

\begin{table}[ht] \small
\begin{tabular}{clc}
\hline
<Remainder & Date & Sign of the Elongation> \\
\hline
1 & Phamenoth 10 & \Taurus \\
<2 & \multicolumn{2}{l}{\parbox[t]{6.5cm}{No maximum eastern elongation occurs in this year.>}}\\
3 & Phaophi 10 & \Sagittarius \\
4 & Payni 22 & \Leo \\
<5 & \multicolumn{2}{l}{\parbox[t]{6.5cm}{No maximum eastern elongation occurs in this year.>}}\\
6 & Tybi 8 & \Pisces \\
7 & Mesore 14 \Libra \\
\multicolumn{3}{l}{\parbox[t]{9cm}{In the eighth year \Venus\, has a point of maximum eastern elongation.}} \\
\hline
\end{tabular}
\end{table}

Mercury is calculated as follows: take the days from Thoth to the birth date and add to these in every case an additional 162. Find the total, and if the sum is more than 360, divide by 360 (a circle) and count the remainder off from \Aries, giving 30 to each sign. The star is where the count stops. In every case make it very near the \Sun. For example: if the \Sun\, was in the beginning of its sign, \Mercury\, can be found at the end of the sign. If the \Sun\, is in the end of its sign, \Mercury\, can be found in the next sign. 

An example: Trajan year 13, Phamenoth 18. The full years from Augustus are 138. I divide by 30,
for a result of 4, <remainder 18>. I multiply 5 times the 4 cycles, and the result is 20. I multiply the remainder <of the original division>, 18, by 12, and the result is 216. From Thoth to Phamenoth I count 1 for each month—total 7. All together this is 243. Now I count this sum off from \Cancer\, giving 30 to each sign, and I arrive at \Pisces. \Saturn\, is there.

Next I divide 138 by 12,for a result of 11, remainder 6. This <remainder> times 12 is 72. To each 12 which I divided <into 138> I assign 1, for a total of 11. Also to each month <from Thoth to Phamenoth I assign 1>, for a total of 7. The grand total is 90. I count this off from \Taurus, giving 12 to
each sign. The count stops in \Sagittarius. \Jupiter\, is there. 

Next Mars as follows: from Caesar to the year in question \textbf{/35P/} is 139 <!>. I divide this by 30, for a
result of 4, remainder 11 <!>. (Since the remainder is odd, I know that I must start counting from \Libra.) I double this figure and get 22. For the months from Thoth to Phamenoth the total is 17 <=7 months $\diamond$ 2 1/2>. The grand total is 39. I count this sum off from Libra, giving 5 to each sign. I stop at \Taurus. \Mars\, is there.

Venus as follows: I divide the 139 years by 8 and the remainder is 3. This indicates a point of maximum eastern elongation during that year on Phaophi 10 in \Sagittarius. I add the rest of the days in Phaophi, 20, plus the days from Athyr to Mechir, 120, plus those in Phamenoth, 18, for a total of 158. I subtract 120 for the \textbf{/36K/} maximum elongation and for \Sagittarius. The result is 38, which I count from Capricorn, giving 25 to each sign. The count stops in \Aquarius. \Venus\, is there.

Since there seems to be great <difficulty> about calculating Venus in nativities, I will explain it with another example. 

Hadrian year 4, Athyr 30: the years from Augustus are 148, which I divide by 8, giving a remainder 4. This indicates a point of maximum eastern elongation on Payni 22 in \Leo. Since this point is not applicable because of its being after the date of the nativity, I go to the one right above it <in
the table>, in the third line, Phaophi 10 in \Sagittarius. So I add the remaining 20 days of Phaophi, the days from Athyr to Mesori, 300, and the 5 intercalary days. The total is 325 of the previous year, plus 90 days from Thoth <1> to Athyr 30 of the current year, for a grand total of 415. From this sum I subtract 120 for the maximum elongation and for \Sagittarius, for a result of 295. I count this off from \Capricorn,
giving 25 to each sign and stop at \Sagittarius\, 20. The star is there.

Another example: Hadrian year 4, Mechir 13: the years from Augustus are 149, which I divide by 8, giving a remainder of 5. This indicates no point of maximum eastern elongation. I go to the point above, which is Payni 22 in Leo. I add the remaining 8 <days> in Payni, plus Epiphi and Mesori <60>, plus the
5 intercalary days. The total is 73. Then I add to this the days from Thoth <1> to Mechir 13, 163. The grand total is 236. From this sum I subtract 120 for the maximum elongation and for the sign \Leo. The result is 116. I count this from \Virgo, giving 25 to each sign. The count stops at \Capricorn\, 16\deg. \Venus\, is there.

I calculate Mercury for the same nativity as follows: I add \textbf{/36P/} the days from Thoth <1> to Mechir 13 for a total of 163; then I add 162 for a grand total of 325. I count this off from \Aries, giving 30 to each sign and stop at \Aquarius\, 25\deg. \Mercury\, is there.

\newpage