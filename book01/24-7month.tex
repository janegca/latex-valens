\section{Seven-Month Children (24K,22P)}

We will append another method to determine if the infant spends a full term in the womb, or less—in which case premature death, miscarriages, difficult childbirths, and fatalities, as well as the birth of seven month children, will occur. 

The determination is as follows: in each case I note the date (month, day) of the birth in the year prior to the nativity, and I calculate the \Moon. I note in which sign it is located.
Next I note the date (month, day) of the birth in the year following the nativity (i.e. two years $<$later$>$), and
again I calculate the \Moon. Having done so, I compare its position $<$then to its position$>$ in the prior year. If I find in both years that the moons are trine with the \Moon’s position at the nativity, I forecast that the conception will be carried to term. If in both years the moons are square with the \Moon\xspace at the nativity, the native’s gestation period will be the minimum factor, \textbf{/53P/} 258 days. If the \Moon\xspace of the preceding year is trine and the \Moon\xspace of the following year is square, he will be 269 days in the womb. Conversely if the \Moon\xspace of the preceding year is square and the \Moon\xspace of the following year is trine, he will be in the womb the same 269 days.

If the \Moon\xspace of the preceding year is square and the \Moon\xspace of the following year is turned away, he will
have an eight-month gestation period and will be stillborn. Likewise if $<$the \Moon$>$ of the first year is trine and that of the second year is turned away, the infant will not survive. If in the two years the moons are found to be in no aspect with the \Moon\xspace at the nativity, the infant will be still-born or will be aborted with danger to the mother. If the moons of the two years are in opposition $<$to the \Moon\xspace of the nativity$>$ and are in harmony, the infant will be of seven-month term. If \textbf{/54K/} the \Moon\xspace of the preceding year is in opposition and the \Moon\xspace of the following year is trine (i.e. with the \Moon\xspace at the nativity or with the Ascendant), the infant will be of seven-month term. The same will be true if the \Moon\xspace is square. If the \Moon\xspace of the preceding year is square and the \Moon\xspace of the following year is in opposition, the infant will be of seven-month term. The \Sun\xspace has the same effect when it is in opposition to the sign in which the new \Moon\xspace occurred.

The Ancients wrote about this topic, darkly and mysteriously. We have cast light on it.

\newpage