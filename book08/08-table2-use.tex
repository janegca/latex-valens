\section{Examples of the Previous Procedure Using Table II (7K,8P)}

An example: Trajan year 6, Choiak 1/2, the eighth hour of the night, a conjunction of \Sun\xspace and \Moon\xspace at \Sagittarius\xspace 8° 30'. I take the time from the day and hour of the new moon to the day and hour of the nativity, which is 7 days 14 hours. This amount is one-half of the period between new and full moon (15 days). I take one-half of the hourly magnitude entered at \Sagittarius\xspace 8°, which is 12, for a result of 6. I consider this the “fraction of the hour” figure. I calculate the hours, 9, $<$for a result of 9 x 12=108$>$, and I
add this result, plus the fraction of the hour, plus 259;20 for the total rising time $<$enklima$>$. I find the total to be 376;20, from which I subtract 360° (i.e. one circle), for a result of 16;20. 

Making a note of this, \textbf{/317K/} I consult the table under the total rising time, and I find entered there at \Aries\xspace 29°, the hourly magnitude 16;24. Therefore this figure is operative. 

I add to the 29° another 8° (which I subtract from the \Sun’s degree-position), and \Taurus\xspace 7° becomes the Ascendant. Making a note of this, I consult the first table at \textbf{/304P/} \Taurus\xspace 7°, and I find entered there 20. Now 20 is one-third of 60. I use one-third of the magnitude as follows: since the magnitude 16;24 is entered at \Aries\xspace 29°, the position which indicated the total rising time, I multiply this 16;24 by 12 and get 196;48. I take one-third of this figure and get 65 1/2 years. He died halfway through his 65th year. As for why we subtracted the 8° from the \Sun’s position, and then added, I will explain that in my next discourse.

The following is a new-moon nativity, known to me by hearsay. Those who wish to know the reality of this science will do these things: Vespasian year 7, Epiphi 25/26, the $<$third$>$ hour of the night, klima 3. The \Sun\xspace in \Cancer\xspace 27° 43', the \Moon\xspace in \Pisces\xspace 12° 52', the $<$preceding$>$ full moon was Epiphi 22, the third hour of the day at \Capricorn\xspace 24°. From the day and hour of the full moon to the day and hour of the birth was 3 days 12 hours. This amount is 7/30 of the period from full to new moon (15 days). I subtract these days from the magnitude entered at \Capricorn\xspace 20°, which is 12;20, and the result is 9;12 $<$!$>$. This will be the “fraction of the hour” figure. 

I calculate the hours, 2, $<$for a result of 2 x 12;20=24;40$>$, and add the fraction of the hours, plus the total rising time, 307, for a total of 340;55. I find this magnitude in the table of total rising times at \Aquarius\xspace 29°. I add the 8°, and the Ascendant becomes \Pisces\xspace 7°. Making a note of this, I consult the table at \Pisces\xspace 7°, and I find 26 entered there, which is 13/30 of 60. The hourly magnitude entered at \Aquarius\xspace 29° was very nearly 13. I multiply this by 12, for a result of 156. Next I take 13/30 of this and get 68. He died halfway through his 69th year.

Another example: Hadrian year 18, Phamenoth 2, the fourth hour of the day. The \Sun\xspace in \Pisces\xspace 9° 46', the \Moon\xspace in \Virgo\xspace 9° 40', the full moon was about to occur, since the light of the moon was full. Calculating the fourth hour as the full moon using the rising times, I put the Ascendant at \Taurus\xspace 29°. In
the table at \Taurus\xspace 29° is entered 10, which \textbf{/318K/} is 1/6 of 60. The hourly magnitude of \Taurus\xspace 29° was 17;27. Twelve times this figure is very nearly 210. I take 1/6 of this for a result of 35. He died at 31 1/2 years of age. The last degrees of the chronocratorship suffered a deficiency with respect to the minutes of the \Sun, \textbf{/305P/} when the degrees are calculated with respect to the factor applicable to each year: 4, 16, 17.

Another example: Antoninus Pius year 15, Athyr 25/26, the end of the ninth hour of the night, klima 6. The \Sun\xspace in \Sagittarius\xspace 2° 52', the \Moon\xspace in \Sagittarius\xspace 6° 48', the new moon had just occurred. The distance $<$between the \Moon\xspace and the \Sun$>$ was not great. When the ninth hour was calculated as the hour of the new moon, the results were not correct. Calculating the eighth hour as the new moon, I put the Ascendant at \Leo\xspace 12°. After the addition of 8°, the figure 4 $<$?$>$ was found to be entered next to it in the table. This is 1/15 of 60. The hourly magnitude entered at \Libra\xspace 20° is 15. This amount times 12 yields 180, and 1/15 of this gives 12. He lived that long.
\newpage