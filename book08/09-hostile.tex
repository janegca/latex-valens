\section{The Hostile Places and Stars. The Critical Places With Reference to Table I (8K,9P)}

It is necessary to examine the hostile places and stars, not only with respect to the other stars, but also with respect to the Ascendant, the \Sun, and the \Moon. When these come into opposition during their
transmissions, they indicate critical points and deaths. Take, for example, \Saturn: it is necessary to examine the degrees in opposition to see to which god’s term they belong (as entered in the table). The
native will die when \Saturn\, is in those degrees, is square <with the Ascendant>, or is in signs of the same rising time, depending on which chronocrator is in effect. \mn{Terms} The same must be done for the other stars, because the rulers of the terms of the degrees in opposition are hostile. These stars indicate destruction when they come to the places <of the rulers> or to the places of the same rising time as the Ascendant.

An example\footnote{The terms Valens uses in these examples do not match his own term table or the term tables of the Egyptians, Chaldeans, or Ptolemy.}: \Saturn\, in \Cancer\, 21°, the terms of \Venus. The point in opposition is \Capricorn\, 21°, terms of \Mars, which was at \Taurus\, 27°. When \Saturn\, is there, the native will die. He died in \Virgo\, because it was square, when calculated by degrees…

<\Jupiter> in \Scorpio\, 14°, the terms of \Saturn, and \Taurus\, 14° <the point in opposition> is also the terms of \Saturn, and this star is not hostile to itself. \Leo\, has the same rising time as \Scorpio, and \Leo\, 14° is in the terms of the \Sun\, Therefore \Jupiter, coming to the places of the \Sun\, or coming to a sign of equal rising time, destroyed the native there.

\textbf{/319K/} \Mars\, in \Taurus\, 27°, the terms of the \Sun. The same position in \Scorpio\, is in the terms of the
\Sun, and \mndl no star is hostile to itself. So I investigate \Leo\, \textbf{/306P/} 27° or the sign of equal rising time, which is \Gemini\, according to the hourly intervals. \Gemini\, 27° is in the terms of \Venus. The native will die when \Mars\, is in \Scorpio\, or <\Aquarius>, the signs of equal rising time, or when it is in signs square with these. 

If anyone calculates \Leo\, 27°, he will find it to be in the terms of \Saturn. \Saturn was in \Cancer; so the native will die when \Mars\, is in \Cancer, in \Sagittarius, or in the signs square with them.

\Venus\, in \Scorpio\, 27°, the terms of the \Sun. The point in opposition is \Taurus\, 27°, the terms of the \Sun. Now the star is not hostile to itself, and so I investigate \Scorpio\, 27°, the sign of equal rising time, which is in the terms of \Mercury. The native will die when \Venus\, is in \Virgo, \Mercury’s position, or in the signs square with it. The same procedure should be followed for \Mercury.

In casting horoscopes for patients struck down by illness, it is necessary to examine the <place> in opposition, the stars in the hostile places, and the stars causing the monthly, daily, and hourly critical periods with respect to the degree-position/sign of the \Moon\, in which the opposing star is found.

First we must speak about the construction of the table, so that anyone wishing to derive it handily anew might easily determine its structure. You will find the structure of the table line by line to be as follows: multiply 1/60 of the magnitude given for each day by 30, and the result will be the span of life. No star will either add to or subtract from it.

This is what I mean: someone supposes that an inscription has been found in the recesses of a temple, an inscription whose progressive increase is 2 for each line. Each line of the inscription is 1/60 of the hourly magnitude of the day, and this figure, multiplied <by 30>
gives the length of life. For instance: the beginning is at Thoth <1>. At the first degree is entered 2 (or one-half of 4). The increase is 2 <per degree> until 6°, and when 6° is reached, the number entered there is
12. Then the sequence is broken: at 6°, the breaking of the sequence occurs. The number 14 (which is, of course, the “illumination” of the Moon) is added to the number associated with 6°. At 7° will be 26; at
8°, 28; at 9°, 30. 

Now that 30 has been completed, begin again with 2. So at <10° is entered 2>, at 11° is entered 4; at 12° is entered 6. The \textit{hexad} of 2 has been completed, and again the sequence is broken and 14 is added to 6. The result is 20, to be entered \textbf{/320K/} at the 13° row.

They proceed thus, breaking the sequence at each 6° and adding \textbf{/307P/} 14 to the previous number. If somehow in the middle of the procedure the number 30 is reached (as was shown for the second \textit{hexad} above), we do not add 14, but start again with 2. So the structure from the beginning (i.e. 1°) sets an <upper> limit to the table, 30. Following is the procedure for knowing the first number of each sign, the basis of which is, is order…

<Tables 1 and 2 are to be inserted here.>

\newpage