\section{The Method for Using the Two Appended Tables}

The first table is designed for finding the length of life, and it derives its basis and its method of use from the degree of the Ascendant. The length <of the seasonal hour> (for the correct klima) which is
entered next to the degree found to be in the Ascendant is multiplied by 12. We then take 1/30th of this amount, and we say that the degrees entered there allot that same number years and that the hour of death is that number of years away. 

In the same way, we can take 1/60th of the result of the multiplication by 12 and calculate that each degree of the sign will allot that many years. 

If the degree in the Ascendant is at a connection/node, the native will be short-lived.

Entering the table at the degree of the sign in the Ascendant, we see which number is placed next to it. We determine what fraction of 60 this number is; we take the same fraction of the result of the multiplication by 12; and we consider the answer to be the years of life. 

It is necessary to calculate the figure (=years) entered in the table first as hours, then as days, as months, and as years. In addition, when
the number 2 is entered next to the degree in the table, it is necessary to examine one-half of the time which is associated with the degree, according to the difference of klima and sign.

For example: the number 2 is entered next to \Libra\xspace 1°. Two is 1/30th of 60. 1/30th of 180, the magnitude <=total rising time of the arc beginning with> \Libra\xspace 1°, is 6. This figure (6) is placed next to each degree. If we calculate with this many years <per degree>, the 30° will allot 180 years, an impossible length of life for a person. So if we take 1/60th of 180, we will get 3 as the amount which 1° <of \Libra> will allot. Three times 30° is 90 (or one-half of 180 is 90—which is the same thing): we can say that \Libra\xspace allots a maximum of 90 years, according to the applicable degree of its magnitude.

Likewise for the rest of the signs: we multiply the magnitude entered next to each degree by 12, then take 1/60 (or 1/2) of it to find \textbf{/288P/} the minimum or the maximum years. Each degree of each sign
\textbf{/301K/} has a different time in <the table’s> progressive increase, and for this reason the seconds and the minutes of the hours and the rotation of the degrees have great effects.

I have written for those who wish to learn every systematic procedure. Each of the other astrological compilers has worked out his own complex procedures, but has not published his solutions, since each is
secretive and begrudging, and neglected his readers. I, however, have investigated with much toil and long experience, and have published <my system>. This seems to be my greatest achievement, to explicate the ideas of others which have been buried in mystery. I myself could have compiled my many procedures using a fog of words, but I did not want to show myself to be like those babblers. It would be laughable to begin speaking against someone without recognizing first my own faults. Therefore if you find me speaking very often about my generosity and openness, please forgive my words. I suffered much, I
endured much toil, I was cheated by many men, and I spent money that seemed to me to be inexhaustible because I was persuaded by mountebanks and greedy men. Nevertheless because of my endurance and my love for systematic knowledge, I outlasted them all. If my readers recognize the accuracy of these systems, they will give us praise with delight. Others, because of their stupidity, will envy and malign us, and they may be exalted by the illumination of mystical and secret things, and they will steal some procedures from my compilation. So on such men I place dire curses, which I think they will suffer.

Let the readers of our collected works, works which explicate all procedures, not say: “This procedure is from the King, this other is from Petosiris, that one is from Critodemus, etc.” Instead let them know
that these men propounded their art in an obtuse and recondite fashion, and thereby showed that their science lacked a true foundation. We on the other hand supplied solutions, and not only revived this dying art, but also banked glory for ourselves and initiated other worthy men, attracting them not with the lure of money, but by recognizing them to be scholars and enthusiasts. We too have been controlled by this type of Fate.

\index{length of life!three-sign method}
\textbf{/302K/} Now let our discussion return to this topic, length of life. All of the previously outlined methods are accurate and \textbf{/289P/} tested in their own system, including the proven and amazing “three-sign” method. It comprises the following: I accurately determine the degree-positions of the \Sun\, and the \Moon\, relative to the degree-position of the Ascendant at the nativity. I enter the table below and I determine (using the method described below) the ending point of the three signs, starting with the \Sun’s position. Then I go to the sign in the Ascendant and determine in which degree it <the ending point> is located <in the Ascendant sign>, and I make this the solar gnomon. \index{gnomon!solar}

Next do likewise <for the number> entered at the \Moon’s sign: apply the degrees of the solar gnomon to it and again see which sign this place belongs to. Looking for that sign in the Ascendant sign, I
consider the resulting <degree-position> to be the lunar gnomon. \index{gnomon!lunar}

Next I determine whether the lunar or the solar gnomon is greater: if the \Sun’s gnomon is greater, the hour requires the addition of the number of signs between them <in the table>; if the \Moon’s, the
subtraction <of the same number>. This position will be the required, scientifically established Ascendant. Therefore let no one be puzzled if the Ascendant is found in a sign different from that which was originally assumed to be the Ascendant. 

After the addition or subtraction, I note the degree-position of the
Ascendant, I enter the table of rising times for the three-sign system, and I investigate how many years are written beside the degree which was found—taking the klima into consideration. Then I make the
prediction. If the \Sun’s, the \Moon’s, or the Ascendant’s position is out of place by one or two degrees, matters must be judged as discordant.

If someone should wish to test this procedure, let him move up one or two degrees and add this to the position of the \Sun\, or \Moon\, (if it seems to be in error), or let him calculate another Ascendant, then
continue the operations in the same way. If he does so, he will find the way. It is possible to prove this procedure using nativities of those who have already died. Do not trust <astrologers> who present
erroneous, hearsay nativities and who are fated to blunder; instead establish a firm foundation so as not to go wrong.

\index{nativity!basis}
Of \mn{Basis} necessity the ray-casting and the conjunctions of the malefics with the \Sun, the \Moon, and the Ascendant must be taken into account, as well as the house ruling relationships, because it is from these
that the factors of the basis <of the nativity> are apprehended.

\textbf{/303K/} We calculate the previously mentioned times of the factors for those who die at birth, or who live only a short time, by calculating first the hours, then the days, months, and years. After completing the first factor (the years), we calculate the second and the third in the same way, first hours, then days and months, and we add the result to the first factor. If we find a nativity of a long-lived person in the signs of short rising times, \textbf{/290P/} we combine the first or third factor, or the second and the third, or the all four together plus the remaining figures entered next to the degree; then we make the prediction.

Again, to know which of the three factors is dominant, calculate as follows: entering the table at the degree in the Ascendant, I note how many years is written next to the first factor. Then I enter the table at
the degree just opposite, and I investigate the number which is written next to the years and months I found. I add this number to the position of the Ascendant in degrees, then I count off from the sign in the
Ascendant—or from the \Moon\, if it is at or following an angle. 

I will look at the rays of the malefics where the count stops to see if they are hindering; if so, it <the ray> will be the anaereta of the cycles. Do the same for the second and third factors.

Alternatively, I will derive the years from the <degrees> of the contact, from the signs, and from the stars, and I will see if the house ruler of “Dissolution” controls the three in any way at all. (The XII Place
is “Dissolution.”) \index{places!12@12th}

It is also necessary to examine the critical time-relationships as we have explained them <previously>. If the number falls a little short, part of the span of years will be deducted. Often the critical
point will occur early or will bring death after the calculated number of years. One must note in which row of the table the Ascendant is located, whether it has 2, 3, or 4 numbers, and so use all the numbers together for the years. For example, say the position of the Ascendant is \Cancer\xspace 8°, which indicates the place of \Taurus. <\Cancer\xspace 9°> is also operative in the same row, and so this degree (9°) will control every operation. Likewise, also in \Cancer, the degrees from 25° to 28° are in the place of \Aquarius. So here 26°
and 27° are also operative.

\textit{<The table of place equivalences is missing in the manuscripts.>}
\newpage