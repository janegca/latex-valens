\section{Sample Nativities}

For example, consider the following nativity: Nero year 1, Athyr 2, the third hour of the day. The \Sun\xspace in \Scorpio\xspace 10°, the \Moon\xspace in \Aquarius\xspace 30°, the Ascendant in \Sagittarius.

I look in the table at 10° of the \Sun’s position, i.e. in the column under \Scorpio, and I find \Pisces\xspace entered there. I transfer to \Pisces\xspace the 10° of the \Sun’s position, and I find \Libra\xspace entered there. I look <for \Libra> in the Ascendant, \Sagittarius.
I find it at 14° and 15°. This will be the solar horoscopic gnomon, and I make a note of it. 

Next \textbf{/292P/} I turn to the \Moon’s sign, \Aquarius, and I see which sign is entered next to 30°. I find \Aquarius\xspace again. So
I transfer to \Aquarius\xspace the solar gnomon, 14°, which I just discovered, and I find entered at that figure the sign \Sagittarius. I investigate this in the sign of the Ascendant, and I find it at 1°, 2° and 3°. This then becomes the lunar gnomon, and it is in the first row <of the table>. The solar gnomon is in the sixth row.

Since the solar gnomon is greater, and since there are 4 rows (which equal 4°) intervening <between row one and row six>, I add this amount to the 14° of the previously determined solar gnomon. 

So now the Ascendant is at \Sagittarius\xspace 18°. Having found this, I enter the <table of> apogonia <at \Sagittarius\xspace 18°>, and I find (under klima 6) in the third factor, 73 years. The native died in his 73rd year. Now if the lunar gnomon had been greater, I would have subtracted from the solar gnomon (14°) the addition/subtraction factor (4°). Then the Ascendant would have been at Sagittarius 10°.

Another example: klima 4, Titus year 1, Phamenoth 20/21, \Sun\xspace in \Pisces\xspace 29°, \Moon\xspace in \Capricorn\xspace 27°, Ascendant in \Scorpio. At 29° <in the column> under \Pisces, the position of the \Sun, is entered
\Cancer. At \Cancer\xspace 29° is entered \Capricorn. I investigate this sign <\Capricorn> in \Scorpio, the sign in the Ascendant, and I find it at 4° and 5°. This, then, is the solar horoscopic gnomon, and I make a note of it. 

Next I go to the \Moon’s sign, \Capricorn, and at 27° I find \Aries. I investigate \Aries\xspace 4° and 5°, the solar gnomon, and in row 2 (at \Aries\xspace 4° and 5°) I find the sign \textbf{/306K/} \Capricorn. I investigate this sign in the Ascendant, \Scorpio, and I find it at the same position, 4° and 5°, in row 2. So the solar and the lunar gnomons agree, and it is clear that the Ascendant needs no addition or subtraction. 

I enter the <table of> rising times at 4°, and I find (for klima 4) for the Ascendant, \Scorpio, in the third factor, 72;33 years.
The native died at age 72 1/2.

Another example: Trajan year 17, klima 2, <Mesore> 17/18, the fourth hour of the night. The \Sun\xspace in \Leo 22°, the \Moon\xspace in \Taurus\xspace 14°, the Ascendant in \Aries. In the table at \Leo\xspace 22°, the \Sun’s position, I find \Virgo. At \Virgo\xspace 22° I find \Cancer. This sign, \Cancer, I investigate in the Ascendant, \textbf{/293P/} \Aries, and I find it at 26°. So this will be the solar horoscopic gnomon. 

In a like manner, since \Taurus\xspace 14° is the \Moon’s position, I next enter \Taurus\xspace and find \Gemini\xspace <at \Taurus\xspace 14°>, then I transfer the 26° of the solar gnomon to \Gemini\xspace and find \Aquarius\xspace entered there <at \Gemini\xspace 26°>. I investigate <\Aquarius> in \Aries, the Ascendant, and I find it at \Aries 21°. So this is the lunar gnomon. Between it and the solar gnomon is one row, which means 1°. I add this to the previously determined <Ascendant, \Aries> 26°, for
a total of 27°. The Ascendant will be at \Aries\xspace 27°. I enter the <table of> rising times for \Aries\xspace in klima 2, and I find entered at 27°, in the second factor, 42;36 years. The native died in his 42nd year.

Another example: Trajan year 18, Payni 14, the fifth hour of the day, klima 1. The \Sun\xspace in \Gemini\xspace 20°, the \Moon\xspace in \Taurus\xspace 27°, the Ascendant in \Virgo. At <\Gemini> 20°, the \Sun’s position, is entered \Cancer; at \Cancer\xspace 20° is entered \Virgo. I investigate \Virgo\xspace in the Ascendant <\Virgo>, and I find it at 1°, 2°, and 3°. This will be the solar gnomon. 

Next, at <\Taurus> 27°, the \Moon’s position, is entered
\Aries. I investigate in \Aries the solar gnomon (1°, 2°, 3°), and I find there \Leo. I look for \Leo\xspace in \Virgo, and I find it at 4° and 5°. <This is the lunar gnomon.> There are no intervening rows, and so the Ascendant is \Virgo\xspace 1°. I enter <the table of> rising times for klima 1, and I find at \Virgo\xspace 1°, in the second
factor, 39;36 years. The native died in his 40th year.

Another example: Hadrian year 12, Athyr 1, the ninth hour of the day, klima 1. The \Sun\xspace in \Scorpio\xspace 8°, the \Moon\xspace in \Capricorn\xspace 17°, the Ascendant in \Pisces. At \Scorpio\xspace 8° is entered \Taurus, at \Taurus <8°> is entered \Libra. \textbf{/307K/} I look for \Libra\xspace in \Pisces, and I find it at \Pisces\xspace 9° <the solar gnomon>. 

Next at \Capricorn\xspace 17°, the lunar position, is entered \Taurus. At \Taurus\xspace 9°, the degree of the solar gnomon, is entered \Virgo. In \Pisces\xspace I find \Virgo\xspace at \Pisces\xspace 15° <the lunar gnomon>. There are two intervening rows, which equal 2°. I subtract this from the solar gnomon, 9°, since the lunar gnomon is greater, and the result is <\Pisces> 7°. This is the Ascendant in \Pisces, and next to it is entered (for klima 1) in the second factor, 26;43 years. The native lived 27 years.

Another example: Vespasian year 1, Epiphi 22, the fifth hour of the day, klima 6. the sun in \Cancer\xspace 28°, the \Moon\xspace in \Scorpio\xspace 3°, the Ascendant in \Libra. At \Cancer\xspace 28°, \textbf{/294P/} the \Sun’s position, is entered \Aquarius; at \Aquarius\xspace 28° is entered \Aquarius\xspace again. In \Libra\xspace I find this <\Aquarius> entered at 4°, 5°, 6°
<the solar gnomon>. 

Next at \Scorpio\xspace 3°, the \Moon’s position, is entered \Scorpio; so in this sign I look at 4° and 5°, the solar gnomon, and I find \Capricorn\xspace entered there. I find <\Capricorn> in \Libra\xspace at \Libra\xspace 22° <the lunar gnomon>. Seven rows are found to be intervening between the lunar and the solar
gnomons, and I subtract this figure from \Libra\xspace 4°, for a result of \Virgo\xspace 27° as the Ascendant. At this position for klima 6, in the second factor, is entered 81 years. The native died in his 81st year. 

Another example: Trajan year 18, Thoth 14/15, the ninth hour of the night. The \Sun\xspace in \Virgo\xspace 22°, the \Moon\xspace in \Aquarius\xspace 4°, <the Ascendant in \Leo. At \Virgo\xspace 22°, the \Sun’s position> is entered \Cancer; at \Cancer\xspace <22° is \Sagittarius. I find this sign in \Leo\xspace at> 1°, 2°, 3°. <Next at \Aquarius\xspace 4°, the \Moon’s position,> is entered \Sagittarius. This sign likewise I find in \Leo, the Ascendant, at the same degrees, 1°, 2° 3°. So \Leo\xspace 1° is the Ascendant, next to which is entered (for klima 1) one year. The native died in his first year.

Another example: Antoninus year 5, Tybi 28/29, the eleventh hour of the night. The \Sun\xspace in \Aquarius\xspace 6°, the \Moon\xspace in \Taurus\xspace 28°, the Ascendant in \Capricorn\xspace 6°. At \Aquarius\xspace 6° is entered \Virgo, at \Virgo\xspace 
<6°> is entered \Cancer. I look for this sign in \Capricorn, the Ascendant, and I find it at 1° and 2° <the solar gnomon>. 

At \Taurus\xspace 28°, the \Moon’s position, is entered \Aries. Transferring to \Aries\xspace 1° and 2°, I find \Leo. Then I find \Leo\xspace in \Capricorn\xspace at 10° and 11° <the lunar gnomon>. There are three intervening rows, so I subtract 3° from \Capricorn\xspace 1° for a result of \Sagittarius\xspace 28° <as the Ascendant. For klima …, in the first factor, is entered 30+.> (These are months.) The native died in his third year.

Another example: Antoninus year 15, Tybi 12, the first hour of the day. The sun in \Capricorn\xspace 20°, the \Moon\xspace in \Gemini\xspace 28°, the Ascendant in \Capricorn. \textbf{/308K/} At \Capricorn\xspace 20° is entered \Libra; at \Libra\xspace <20°> \Pisces, and I find this sign in \Capricorn, the Ascendant, at 29° and 30° <the solar gnomon>. 

Next at \Gemini\xspace 28°, the \Moon’s position, I find \Aquarius. At \Aquarius\xspace 29° (the solar gnomon) I again find \Aquarius. I find this sign in \Capricorn\xspace \textbf{/295P/} at 23° and 24° <the lunar gnomon>. There are two intervening rows, and I add 2° to \Capricorn\xspace 28° for a result of\Aquarius\xspace 1° as the Ascendant. For klima 6, 0;44 years is entered at Aquarius 1°. The native died in his first year.

Another example: Antoninus year 21, Athyr 28/29, the third hour of the night. The \Sun\xspace in \Sagittarius\xspace 6°, the \Moon\xspace in \Aquarius\xspace 3°, the Ascendant in \Cancer. At \Sagittarius\xspace 6° is entered \Cancer. <In \Cancer, \Aries. In \Cancer,> the Ascendant, I found it <\Aries> also at 6° <the solar gnomon>.

Likewise at \Aquarius\xspace <3°>, the \Moon’s position, is entered \Cancer; at \Cancer\xspace 3° is entered \Virgo. In \Cancer, the Ascendant, I find \Virgo\xspace at 20° <the lunar gnomon>. There are five intervening rows, and so I subtract 5° from \Cancer\xspace 6°, for a result of \Cancer\xspace 1° as the Ascendant. Next to it is entered 1;2. The native lived 1 year.

Another example: klima 6, Trajan year 8, Pharmouthi 26. Most of the sources report the Ascendant as \Cancer, since they want the benefics to be at the angles, but we have found from our calculations that the Ascendant was in \Gemini. The \Sun\xspace in \Taurus\xspace 3°, the \Moon\xspace in \Sagittarius\xspace 21°. At \Taurus\xspace 3°, the \Sun’s position, is entered \Aquarius. At <\Aquarius\xspace 3° is found…> I found it at \Gemini\xspace 23°. <At \Sagittarius\xspace 21°, the \Moon’s position, is entered… I found it in> \Scorpio. In \Gemini\xspace I found this sign <\Scorpio> at 18° <the lunar gnomon>. I add 1° for the one intervening row to <\Gemini> 23° <the solar gnomon>, for a result of \Gemini\xspace 24°, which is the Ascendant. Entering the table at klima 6, I found 21;55. The native lived 22 years 45 days.

One who pays attention to these collected rules and methods for calculating critical times will not go astray. In the same way that the house rulers, <vital sectors>, or the ray-casting <stars> sometimes produce an increase, sometimes a decrease of years, proportionate to the aspects of the benefics, so this method too
can produce an <increase or a decrease>. There is also the following “further analysis” researched by us with great labor. We will append it with examples.

Hadrian year 9, Phamenoth 28/29, the third hour of the night. The \Sun\xspace in \Aries\xspace 6°, the \Moon\xspace in \Aries\xspace
30°, the Ascendant in \Scorpio. At \Aries\xspace 6° is entered \Sagittarius, at \Sagittarius\xspace 6° \Cancer. I find this sign at \Scorpio\xspace 22°. This \textbf{/309K/} then will be the solar horoscopic gnomon. 

Consulting the table at <\Aries> 30°, the \Moon’s position, I find \Scorpio. In \Scorpio\xspace I found 22°, the degree number of the solar gnomon, in the same place. So the solar and the lunar gnomons are in accord. \textbf{/296P/} So we make our further analysis as follows: I take the remaining degrees of the sign in the Ascendant, 8°, plus the degrees of the \Sun, 6°, plus those of the \Moon, 30°, for a total of 44°. I count this off from the \Moon’s sign and the count
stops at \Taurus\xspace 14°, the place of \Gemini. I look for this sign <\Gemini> in \Scorpio, the Ascendant, and I find it at 12°. There are three rows between <\Scorpio> 12° and 22°, and so I add 3° to 22°. \Scorpio\xspace 25° is definitely the Ascendant. At this degree (for klima 6) in the first factor is 31;28 years. The native died
in his 31st year. Most <astrologers>, however, avoiding the situation of \Mars\xspace as a yoke-fellow <\Scorpio\xspace is a house of \Mars>, reported the Ascendant to be \Libra.

Another example: Hadrian year 15, Epiphi 16, the third hour of the day. The \Sun\xspace in \Cancer 20°, the \Moon\xspace in \Gemini\xspace 25°, the Ascendant in \Virgo. At \Cancer\xspace 20° is entered \Virgo, At \Virgo <20°> \Cancer. I find this sign in <\Virgo>, the Ascendant, at 20° <the solar gnomon>.

Likewise at <\Gemini> 25°, the \Moon’s position, is entered \Aquarius. At \Aquarius\xspace 20°, the solar gnomon, is found \Aries. I find this sign in \Virgo\xspace at 29° <the lunar gnomon>. Since the lunar gnomon is greater than the solar, and since there are two intervening rows, I subtract 2° from 20°, for a result of \Virgo\xspace 18°. 

We make the further analysis as follows: I take the remaining degrees of \Virgo, 12°, plus the degrees of the \Sun, 20°, plus those of the \Moon, 25°, for a total of 57°. I count this off from the \Moon’s sign; the count stops at \Cancer\xspace 27°, which is marked as the place of \Aquarius. Looking for \Aquarius\xspace in \Virgo, I find it at 26°. There are <2> rows between 26° and 18°, and I subtract this from 18°. \Virgo\xspace 16° is the result. For klima 2, the figure 21;20 is entered at \Virgo\xspace 16°. The native died in his 21st year.

Another example: Nero year 14, Thoth 14/15, the eleventh hour of the night. The \Sun\xspace in \Virgo 25°, the \Moon\xspace in \Aries\xspace 10°, the Ascendant in \Virgo. At \Virgo\xspace 25° is entered \Capricorn, at \Capricorn\xspace <25°> is entered \Scorpio. I look for this sign in \Virgo, and I find it \textbf{/310K/} at 11° <the solar gnomon>. 

Next at \Aries\xspace 10°, the \Moon’s position, is entered \Virgo; at \Virgo\xspace 11°, \Scorpio. <The lunar and solar gnomons
are the same.> I take the remaining degrees in \Virgo, 19°, plus the degrees of the \Sun, 25°, plus those of the \Moon, 10°, for a total of 54°. \textbf{/297P/} I count this off from the \Moon’s sign, and the count stops at \Taurus\xspace 24°, at which \Cancer\xspace is entered. I find this sign <\Cancer> at \Virgo 20°. There are three rows between <Virgo> 20° and 11°, and I subtract this from 11°, for a result of 8°. The Ascendant was \Virgo\xspace
8°. Entering the <table of> rising times, I find (for klima 1) in the third factor, the years 86. The native died at that age.

Another example: Trajan year 12, Payni 8, the second hour of the day. The \Sun\xspace in \Gemini\xspace 13°, the \Moon\xspace in \Capricorn\xspace 4°, the Ascendant in \Cancer. At \Gemini\xspace 13° is entered \Libra, at \Libra <13°>, \Virgo. I find that sign <\Virgo> in \Cancer, the Ascendant, at 20° <the solar gnomon>. 

Next at \Capricorn\xspace 4°, the \Moon’s position, I find \Virgo. At \Virgo\xspace 20°, the solar gnomon, I find \Cancer. I find that sign in the Ascendant at 4°. This is the lunar gnomon. Since the solar gnomon is greater than the lunar, I add the intervening six rows to 20°, for a result of \Cancer\xspace 26°, which is the <provisional> position of the Ascendant. Next I take the remaining 4° of \Cancer, plus the 13° of the \Sun, plus the 4° of the \Moon, for a total of 21°. I count this off from \Capricorn; the count stops at 21° of the same sign. I investigate this position in \Capricorn, and I find \Libra. I look for \Libra\xspace in \Cancer, and I find it at 18°. There are two rows between this figure and 26°, and I add this again to 26° for a result of \Cancer \xspace28° as the Ascendant. That was the further analysis. 

Entering the table for klima 6, I find 30;25 for the first factor, 67;55 for the second, and 110;28 for the third. We calculate the first factor to be the years, the second and third to be the months; these total approximately 181 months, or 15 years, plus the 30 years of the first factor. Together they total 45 years. The native died in his 45th year.
Just as I have said, I calculated first the hours entered beside the factors, then the days, the months, and the years.

Another example: Domitian year 2, Pachon 20, the first hour. The \Sun\xspace in \Taurus\xspace 27°, the \Moon\xspace in \Virgo\xspace 20°, the Ascendant in \Gemini. At \Taurus\xspace 27° is entered \Aries, at \Aries\xspace <27°> \Cancer. I find that
sign in \Gemini\xspace at 20° <the solar gnomon>. 

At \Virgo\xspace 20°, the \Moon’s position, is entered \textbf{/311K/} \Cancer, at \Cancer\xspace 20° (the previously determined solar gnomon) is entered \Virgo. I find that sign in \Gemini\xspace at 4°
<the lunar gnomon>. There are six rows between <\Gemini> 4° and 20°; so I add 6° to the 20° for a result of \Gemini\xspace 26°. Next I take the remaining degrees of \Gemini, 4°, plus the 27° of the \Sun, plus the 20° of the \Moon, for a total of 51°. I count this off \textbf{/298P/} from the \Moon’s sign, and the count stops at \Libra\xspace 21°, in the place of <\Cancer?>. I find that sign at \Gemini\xspace 11°/19°<?>. There are five <!> intervening rows; so I add 5° to \Gemini\xspace 26°, for a result of \Cancer\xspace 1° as the Ascendant. For klima 4, I find 71 years for the third factor—and this was his length of life.

Another example: Domitian year 5, Athyr 24, 5 hours <of the day>. The \Sun\xspace in \Sagittarius 3°, the \Moon\xspace in \Gemini\xspace 4°, the Ascendant in \Pisces. At \Sagittarius\xspace <3°> is entered \Sagittarius, again at \Sagittarius\xspace 3°, \Sagittarius. I find this sign <\Sagittarius> in \Pisces\xspace at 11° and 12° <the solar gnomon>. 

Next at \Gemini\xspace 4°, the \Moon’s position, is entered \Virgo, at \Virgo\xspace 11° is entered \Scorpio. I find this sign in \Pisces\xspace at 26° <the lunar gnomon>. There are five intervening rows, and I subtract 5° from 11°, for a result of \Pisces\xspace 6°. Now I take the remaining 24° <of \Pisces>, plus the 3° of the \Sun, plus the 4° of the \Moon, for a total of 31°. I count this off from \Gemini, and <the count> stops at \Cancer\xspace 1°, the place of \Scorpio. I find this sign in \Pisces\xspace again at 26°. Seven rows intervene <between \Pisces\xspace 6° and 26°>. I subtract this amount <7°> from \Pisces\xspace 6°, for a result of \Aquarius\xspace 29°, which is the Ascendant. For klima 4, at \Aquarius\xspace 29° is entered 22;33 for the first factor and 42;27 for the second, a total of 65. He died in his
65th year.

Another example: Titus year 2, Choiak 1, 91/2 hours <of the day>. the \Sun\xspace in \Sagittarius\xspace 8°, the \Moon\xspace in \Taurus\xspace 27°, the Ascendant in \Taurus. At \Sagittarius\xspace 8° is entered \Cancer, at \Cancer\xspace <8°>, \Taurus. I find this sign <\Taurus> in itself, in the Ascendant, at 1°, 2°, 3° <the solar gnomon>. 

At \Taurus\xspace 27°, the \Moon’s position, is entered \Aries; at \Aries\xspace 1°, 2°, 3° is entered \Leo. I find this sign in \Taurus at 19° <the lunar gnomon>. There are six intervening rows. I subtract this amount from \Taurus\xspace 1° for a result of \Aries\xspace 25°. The remaining degrees of \Aries, 5°, plus 8° of the \Sun, plus 27° of the \Moon\xspace total 40°. I count this off from the \Moon’s sign, and <the count> stops at \Gemini\xspace 10°, the place of \Sagittarius. I find this sign in \Aries\xspace at 6° and 7°… There are six<?> rows between this position and 25°. I add this \textbf{/312K/} to \Aries\xspace 25°. The Ascendant is \Taurus\xspace 1°. For the klima of Babylon, 0;48 is entered for the first factor, 24;<56> for the second, and 53;4 for the third. I added the three factors to get 78;48. He lived 77;42 years.

\textbf{/299P/} Another example: Antoninus year 14, Mechir 23, the ninth hour of the day. The \Sun\xspace in \Pisces\xspace 3°, the \Moon\xspace in \Leo\xspace 13°, the Ascendant in \Cancer. At <\Pisces> 3°, the \Sun’s position, is entered \Pisces, <at \Pisces\xspace 3° is entered \Pisces\xspace again, and> I find that sign in \Cancer\xspace at 10° <the solar gnomon>.

Likewise at \Leo\xspace 13°, the \Moon’s position, is entered \Aquarius, at <\Aquarius> 10°, \Libra. I find this sign in \Cancer\xspace at 18° <the lunar gnomon>. There are two intervening signs; so I subtract 2° from 10° for a result of <\Cancer> 8°. I take the remaining degrees of \Cancer, 22°, plus 3° for the \Sun, plus 13° for the \Moon, for a total of 38°. I count this off from the \Moon’s sign, and the count stops at \Virgo\xspace 8°, a place of \Libra. I find this sign in \Cancer\xspace at 18°. There are three signs between this position and 8°. I subtract this <3°> from 8°, and the result is \Cancer\xspace 5° <as the Ascendant>. 5;30 is entered there. He died in his sixth year.

Another example: Hadrian year 5, Pachon 23/24, the tenth hour of the night. <The Ascendant was said to be> in \Capricorn, but we found it to be in \Aquarius\xspace when calculated as follows. The \Sun\xspace in \Taurus\xspace 29°, the \Moon\xspace in \Scorpio\xspace 15°. At \Taurus\xspace 29° is entered \Scorpio, at \Scorpio\xspace <29°>, \Libra. I find this sign in \Aquarius\xspace at 10° <the solar gnomon>. 

Next at <\Scorpio> 15°, the \Moon’s position, is entered
\Virgo, in \Virgo\xspace at the previously determined solar gnomon, <10°>, is entered \Libra. I find this sign in \Aquarius\xspace at the same degree-position. <The lunar and the solar gnomons are the same.> The remaining degrees in \Aquarius, 20°, plus the 29° of the \Sun, plus the 15° of the \Moon total 64°. I count this off from
the \Moon’s sign and <the count> stops at \Capricorn\xspace 4°, a place of \Virgo. I look for \Virgo\xspace in \Aquarius, the Ascendant, and I find it at 6°. There is one row <between \Aquarius\xspace 6° and 10°>, and I add this 1° to <\Aquarius> 10°, and find the Ascendant to be \Aquarius\xspace 11°. At this position (for klima 2) is entered 31;25 in the second factor. He died at the end of his 32nd year.

Be aware that the given Ascendant must not be assumed to be correct in every case, particularly for those born at night or during the winter season when the sun-time can only be estimated because of the cloudiness of the sky. In such cases, evaluate according to the calculated Ascendant and the signs on either side <of it>. If you wish to test the effects of the sign rising just before <the Ascendant>, it will be necessary to subtract from the \Moon’s position at the calculated time \textbf{/313K/} an amount of time correct for the <particular> sign and hour, using the \Moon’s daily motion. For the sign which follows the Ascendant, add the correct amount to the \Moon’s position. When done in this way, the analysis will be considered infallible.

\textbf{/300P/} An example: Hadrian year 3, Phamenoth 29/30, 11/2 hours of the night. The \Sun\xspace in \Aries\xspace 7°, the \Moon\xspace in \Pisces 2°, the Ascendant in \Scorpio. This sign by itself does not reveal the underlying number of years, but the number is found from the results of the forecast and from the analysis of \Sagittarius\xspace in \Scorpio\xspace as follows: at \Aries\xspace 7° is entered \Sagittarius, at \Sagittarius\xspace 7° is entered \Cancer. I investigate this sign <\Cancer> in \Sagittarius, which must be the Ascendant if the real one is to be determined, clearly at the same degree-position, 6° <same row as 7°>. Next at \Pisces\xspace 2°, the \Moon’s position, is entered \Pisces; at \Pisces\xspace 6° is entered \Taurus. <I find> this sign in \Sagittarius at 12°. There is one intervening row, and I subtract this amount from <\Sagittarius> 6°, for a result of \Sagittarius\xspace 5°. Next I take the remaining degrees in \Sagittarius, 25°, plus 7° for the \Sun, plus 2° for the \Moon, for a total of 34°. This I count off from the \Moon’s sign, and the count stops at \Aries\xspace 4°, a place of \Capricorn. I find \Capricorn\xspace in \Sagittarius\xspace at 24°. There are seven rows between this figure and <\Sagittarius> 5°, and I subtract this 7° from \Sagittarius\xspace 5° for a result of \Scorpio\xspace 28°, which is the Ascendant. For klima 2 <at \Scorpio\xspace 28°> is entered as the first factor 33;<36>. He died at the end of his 33rd year.

The preceding detailed analysis being as described, we can find another method occasionally suitable for a few nativities. We will append it, lest anyone become entangled and hence reject this whole
procedure. 

If the luminaries are found in square, in opposition, or in the same sign (i.e. the moon with the sun), we calculate as follows: for example, the \Sun\xspace in \Sagittarius\xspace 30°, the \Moon\xspace in \Sagittarius\xspace 25°, the Ascendant in \Virgo. At \Sagittarius\xspace 30° is entered \Aries, at \Aries\xspace 30°, \Scorpio. I find this sign in \Virgo\xspace at 11°. This is the solar gnomon. 

Likewise at \Sagittarius\xspace 25°, the \Moon’s position, is entered \Scorpio.  I investigate \Scorpio\xspace 25° (the same degree-position), and I find there \Gemini. I look for <\Gemini> in \Virgo, and I find it \textbf{/314K/} at 26°. This will be the lunar gnomon. There are five rows between this figure and the solar gnomon. I subtract this 5° from <\Virgo> 11°, for a result of \Virgo 6° as the Ascendant. 

<Another example:> …For klima 7, 71 years is entered there, which was the length of his life.

Another example: the \Sun\xspace in \Sagittarius\xspace 3°, the \Moon\xspace in \Sagittarius\xspace 7°, the Ascendant \textbf{/301P/} in \Libra. At \Sagittarius\xspace 3° is entered \Sagittarius, and I find this sign in \Libra\xspace at 11°. This is the solar gnomon. 

Likewise at \Sagittarius\xspace 7°, the \Moon’s position, is entered \Cancer; at \Cancer\xspace 7° (the same degree-position again) I find \Aries. I find this sign in \Libra\xspace at 18°. This is the lunar gnomon. There are two intervening rows. I subtract this 2° from the solar gnomon, 11°, for a result of <\Libra> 9°. So \Libra\xspace
9° is the Ascendant…

…with the \Sun\xspace and the \Moon\xspace together. These hold everything together; they control the phases of the stars whirling, each in its own way, and alternating their effects. But in our system, there are 360° in the circle of the heavens which the \Sun, the cosmocrator, gallops over in the course of one year, allotting a fixed
time to each person, but changing the types of death. For it is possible to see many men dying in the same year, not all, however, on the same day or hour nor from the same affliction or doom. The \Sun\, traverses one degree in one 24-hour period, but it does not bring the same things to those born during that time period because of the variation of minutes and hours. In fact, its position, its hours, and its minutes cause a great variation in the addition and subtraction of years, particularly when it comes at the sequence breaks. For example: at \Leo\xspace 12° is entered 6, a magnitude which indicates 20 years 6 months. At 13° of the same sign, since it is a sequence break, is entered 20, which indicates 75 <!> years. Consider the same to be true for the rest of the signs. As the result of a jump from the lesser to the greater numbers in the sequence, the native may become long-lived, or vice versa, as a result of a jump from a large to a small number, he may become short-lived. There are sequences which give men an average lifespan, when the number located there is an intermediate value. It is from these considerations that one can make the determination for twins, who are born at very nearly the same hour.

For every nativity the solar and the lunar degree-positions must be closely observed to see if they are coming to a node. (If so, they cause a lack of success in enterprises; even more, they also involve men in violent dooms and afflictions \textbf{/315K/} and bring a death that is extraordinary, far-famed, sudden, and unexpected. The diseases of these men are dangerous and incurable.) Consequently, the basis of this method is secure and infallible; the factors however can be in error because of <incorrect> degree-positions of the luminaries and the Ascendants. Therefore \textbf{/302P/} the examinations and the calculation must be done with all care and precision, since this science promises nothing casual or ordinary, but rather divinity and immortality.

Since the topic of long- and short-lived men has been raised, we will go on with it because of its strangeness and the disbelief of error. We want our treatise to remain unassailable. Whenever the degree in
the Ascendant is found to be in the correct sequential order, (i.e. by the progressive addition of 2°), there will rarely be any error about the length of life. If it is off by one degree, there will not be a great difference in the years. However, when it comes at the break in the sequence or at the 30° point, there is need of much caution because of the extreme variation in the years. For example: at \Cancer\xspace 27° is entered 104 years; at \Cancer\xspace 28°, <6> years. So here is a great difference, and one can imagine that someone born at
these degrees will live or die more or less time than that indicated for the exact degree.\footnote{Riley: This statement is true: the second place with its minutes indicates which are the long-lived or short lived
of those nativities born not at the exact degree. - marginal gloss} This procedure seems to be unbelievable, but the scientific, exact degree of the \Sun\, manifests its influence, allots the fixed number of years, and dulls the persuasiveness of error. The numbers associated with the 30° segments in the summer hemisphere should be closely observed.

In all cases, some men will live the lifespans <indicated there>. But it is <im>possible for very many to live the periods entered in the winter hemisphere, e.g. 91 or 75. It is necessary to calculate from the equinoctial signs, \Aries\xspace and \Libra, the additions and subtractions of the years according to their corresponding magnitudes, and it is necessary to know the number of years for each and every sign. It is not possible for \Libra\xspace <?> to allot more than 91 years, nor \Capricorn\xspace more than 75. If some conscienceless <astrologer>, when calculating nativities in this <winter> hemisphere, says the someone lived more than
91 years, right then one can know \textbf{/316K/} that he is lying and is willing to use invalid procedures. While convicting himself of ignorance, he is trying to eclipse the truth…

If the factor for the length-of-life is imprecise by <only> a fraction of an hour, but is nearly accurate and is taking effect at the date in question, then an almost fatal critical point and a dangerous crisis will happen but death will not ensue, \textbf{/303P/} because the interval between the years is doubtful and vague. For example, if someone is born at the third hour, it will also be necessary to examine the second and the fourth hour, and their length-of-life factors, to see if they are related to the given hour. (It is also possible
for the observation of the Ascendant to have been wrong.) I do not say to consider the hours more distant from the third, such as the seventh, the eighth, the ninth, the tenth, the eleventh, and the twelfth, because the matter will become too elaborate—just the closest. 

It is necessary to calculate one-half of the years for the degrees following <in the table>, since (as I said) each degree in the table increases its number by a factor of two, and one-half of this figure is calculated as the years. For example: if, when multiplied by 12, its magnitude is 7, then this is one-half of 16, 15, or 14 1/2 years. 

If the degree in the Ascendant is at the sequence break, this will cause one subtraction of years from the amount, i.e. from 7, 16, 15, or 14 1/2 <?>.

It is necessary to determine the degree-position of the Sun precisely: if it is given in minutes, I subtract a corresponding amount…

…From the observation of the \Sun, I find the Ascendant to be at \Virgo 1°…

\newpage