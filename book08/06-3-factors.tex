\section{The Computation of the Rising Times and of the Three Factors}
\index{rising times}
The computation of the rising times and of the three factors is done as follows: assume, for example, that the rising time of \Aries\, in klima 2 is 20\footnote{The ascension values given for each sign in klima 2 match those under System A for Babylon which all have zero minutes and seconds.}. I double this figure and get 40. I assign to

\begin{tabular}{lr}
\Aries\, 1° & (This is 40 minutes.) 0;40\\
\Aries\, 2° & 1;20 \\
\Aries\, 3° & 2;0 \\
\Aries\, 4° & 2;40 \\
\Aries\, 5° & 3;20 \\
\end{tabular}

and so on to \Aries\, 30°, adding 0;40 to total the 20 years.

Next we fit the second factor, starting from the first degree in a similar way: since \textbf{/291P/} \Taurus\, rises in 24, I double that figure and get 48, which I add to the 20 of \Aries\, for a result of 

\begin{tabular}{lr}
\Taurus\, 1° & 20;48 \\
\Taurus\, 2° & 21;36 \\
\Taurus\, 3° & 22;24 \\
\Taurus\, 4° & 23;12 \\
\Taurus\, 5° & 24;0 \\
\end{tabular}

and so on to \Taurus\, 30°, adding 0;48 to the 20 <of \Aries\,> and getting a total of 44;0.

We will derive the third factor in a corresponding way, as follows: the first degree of the third factor is

\begin{tabular}{lr}
\Gemini\, 1° & 44;56 \\
\Gemini\, 2° & 45;52 \\
\Gemini\, 3° & 46;48 \\
\Gemini\, 4° & 47;44 \\
\Gemini\, 5° & 48;40 \\
\end{tabular}

and so on to <\Gemini\,> 30°. By successively adding 0;56 we get a total of 72;0.

Do the same for the rest of the signs: apply the rising time of the sign as the first factor, then calculate for each degree to find the total years. Then the second factor is in the next sign, and the third factor is in
the third sign, all according to whatever klima is required.

We explained the rising times in Book I\footnote{See \S{1.7}}; now we are specifying the factors. I think that my exposition was magisterial. It was sufficient, in the case of the rest <of the other applications of the rising
times> to leave them unspecified, or when giving sample nativities, to expressly declare not just seven (as some do), but even more causes. (However we must not compare ourselves to men of wickedness and envy.) Lest anyone consider that we have taken on a fickle character, I will append and treat in detail some nativities. It is clear to us that the Ancients used this method, judging from the mystic statement of the Compiler: “The place derived from the rising times forecasts the limit of the entire time of life. \textbf{/305K/} Then if it allots the living times, it also seems to control the auspicious and inauspicious times.”

\newpage