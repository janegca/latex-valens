\section{The Fixing of the Degree in the Ascendant with Reference to the Two Appended Tables}

\index{Ascendant!finding}
First of all, it is necessary to determine the precise position of the \Sun\, in degrees, then to examine the previous new moon (if it was a new-moon nativity <=between new and full moon>). Find when the new moon occurred—at what hour and what degree of the zodiac. Then add the total number of days and hours from the new moon to the day and hour of the birth and calculate what fraction this period is of the total time between new and full moon (i.e. 15 days). Then subtract this from the magnitude of the \Sun’s degree position. (This procedure is for day births; for night births subtract from the degree in opposition to the \Sun.) Treat the result as a fraction of an hour.

Alternatively, count from the position of the \Sun\, to that of the \Moon\, and take 1/12 of that figure. See what fraction of the 15 days this is. Then subtract this from the magnitude <of the \Sun’s position>, and consider the remainder to be a fraction of an hour. So, if hour 4 is given, we consider this as 3 hours plus a fraction—not everyone can be born on the stroke of the hour. Because of this fact, twins have much variation in their lives>due to the changes of hours, signs, and sequences, because it happens that one of the twins will be short-lived, or will die immediately at birth, the other may be long-lived, depending on whether the hour does or does not make the connection.

If the \Moon\, is past full, it is necessary to calculate and find the fraction in the same way. When the \Sun\, and \Moon\, are in syzygy, or if the \Moon\, is in the sign of the full moon and is no more than 2° or 3° from opposition, the hour will be considered full. Its fraction will be the number of degrees the \Moon\, (i.e.
at its photismos <=phase>) is from either new or full, whichever is the case. 

If the hour is not in accord with the sect/distribution, by moving ahead or back one degree, we can determine the error in the data, particularly for a native who has died. For it is possible from these to find the years of life. \textbf{/298K/} For this reason the investigation must not be carried out carelessly; instead \textbf{/285P/} the prediction must be made after working with skill and accuracy. 

It is necessary to determine if the distance from the day and position of
the new moon to the birth date, i.e. to the \Moon’s position then, must be considered, or from the \Sun’s position to the \Moon’s, or from the photismos (i.e. the point in opposition to the \Sun) to the full moon.

As an example: \Sun\, in \Taurus\, 3°, \Moon\, in \Aries\, 2°. The distance from the \Sun\, to the \Moon\, is 329°\footnote{Counting is in the order of the signs: 27° of \Taurus\, + 30° of \Gemini\, + ... + 2° \Aries = 329° = 360° - 3°\Taurus\,(33°) = 329°.}, which is 27 lunar days\footnote{The \Moon\, travels 360° per month or 30 days; 329/360 * 30 = 27.42.}; the distance from the <previous> new moon to the \Moon’s position is 29 days. I count off the days from new to full moon, 15; there are 14 days <since full moon>. I multiply this by 12°
for a result of 168°. The distance from the point in opposition to the \Sun\, (\Scorpio\, 3°) to the \Moon’s position is 149°. Now 168° exceeds 149° by 19°. It is necessary to calculate in such a way that the
numbers are equal. So the distance <to be considered> will not be from the time and position of the new moon, but from the position of the current syzygy, i.e. the full moon. Therefore it is necessary to subtract the apparent excess, 19°, which equals 1 1/2 lunar days, from this amount. 

So, if we subtract the 1 1/2 days from the 14 days, the result will be 12 1/2, the total from the point in opposition to the \Sun\, to the \Moon’s position. Since from the \Moon’s position to the next new moon there are 32°, which is 21/2 lunar days, if we add the 2 1/2 to the 12 1/2, <we find> the 15 days of the cycle <from full to new> will be completed. (The synodic period of the \Moon\, is 29 days; the sidereal period is 27 1/3; the anomalistic period is 27 1/2.)

An additional procedure: calculate the distance from the new moon to the \Moon’s position <at the nativity>, or from the full moon to the \Moon’s position. Then, if the amount is less than 180°, use the
indicated method. If it is more than 180°, subtract 180° and determine what fraction of the <\Moon’s> motion the remainder is. Then multiply this by the hourly magnitude.

Another procedure: alternatively we find the amount by multiplying by 12 the figure entered at the \Sun’s position \textbf{/286P/} (for night births, multiply the figure entered at the point in opposition to the \Sun). Then we multiply this figure by the time in hours given for the \textbf{/299K/} delivery. After casting out 360, we consider the remainder to be the horoscopic gnomon. Then take the distance according to rising times from the \Sun\, to the \Moon\, and compare it to this first horoscopic gnomon. If the solar gnomon is greater, add a factor to the Ascendant. If it is less, subtract. The factor to be added or subtracted is indicated by the excess of the solar magnitude. The total amount (before the adjustment) will be derived from the addition or subtraction of the hour or fraction of an hour.

\newpage