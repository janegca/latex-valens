\section{The Scientific Construction of the Second Table}

Having described the organization of the table, we must go on to give the rationale for its construction. The additional factor in the sequence, 14, indicates the “illumination” of the Moon, while the progressive
increase of 12 $<$from sign to sign$>$ indicates the “fingers” of the Sun. Two times 14 equals 28, the Moon’s period. So, since 2 is entered next to Libra 1°, we subtract 1;40 from it for a result of 0;20 - which is a magnitude of one-third. (This divided into 60 yields 180. I split 180 up into sixtieths for a result of 10,800. I divide this into 540°, i.e. into 11/2 circles, and the result is 0;20.) This figure will be placed beside Libra 1°. This 3 becomes 60, which is equivalent to one year. We bring the remaining years into the calculation as follows, adding 2;20 to each $<$degree$>$:

\begin{tabular}{lr}
Libra 1° & 0;20 \\
Libra 2° & 2;40 \\
Libra 3° & 5;0 \\
Libra 4° & 7;20 \\
Libra 5° & 9;40 \\
Libra 6° & 12;0 \\
\end{tabular}

Next we add 14 (the Moon’s illumination) in this sequence to the 12, for a result of 26;0. We subtract 1;40 from this and are left with 24;20 to be entered next to Libra 7°. We then revert to the factor 2;20 and place 26;40 next to Libra 8°:

\begin{tabular}{lr}
Libra 7° & 24;20 \\
Libra 8° & 26;40 \\
Libra 9° & 29;0 \\
Libra 10° & 1;20 \\
Libra 11° & 3;40 \\
Libra 12° & 6;0 \\
\end{tabular}

and so on.

\textbf{/283P/} At the point where the sequence is broken and at the initial numbers we must subtract 1;40,\textbf{/297K/} then continue with the addition factor 2;20. Use this procedure in the same way for the rest of the signs: we will find (for the second table) the figure to be entered next to the first degree of each sign in the same way we found it for the first table.
\newpage