\section{The Construction of the First Table}

The first table from 1° to 6° is constructed as follows: the figure 2 is entered next to Libra 1°

\begin{tabular}{lr}
Libra 1° & 2 \\
Libra 2° & 4 \\
Libra 3° & 6 \\
Libra 4° & 8 \\
Libra 5° & 10 \\
Libra 6° & 12 \\
\end{tabular}

There is a progressive increase of 2.

At Libra 7° this sequence is broken and a factor of 14 is added for a total of 26 to be entered next to Libra 7°:

\begin{tabular}{lr}
Libra 7° & 26 \\
Libra 8° & 28 \\
Libra 9° & 30 \\ 
Libra 10°  & 2 \\
Libra 11°  & 4 \\
Libra 12°  & 6 \\
\end{tabular}

Here again 14 is added to the sequence for a total of 20 next to Libra 13°:

\begin{tabular}{lr}
Libra 13° & 20 \\
Libra 14° & 22 \\
Libra 15° & 24 \\
Libra 16° & 26 \\
Libra 17° & 28 \\
Libra 18° & 30 \\
\end{tabular}

Here again 14 is added to the sequence for a total of 44, from which 30 is subtracted, leaving 14. This figure will be entered next to Libra 19°:

\begin{tabular}{lr}
Libra 19° & 14 \\
Libra 20° & 16 \\
Libra 21° & 18 \\
Libra 22° & 20 \\
Libra 23° & 22 \\ 
Libra 24° & 24 \\
\end{tabular}

Again 14 is added to the sequence for a total of 38, from which 30 is subtracted, leaving 8. This figure will be entered next to Libra 25°:

\begin{tabular}{lr}
Libra 25° & 8 \\
Libra 26° & 10 \\ 
Libra 27° & 12 \\
Libra 28° & 14 \\ 
Libra 29° & 16 \\ 
Libra 30° & 18 \\
\end{tabular}

So every 6° the sequence will be broken, 14 will be added, then 2 will be added <per degree> in each sign. Therefore Libra will have the figure 2 next to Libra 1° and 18 next to Libra 30°. Leo and Pisces have the same arrangement of figures as Libra.

Next in order Scorpio will have 14 next to Scorpio 1°, 2 will be added to each degree in the series, giving 24 next to Scorpio 6°. Then the sequence is broken, finishing with 30 next to Scorpio 30°. Aries and Virgo will have the same numbers.

In order to explain the construction more briefly so that the \textbf{/296K/} table as a whole and its particulars may be remembered, calculate the increments <between signs> as follows: 2 is entered next to Libra 1°. To this figure I add 12 (for the \textbf{/283P/} circle of signs) for a total of 14. Scorpio has this figure entered next to Scorpio 1°. Add 12 again to this 14 for a total of 26. Sagittarius has this figure, 26, next to Sagittarius 1°. Going in the order of signs and adding 12, we can find the correct figure to be entered next to the first degree of each sign. By adding 12 <to the figure at 1° of each sign> and by breaking the sequence with the addition of 14, we can construct the entire table. The figure next to Sagittarius 1° will be the same as
Taurus 1°; Aquarius 1° will be the same as Cancer 1°; Capricorn 1° will be the same as Gemini 1°. 

In one respect these pairs will have similar powers and will support each other mutually, but in other respects they will be different because of their different rising times. 

This table also has the years tabulated beside the figures and the degrees as an example <of the procedure>. Intelligent students will easily grasp the precise calculations for each klima and for changes in the location <of the nativity>.

\newpage