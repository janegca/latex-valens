\section{How to Establish the Hour of Birth for Twins}

The following will be the method for twins: if the first twin is said to have been born at the first hour, assume that both were born in the same one-half, one of them in the … quarter, the other in the other
quarter, the fourth. It is also possible for both to have been born in the same quarter hour and for one to have followed the other.

If the first is said to have been born at the beginning of hour 2, the second at hour 3, assume that he was born at hour 2 1/2.

If he is said to have been born at hour 5, assume he was born at hour 4 1/2.

If at hour 7, assume it was hour 6 1/2.

If at hour 8, assume it was hour 7 1/2.

If at hour 10, assume it was hour 9 1/2.

If at hour 11, assume it was hour 10 1/2.

If at hour 12, the report is accurate; it is not possible for another interval to be made $<$between them$>$.

If the first is said to have been born well into hour 1 and the second at hour 2, it is not possible. The second will have been born at hour 2 1/2 or 2 3/4.

If the first is said to have been born at hour 2, the second at hour 3, it is not possible; he was born at hour 3 1/2.

If the first is said to have been born at hour 3, the second at hour 4, this is not possible; he was born at hour 3 or hour 4 1/2.
…

If the first is said to have been born at hour 6, the second at hour 7, this is not possible; the second was born at hour 6 or 7 1/2.

If the first at hour 7, the second at 8, assume that it was 8 1/2.

If the first at 8, the second at 9, this is not possible; the second was at hour 8 or 9 1/2.

If the first was at hour 9, the second at 10, this is not possible; the second was at hour 9 or 10 1/2.

If the first was at 10, the second at 11, assume that the second was born at hour 10, or in the first hour of the night.

It is possible for twins to be born in the same quarter of an hour. The rapid rotation of the hours, bringing with it a change of degrees, \textbf{/300K/} makes the possible points of time uncountable.

\textbf{/287P/} This rapid motion brings great effects from the briefest of intervals, making one twin long-lived; or it can bring small effects from great intervals, making the other twin short-lived. It is necessary then to determine the intervening degrees and to calculate the difference.

\newpage