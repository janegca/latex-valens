\section{The Length of Life Found by Using the Full-Moon and the Horoscopic Gnomon}
\index{length of life}
I now have methods for a ready determination of the matters in question thanks to my constant practice and the underlying theorems of my flexible system. Still, the methods for making final decisions and for critically evaluating <horoscopes> needed more time—and time has become short for me. Man’s life is but a blink of time, even if he seems to be enjoying a long lifespan, so just like a father who, weighed down by illness at the end of his life, leaves his last brief commandments to his children before the silence of death creeps on him, I have carefully reviewed the collected chapters of my theoretical treatise, and I have noted the important topics. In so doing, I have presented the beginnings of an approach <to this science> to the lovers of beauty.

Now if the mind were long-lived or immortal, any decision would be free of doubt and would be simple—“But the gods know all.” Now since the length of life is the most essential topic, it has been treated in many different ways in the preceding books. I came into contact with a man who was boasting of some method, and I discovered (after much trial and error) a complicated method of regular calculations, a system which I myself had previously known to some extent and which now I will explain to everyone because of my zeal, after pruning off all excess verbiage. 

<We know that> every method when combined with another and tested gives us an exact scientific viewpoint. Let our method be this: we will use the Light-bringing \Moon\, and the Ascendant. These two,
mingling their influences according to the hourly motions which lead to their positions at any given moment and their visible appearance, make the aphetic point and indicate the beginning and the end. They have a mystic power over conception and delivery. Neither \Mars\, nor \Saturn\, will be considered as destroyers nor the benefics as helpers. Instead, when the \Moon\, is found to be the apheta, it is necessary to observe closely its contacts and its sextile, square, and opposition aspects with respect to the Ascendant, particularly when they are in signs of the same or equal rising time, signs of the same power, the listening \textbf{/258K/} or beholding signs, or in the antiscia. Similarly \textbf{/247P/} if the Ascendant is the apheta, examine its aspects with the \Moon\, using the rising time. \index{aspects!Asc-\Moon}In my experience it seems best to judge those degrees as fatal and those aspects as powerful, i.e. those aspects of the \Moon\, and the Ascendant with each other—also their squares and oppositions, since these aspects have an extraordinary power when at the angles.

Now, many mistakes are constantly being made about the aphetic (or anaeretic) places, and almost anything can present difficulties to this art, since the lack of any proof of its accuracy can give rise to criticism and rejection. Because I have found the truth, I feel it necessary to correct the mistakes. So, when the aphetic points are found to be unrelated to the apparent anaeretic point, it is necessary to examine
both degree-numbers using the rising times and to consider the <resulting number> as the length of life—if it does not exceed the maximum period, because rarely will anyone live longer. In addition when a great interval is found between the apheta <and the anaereta>, and when the \Moon\, and the Ascendant are in the
same sign, or when (in the case of another aphetic point) the interval between the two does not happen to be found in the signs of long rising time, with the result that the aphetic places are very near each other—in such a case, add the time-periods using the rising times, take half of the sum, and consider that to be the length of life. It is necessary to examine the vital sector in this way, not only in the case of long intervals and signs of long rising times, but also in the case of small intervals and aphetic distances (viz. in the signs of short rising times).

With respect to the determination of the aphetic points of the chronocratorships, the degree-positions of the \Moon’s phases in each sign, when calculated with reference to the aphetic points and correlated with the time periods corresponding to each interval, indicate the anaeretic point—particularly when the \Moon\, is the apheta. When the \Moon\, is the apheta, it is destroyed by itself. The relevant phases are those of new moon, full moon, and the two quarters, each being effective when moving toward the Descendant.

\index{The Ancients}
Let no one think we have composed this treatise in too complex and complicated a fashion. It is my favorite occupation to inform my readers of every method of inquiry. \textbf{/259K/} It is possible for those readers to train their minds over time in these systems, to discard vulgar notions, and to embark on the exact, scientific way. I have claimed in the preceding books that I have elucidated what was obscurely composed by the Ancients—indeed I have expounded their correct opinions \textbf{/248P/} so that I might not seem to be
uninformed about their studies. I have also compiled here my own discoveries. If the reader trusts in this information, he will have an unexceptionable explanation <of this art>. 

One \mndl must observe in which stars’ signs or terms the aphetic or anaeretic places occur, because it is from these (i.e. from their natural activity) that crises and deaths can be determined. It is also necessary to examine which star is afflicted or helped by which, which star is in harmony or is not, and what their mutual configurations were at the nativity, at transits, and at the transmissions of the chronocratorships.

Much of our discourse has been to explain the Ascendants by degree and by sign, and the operative degrees, places which have a complicated and complex relationship with the topic “length of life,” even if they do not have their mutual points of contact in the same (or nearly the same) degrees or signs—a thing which is rare and wonderful to most men. But in the cosmic revolution, all things are possible, attainable, holy, and true, and become so through thousands of complex paths, which when investigated, will reveal the truth to one who searches not at random but with scientific skill. For the universe itself is not random, and day and night it shows without stint to everyone the good and holy things it contains. When <merely> seen, it is not grasped; but when apprehended with the intelligence, each thing is known to the extent possible for a man. Therefore, what has been seen, known and said is comprehensible, and in fact I have grasped it in my turn. I have ignored who <the discoverers> were, from where they came, and how they made their discoveries. Others have written endless words about such matters. I have interpreted
what has been discovered and proven by the observations of the chronocrators, and I have been mystically inspired to compile this treatise, hoping to have laid a strong foundation in my writings, revising them \textbf{/260K/} and constantly laboring (though with great pleasure), because the functioning of the cosmic bodies and the discovery of long-sought treasure, a discovery involving new theorems, drives me on.

\newpage