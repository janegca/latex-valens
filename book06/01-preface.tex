\section{Preface}

Every art and every craft is pleasing in proportion to the activity and the intelligence of its practitioners, or in proportion to their bodies’ fitness or unfitness for the activity in question. As a result, many criticize and mock their neighbors $<$?$>$ because they do not share the same activities—of course it is impossible and undesirable for everyone to know the same things. Each man believes himself to think and
plan better than anyone in that sphere of activity which he has and which he thinks is best. For example, often a man is found to be craftsman-like and systematic in all types of activities, and he is successful even though illiterate. Another man, well educated, is easily deceived like one inexperienced in business and is betrayed by his naiveté and his disorganization. This man in his unhappiness thinks that the possession of an education is useless and considers the ignorant man to be happy. Such differences are the works of Fate and Fortune, who come quietly, ineffably to men and who, without sense or decency, make some men fortunate, other men wretched. Life, acting with criminal deception, in her course exalts and glorifies men in some respects and leads them to prosperity and preeminence, so that many men become lovers of such things. At other times life pains, wastes, and withers men, and brings them to obscurity, danger, and hatred. Life even turns some to the loathsome arts and sciences, and endows them with hardened and stubborn minds, activities, and fortune. \textbf{/241K/} All this happens and comes to pass mixed with pleasure, satisfaction, and pain, depending on the changes of
situation and the recurrent cycles of the chronocrators. 

I have written this because I have prided myself on the knowledge bestowed on me from heaven by the Divinity, knowledge which is now dishonored and rejected, even though it is primordial and governs everything in life, \textbf{/231P/} and even though without it there neither is nor will be anything. Now even its name seems to be hated, although men before our times prayed for it and blessed themselves by it. I am grieved by this, and I envy the old kings and rulers who devoted themselves to such matters. I am envious because I was not fortunate enough to live in those times which saw such a climate of free and ungrudging speech and inquiry. Their devotion to this science was so enthusiastic and so steadfast that they left the earthly sphere and walked the heavens, associating with the heavenly souls and the divine, holy Minds. Of this Nechepso is a witness when he says:
\begin{verse}
		I seemed to walk the midnight aether, \\
		and a voice from heaven echoed around me, \\
		at which the dark robe covered my flesh, \\
		bringing the gloom of night…\\
\end{verse}	
and so on. Who would not consider this knowledge to be superior to any other and to be blessed, since by means of it one can know the Sun’s ordered paths which foretell the changing seasons when it enters the tropics in the advances and retardations of its motion; one can know the risings and settings, the days and nights, the seasons’ cold or heat, and the weather? It is also possible to know the varying paths of the Moon, its inclinations and departures, its waxing and waning, its heights and its depths, the direction of its winds, its contacts and separations, its eclipses, its near eclipses, and all the rest. From all this it seems possible to understand everything on earth, in the seas, in heaven, as well as the beginning and end of all events. Likewise for the five other stars, with their motions, their uneven paths, and their varied phases. Although they are called “variable” and “wanderers” $<$planetai$>$, they have a fixed nature and return to the same places in regular cycles and periods. 

\textbf{/242K/} But as it is now, the inquiry into and the rectification of astrological matters has been hindered—or withered—by fear. Man’s intentions, rejected and unsupported by rationality, do not remain
constant, but jumping from here to there, always return to the original state of oblivion. Even though a man has good intentions and loves wisdom, still he easily becomes hesitant and chooses ignorance rather
than the hazards of virtue.

Nevertheless, in all men the drive $<$towards learning$>$ is strong, even though repressed and pained, and it remains constant. In my case, \textbf{/232P/} colorful horse races and the sharp crack of the whip have not carried me away, nor have the rhythmical movements of dancers, the vain charm of flutes, of the Muses, of melodious song, of those things which attract an audience by tricks and jesting delighted me. No, I have not even shared in those harmful, though profitable, actions, those actions of mingled pleasure and pain $<$=love$>$, nor have I consorted with those polluted and wretched $<$prostitutes$>$. Instead, when I had once discovered this divine and revered knowledge of heavenly bodies, I wished to purify my life too from all evil, from all taint, and to keep my soul immortal. From that time I felt I was associating with divine beings, and I kept my intellect sober, in order to seek truth clearly.

Now I have compiled many amazing things, things which can lead my readers back to the Ancients and their lore. I have repeated these doctrines because the Ancients expounded many methods which are
difficult and abstruse. I considered it necessary in this treatise to explain the mystic and secret methods concerning propitious times and the future. In so doing, I hope our science will appear understandable, be supported by the facts, and drive away hostility. It will convert its enemies and show itself to be holy and revered. I do not care if I seem to be speaking repeatedly of the same matters. For I discovered some
things and simply appended them to my earlier writings, due to the sudden rush of enthusiasm caused by my discoveries—the compiler was in ecstasy, particularly about these matters, and he felt that he was meeting God face to face. Other matters $<$I wrote out$>$ in an orderly manner. Consequently, if any envy has intervened to damage our treatise, the substance will be found written out elsewhere in the book.

\newpage