\section{Propitious and Impropitious Times According to the Degree-Intervals and Contacts of the Chronocrators (1K,2P)}

\textbf{/243K/} In the previous books the distribution of the chronocrators which had been roughly calculated by signs alone has been explained. Now we must speak of the intervals and contacts calculated in degrees, a method which I treated only obscurely before. Experience has led me to clarify these matters.

For any nativity it is necessary to determine precisely the stellar chronocrator to the degree, using the treatise, \textsl{The Perpetual Tables}\footnote{``Before Ptolemy’s Handy Tables became available sometime in the 3rd century ce, to obtain planetary positions astrologers used the \textsl{Perpetual Tables} (\textsl{aiōnioi kanones}, cf. Almagest 9.2), which appear to have been based on Babylonian planetary period relations and simple geometrical models.''\href{https://www.academia.edu/31038622/_Astrology_The_Science_of_Signs_in_the_Heavens_In_The_Oxford_Handbook_to_Science_and_Medicine_in_the_Classical_World_edited_by_P_T_Keyser_and_J_Scarborough_Oxford_PROOFS_2018_} {Astrology: The Science of Signs in the Heavens} by Glen M. Cooper, Academia.edu, 2018.}. It is necessary to take the distance in degrees from each star \textbf{/233P/} to its point of contact (a position expressed in degrees) with another star (as near or as far as one wishes) according to the applicable chronocrator and the sequence of signs. Add the number of degrees and allocate them it using the hours, days, months, and years of the star’s period.\footnote{“The distance and contact in degrees: suppose the \Moon\xspace is in \Virgo\xspace 21° 30', \Saturn\xspace in \Libra 4°. The distance is 12 1/2°. This amount multiplied times the Moon’s period (25) equals 312 1/2. We consider this figure to be days, months, hours, or years. I say that after 312 1/2 days, the Moon transmits to Saturn, and I will make the resulting forecast from the nature of the transmission from the moon to Saturn.” - a note in Marc. gr. 334, ff. 172-173 (432.26-30P) [Riley]} Then forecast the effects of the stars from their natural activity and from the predictive effects of their transmissions to each other at the chronocratorship/time-period which the nativity has reached. 

If the total number is less than the length of time <reached by the nativity>, examine the magnitude of the difference and forecast that the results will happen after that amount of time. 

If the total is greater than the time, subtract one factor, then count the remainder from the degree-position of the apheta. See if the rays of the stars incline to that point either at the nativity or in transit. Great and remarkable forecasts are often made when this is done. 

If a rough distribution <i.e. not by degree> shows nothing, and if the time is different after the factor or the remaining time is determined <?>, then reinterpret the number as a lesser period (i.e. months or days), correlate the factors with the stars’ periodic cycles, and count until the time in question.

In addition, these varied aphetic points and contacts between benefics and malefics indicate varied, multifaceted, and easily changed results. For it is possible to see many men who experience many things at one time, since aphetic points or operative places crowd together and mingle at the chronocratorships in question. A few men do not even have an alternation of good and bad, but stay in the same condition.
Others are exalted and attain an unsurpassable fortune (according to \textbf{/244K/} their basis from the beginning). Although they are even blessed by the public, they come to death or danger. Others are in difficult circumstances, are failures in their livelihood, have no good prospects, and are grieving in vain. Their fortunes however reverse themselves into the same condition, or even gain a greater magnitude. This happens because sometimes a transmission occurs at an eclipse and indicates a great threat, but the threat is diminished by a chronocrator in the arrangement, i.e. by a different transmission of a benefic. In the same way the transmission of a benefic is hindered by some afflicting cause.

The degree-numbers, which are used as factors, remain the same from the beginning and are counted in the cycles as far as the chronocratorship in question extends. As for the propitious, yearly aphetic points, the distributions are found using the suitable number for the same places and stars, \textbf{/234P/} and they show
the general tenor of the forecast. But if another <transmission> is entangled with it at the change of chronocrators, affairs become altered and occasions for good and bad arise. \mndl Therefore it is necessary to pay sober attention to the numbers: examine the angles and the signs preceding or following the angles to see if the signs are suited to the presence of the stars. Note the risings, the setting, the sects, and (to be brief) whatever we have explained in the preceding books. (We are not speaking to the ignorant or the uninitiated.) It is also necessary to examine the transits at given times, because they have a great effect for overturning or rectifying affairs, particularly when they are passing through operative places or are beholding the ruler of the chronocratorship while in conjunction, in aspect, or in opposition.

It \mn{3° Orb} is necessary to take into account not only the degree-position at which the contact occurs, but the 3° on each side of the contact (as we made clear previously), because it will be obvious from these degrees whether the stars’ effects will occur early or late—just as with eclipses, one can see the period of totality and the period of separation <to be within this distance>. It is also necessary to see if the transmitter beholds the receiver in an appropriate and operative way. The brief extent of the contact shows what will
happen, but it will last only until another contact comes to share in the operation and to weaken that first contact’s effects.

\textbf{/245K/} He allots to each star the time appended here\footnote{The times allotted are the planet's minor years allotted as months for signs and as days for degrees.}:

\begin{longtable}{c c c}
\toprule
\textbf{Planet} & \textbf{Months per Sign} 
	& \textbf{Days per Degree} \\
\midrule
\Sun & 19 & 19 \\
\Moon & 25 & 25 \\
\Saturn & 30 & 30 \\
\Jupiter & 12 & 12 \\
\Mars & 15 & 15 \\
\Venus & 8 & 8 \\
\Mercury & 20 & 20 \\
\bottomrule
\caption{Planet's Months and Days}
\end{longtable}

In this way the zodiac is subdivided by the periods and cycles of each star. 

Whenever <an astrologer> makes a vital sector/transmission from one star to another, let him be sure to note the right numbers and which star is transmitting to which, because the numbers are not the same nor are the effects of the transmitters and the receivers the same. For example, if one finds a transmission from \Jupiter\xspace to \Mars, then one from \Mars\xspace to \Jupiter, the numbers for the chronocrators will not be the same because of their \textbf{/235P/} differing periods nor will the forecast be the same. (It is better for \Mars\xspace to transmit to \Jupiter than for \Jupiter\xspace to transmit to \Mars.) 

Whenever vital sectors\footnote{Schmidt says ``releasing'' (VRS5 p.66).} are charted from every star to every other star, it is necessary to determine and forecast results depending on whether the transmissions of benefics or of malefics predominate.
“But,” someone will say, “the chronocrators for twins will be the same, since the same stars are located at the same degree-positions!” I answer that in such nativities, the shift of just the Ascendant alone
alters the angles and the fortune and condition of the native, and it occasionally brings his end. Because of these shifts, it is necessary to take into account the distributions with respect to the position of the angles and the stars’ aspects with each other, if they are not at the angles with respect to the horizon. In addition, if a star at its current degree-position should have another star in superior (or in any) aspect at a distance equivalent to that between angles, i.e. if the first star were in the Ascendant, the other would be at MC according to the distance between the signs, (or we could use the distance between any two angles\footnote{Does this mean take the distance between any two angles as equivalent to a square (recognizing that the MC and ASC are always 90° apart in ascension degrees)? or use planet longitudes to find if they are 90° from any angle?})—in such a case the aspect will be especially vigorous.

In its effect this system <of chronocrators> \textbf{/246K/} might be compared to the game of white and black pieces—for life is a game, a pilgrimage, and a festival. Competitive men devise wicked traps for each other, move their pieces along the many straight rows, and put their pieces down in various places when summoning each other to a skirmish. As long as the place happens to be unguarded, the counter moves unchecked according to the will of the player: it flees, stays, pursues, attacks, wins, and loses in turn. If it is surrounded by the opposing pieces as if in a net and finds the straight rows to be blocked, it is intercepted and captured. In this way, of the two players, one finds momentary pleasure and enjoyment for himself, the other momentary mockery and pain—momentary because the one who had been in despair suddenly comes back into the game by means of some stratagem and gives back the burden of despair to the other player, who had just now laid it on him. The stars’ effects should be viewed in the same way: as long as a benefic controls the chronocratorship
and no malefic comes into contact, this active, healthy, easy, and successful benefic gives the one who is living through this period the reputation of being fortunate, bold, and intelligent, even if he is a boor.
Even if he is unworthy of the happiness bestowed on him by the current situation, he prides himself and rejoices \textbf{/236P/} at what he has. He does not attend to the changes of situation, and he causes bitter grief for many men. 

But when a malefic dominates the chronocratorship, it is impropitious, diseased, impossible to overcome, and full of setbacks, so that the one who is living through this period is said to be helpless and cowardly in the face of evils, in fact wicked, even if he is in truth a worthy man. Such a man, although driven to despair because of the evil of the times, still resists the fickleness of Fortune by the force of his
reason, and shows himself to be noble and resourceful. 

If \mnbm afflicted benefics control the chronocratorships, both good and bad happen: damage with profit, ill-repute with high rank, ruin, dangerous accusations, fears, easily-cured diseases. As a result, those who are living through this state of mixed pleasure and pain do not come to their ends with either unalloyed bad or unalloyed good.

It is necessary to compare the influences of the transmissions to see which type \textbf{/247K/} is more vigorous: if the malefic, then the native will fail in his expectations for good, will feed on vain hopes, and
will experience bitter grief; if the benefic, the native will overcome delay, distress, reversals, and expenses, or he will escape from injuries, suffering, and mortal crises combined with fears and dangers. Of necessity he will endure punishment and will consider that very thing an excellent gain <?>. 

The \mn{basis} general tenor of the transmissions must be viewed with respect to the basis of the nativity and the distribution of the overall chronocrators—as was explained previously. It is necessary to determine if the nativity is distinguished, average, exalted, ruined, subject to the law, etc. The stars in the cycles of the chronocrators can assume the
causative power which the nativity’s foundation has previously indicated.

The long intervals have a slower causative effect; the closer intervals have a more rapid effect. Some stars, traversing the chronocratorship/in their movement at the chronocratorship have similar effects: they
are generally operative for a short time in the chronocratorship of a benefic [or a malefic], and they indicate success, trust, profit, blessedness, benefactions, and <honors, but when the interval is greater, or even different> and a malefic receives the chronocratorship instead, the situation changes to one of ruin, disgrace,
misfortune, and sudden danger due to accusations and hatred. 

Occasionally, when \textbf{/237P/} another chronocrator takes effect, particularly when a benefic receives the transmission, a restoration of fortune and rank occurs. 

It \mn{Ascendant} is also necessary to make a determination about the “encounters” with the degree in the Ascendant, because it is from these encounters that the critical, the diseased, the injurious, and the dangerous chronocratorships are discovered, and on the other hand, the healthy, the delightful, the lovely, and the desired times—all according to the presence of benefics or malefics. Through these, the soul becomes undisturbed and strong, feels itself able to do what is best, stretches itself out to the maximum, shows itself a benefactor, and is honored and blessed. The body <=health> is soothed as well. I suspect that those men whom we mistakenly called good and bad have gained their appellations from these <encounters>. \textbf{/248K/} (The scientists <say> they keep their original makeup.) 

Whenever <the soul> is afflicted, it is ensnared and blinded. It is marked off as much-hated, wicked, and evil, and even crazy and mad, so that it contrives something dangerous for itself. If it is so burdened then it cannot bear to live its whole lifetime to the end, being distraught in its misery, and it separates itself from the body. It then travels with its compelling demon, carried wherever that demon wishes.
\newpage