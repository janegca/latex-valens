\section{The Determination of the Moon and the Ascendant at the Conception}
\index{conception}
In every nativity the \Moon’s position is figured by longitude and latitude. Note where the latitude of the \Moon\, is located at the nativity. \textbf{/249P/} The \Moon’s position at the conception will be the position we have found. For in all nativities, the latitude <of the \Moon> at the delivery will be the same as the latitude which the \Moon\, had at the conception. In the fulfillment of time the \Moon\, cannot exceed its own “guidepost.” Count back the days, and set the \Moon\, at that <correct> degree, then see where the previous new moon occurred. Add the days from that point, and you will know at what day and hour the conception occurred and whether the native was an 8- or a 10-month child.

For example, if the \Sun\xspace at the conception is in \Aries\xspace and the \Moon\xspace is 120° past the new-moon phase, there will be a full-term gestation period of 270 <days> (for this to be the case, it is impossible that the distance exceed 120°, because in this distance the full-term “Ascendant” <?> is completed\footnote{Likely means that this time is required to produce a `perfect' or full-term nativity.}) and the birth will be in the tenth month. If it does exceed 120°, in every case add the days to the nine months; after summing the total, you will know the date of the conception. In each case search out the degree of the latitude and see when (i.e. in how many days and hours) the \Moon\, will come to the same guidepost. Just like a revolving wheel, it will slow down when coming to its own place. 

I have composed this book not artistically as some do, performing an enticing “concerto” in their arrangement of words and their use of meter, charming their listeners with their mythological, mystifying
obscurities. Although I have not used fine language, I have experienced much, have expended much toil, and have personally examined and tested what I have compiled. Experience is better and more reliable than mere hearing, because one who hears has only an unreliable and doubtful grasp. One who has had experience, has tried many things, and has remembered them, validates what he has experienced. Men who are naturally malicious, ambitious, and contrary-minded \textbf{/261K/} are easily seen to be such and are punished, but their nature cannot be subdued: it can be made mild because of its shame and sorrow when pressure is put upon it, but when it is proven wrong, it becomes angry and bold at this provocation.

\index{the young}
One can note this phenomenon particularly among the young: they want to act differently <?>, they hand over to another the control over bad or good, they are carried along willfully and are forced to take the lead <?>, to act in a contrary manner, and to be bold in the face of everything. They become alien to their family and friends; they enjoy the company of their enemies. \textbf{/250P/} Since they lead the future with a halter, they despise everyone and rejoice in others’ troubles. As a result, evil comes on them, and they pray for dangerous harm <to come> through their enemies <?>. They suffer the reverse and are in pain to no avail; they do not honor the gods nor fear death, but are led by a demon. The end of such men is swift and
dangerous, and their life is easily crushed.  It is better for men, as far as possible, to put stiff-necked pride from their minds and to avoid boldness, to strip themselves bare and to surrender themselves to reason. For no one is free; we all are slaves of fate and if we follow her voluntarily, we will live undisturbed and without grief as a whole, having trained our minds to be confident. If someone adopts a false cast of mind and attributes the possibility of acting to himself, he will be refuted by the impossibility of his acting and will be a laughing-stock. Then he will remember these words of the tragic Euripides:

\begin{verse}
\small
Lead me, O Zeus, and you, O Fate, \\
Wherever you have assigned me to go.\\
I will follow even if I hesitate. If I did not wish, \\
Having become base, I will suffer this anyway. \\
\end{verse}

\index{\textit{Nemesis}}
In any type of systematic (or even unsystematic) art or talent, or in any other occasion, Nemesis will be the charioteer, holding the balance as in the mythical picture, showing that nothing is done \textbf{/262K/}
beyond measure. Her wheel is lying at her feet, indicating that what has <happened> is unsteady and insecure, since the wheel is unstable when it rolls by itself. In the same way these men who criticize and boast have an intelligence that always revolves in the same spot, and an inflexible reasoning ability which entangles them in passions. They live in a sweat, not able to attain those things which first they despised, but later wanted, when they had already lost them.

The End of Book VI of the Anthologies of Vettius Valens.

\newpage