\section{Why the Ancient Impressed Their Particular Colors on the Five Planets and the Sun and the Moon (2K,3P)}

We must assay the quality of the stars. We might use painters’ colors as a comparison: each color has a certain chromatic or translucent quality, and with this quality the color delights the beholders and is
useful for many things. If an incompatible color is mixed with it, it becomes muddied and changed, no longer having its previous nature nor preserving the hue of the color which was mixed in. It cheats the eye of either color, since it has taken on a false, coerced color, one dark and repulsive. On the other hand, the artist occasionally makes a harmonious mixture of colors and creates a lucky and beautiful blending. (Still, the artist depicts man by means of many colors, showing him to be a shadow of reality and truth.) In just such a way, the stars sometimes preserve their original nature when they are alone, but when one star mixes with another, \textbf{/238P/} the individual qualities of their nature are changed. 

The stars put a halter on men and lead them with pain, degradation, and various combinations of these <forces> to the business of life, in which men are trained in many ways, win the crown of endurance, and become what they were not before. Wherefore the Ancients were correct in comparing the stars to colors: \textbf{/249K/}

\textbf{Saturn’s} color is black, since it is the symbol of time. The god is slow, and therefore the Babylonians called it Phainon <Illuminator>, since everything is illuminated in time.

\textbf{Jupiter’s} color is brilliant; it is the cause of life and the giver of good.

\textbf{Mars’} color is orange, because the god is fiery, cutting, and consuming. The Egyptians called it Artes <The Hook>, since it is the diminisher of goods and life.

The \textbf{Sun’s} color is translucent because of the purity and transparency of its light.

\textbf{Venus’} color is changeable in hue, since it rules the desires and extends its control over many things, good and bad. It seems to rule by itself many appropriate and inappropriate occurrences in life. Since it has been allotted the circle which is below the Sun (whose zone is in the middle, dividing the stars), it receives the emanations of the stars above it and of those below it, and brings about various desires and actions.

They make \textbf{Mercury’s} color pale yellowish-green, like bile. It rules intelligence, discourse, and bitterness, and as a result those born under Mercury are like it in nature and color.

The \textbf{Moon’s} color is like that of the air. Its course is inconstant and varied, as are the deeds and intentions of those governed by the moon.

In order not to seem to tell all this twice, the nature of the stars is discussed in Book I.

\newpage