\section{To Discover Which Star Controls the Current Days When Using the Preceding System.}

\index{distribution!10-year, 9-month!days}
In addition, if we wish to know which star controls the current days, we will calculate as follows: multiply the years which have passed since birth by 365 1/4. Get the number of days from the birth date to
the date in question, and add it to the previous result. Divide by 129, \textbf{/245P/} and note how many cycles \textbf{/256K/} there have been. 

From the answer we cast out the weekly cycles <=divide by 7>, and the remainder of this division (which will be less than seven!) will show which star controls the days.

For example, take the preceding horoscope: multiply the full years, 52, by 365 1/4. The result is 18,993. There are 123 days from Mechir 12 to the day in question, Payni 15. When this figure is included, the total becomes 19,116. I divide this by 129, and get 149 with 24 days of the next 129-day cycle as a remainder. Next I divide this (i.e. the 149) by 7 (the week-number), and get 21 with 2 cycles remaining. Now since—as was stated above—the \Moon\, was found to be the apheta, I gave the first cycle of the week to the \Moon. \Venus, the star immediately following the \Moon, has the second cycle, 24 days, of which it gives to itself 8 days, then to \Jupiter\, 12 days. The remaining 4 belong to \Saturn, the days when the native was condemned.

Alternatively, the attempt could be made to equate the weekly cycles to 49-day periods and to count these from the apheta, giving one to each star. The star at which the count stops will rule the 49-day period and the first cycle of the weeks. Then transmit the cycles to the other stars in order. 

Take the preceding horoscope as an example: since there were found to be 149 weekly cycles, I subtract three 49-day periods from that <49 x 3=147>. There are two cycles remaining of the fourth 49-day period. I count these four periods from the \Moon, the apheta, and stop at \Saturn; so \Saturn rules the fourth 49-day period. It assigns the first weekly cycle to itself, the second to \Mars, the star immediately following it, which has 24 days. \Mars\, assigns (from this 24) to itself 15 days. The remaining 9 belong to the \Sun.

\subsection{That a Few [Astrologers] Misuse the Aphetic Point in the Preceding System}

Most <astrologers> distribute the chronocrators for each nativity using the seven-zone system beginning with \Saturn; they put \Jupiter, \Mars, the \Sun, \Venus, \Mercury, and the \Moon\xspace following in that order. In the rotation of the chronocrators, they examine the master of the week and of the days using the same system. 

Such a procedure does not satisfy me, because those who use it \textbf{/257K/} find the same chronocrators for most nativities. I prefer (as has been stated) \textbf{/246P/} to put the aphetic point at the \Sun\xspace and \Moon, or at the star found to be following the Ascendant, then to allot to the other stars in order, just as they happen to be situated by sign and by degree at the nativity.

\newpage