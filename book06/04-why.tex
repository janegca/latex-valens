\section{Why Malefics Seem to be More Active than Benefics (3K,4P)}

Malefics seem stronger than benefics. Just as a drop of black or brown, falling into a container of brightly colored paint, dims the color’s beauty, and the large quantity of brilliant color cannot hide the dark stain, however small—so it is with malefics, stars which can attack men and rob them of the things in which they seem to be fortunate: family, livelihood, health, rank, beauty, and whatever is rare. 

Malefics \mn{Malefics} take away \textbf{/239P/} possessions, \textbf{/250K/} involve men in crises, injuries, diseases, and they stain nativities. Good men, who think that everyone has the same beliefs as they so, who live naively, trusting and helpful, easily succumb, even if they are not <worthy of such an evil fate>. Still, they are celebrated for their
nobility. 

On the other hand, bad men, who think that everyone is like them, who do not trust even their own people or those whom they should trust, and who are unreasonably greedy, are deprived of their own property, and thus give pleasure to the public. For <Fortune> preserves some men, even though they are unwilling, until the time when it wishes to turn them upside down, pushing some to the heights, and some to their doom. In such cases, many men pray in vain for their lives, yield themselves to trouble, and are grief-stricken at their circumstances. They criticize the tardy arrival of death, and they contrive something against themselves or manage a violent end. 


Men are by nature mockers of others and criticizers of faults; they rejoice greatly at the troubles and the trials of neighbors, but later repent of their deeds and show themselves as the defenders of others’ faults…

Just as with the preconceptions with which anyone makes decisions—they have an appearance of the truth not only to those who believe them but also to many others who are forced by their misconceptions
to act in such a way, even if they have differing opinions. It is the same with those who compile astronomical tables: some facile men are considered as able to lead others to the truth; these are men to
whom the ignorant are attracted. But the scientists and the precisionists are rejected and condemned either from jealousy or from the crookedness of <their readers’> approach to science. Therefore, it is necessary to test their precision and to stick to the scientific viewpoint, even if approximations are sometimes necessary. Note that even Apollinarius, who calculated the visible motions using the old observations and publications of many returns and spheres, and who met with criticism from many readers, confesses that his calculations were one or two degrees in error. The source of the miscalculation is easily understood—as I myself using the results which have already occurred, have tried \textbf{/251K/} to establish the precise degree of a
star by noting its natural effects as the \textbf{/240P/} Ancient says. As our exposition proceeds, the facts themselves, brought before your very eyes, will clarify what I have said. When the degree is found, then it
is possible to make definite forecasts about the future. Determining the precise degree is difficult, but not impossible.

\newpage