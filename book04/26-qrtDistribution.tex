\section{The One-fourth [Method] for the Distribution of the Chronocratorships According to the Spheres Upwards—According to Critodemus}

\index{distribution!method!One-Fourth-Part}
\index{Critodemus@\textit{Critodemus}}
\begin{center}
\begin{tabular}{cc}
\toprule
\textbf{Planet} & \textbf{Years} \\ \midrule
\Moon 		& 1 \\
\Mercury 	& 2 \\
\Venus 	& 3 \\
\Sun 		& 4 \\
\Mars 		& 5 \\
\Jupiter	& 6 \\
\Saturn	& 7 \\
\bottomrule
\end{tabular}
\end{center}
The total is 28 years.

The degree-assignment <monomoiria> is done as follows: whatever <star> rules the sign in which the
\Moon\xspace is located, that <star> is taken as the first chronocrator, then the rest in the order of their spheres.

For example: the \Moon\xspace in \Libra\xspace 6°. \Venus\xspace is taken first as the ruler <in \Libra>, \Mercury\xspace second, the \Moon\xspace third, \Saturn\xspace fourth, \Jupiter\xspace fifth, \Mars\xspace sixth. Therefore <\Libra\xspace 6°> is assigned to \Mars\footnote{\textit{[i.e. \Venus, the ruler of Libra, in the 6th degree is in the monomoiria of \Mars so we start the distribution from him.]}}. 
\enlargethispage{\baselineskip{2}}
Now \Mars\xspace is taken first \textbf{/203K/} as the ruler of the degree-assignment of the \Moon\xspace for 5 years, then the stars coming after \Mars\xspace at the nativity are taken in order. After the 28 years are completed, begin again with the star coming after \Mars.\footnote{Assign the 10 years 9 months counting from the \Sun\xspace for day births, from the \Moon\xspace for night births. For day births, if the \Sun\xspace is unfavorably situated, start from the \Moon; for night births start from the \Sun. If the \Sun\xspace and the \Moon\xspace are both “not dominant,” start from the houseruler or any other favorably situated star - marginal note [Riley].}

\newpage