\section{The Position of the Month}

\index{distribution!method!12 year cycle}
“You will get the month as follows: <for day births> determine the distance from the \Sun\xspace at the moment in question to the \Sun\xspace at the nativity, then count that distance from the sign which has been allotted the year. 

For night births, determine the distance from the \Moon\xspace at the moment in question to the \Moon\xspace at the nativity, then count <that distance> from the sign which has been allotted the year. 

Observe the sign in which the new moon occurred, if the nativity was at the new moon; in the same way observe the sign in which the full moon occurred if the nativity was at the full moon. The months will be operative in these signs and will have this point as the beginning: however many days the \Moon\xspace at the nativity was from the new- or full-moon position, that same amount will indicate the beginning of the month. If the \Moon, for example, was 3 or 5 days from new or full, the beginning of the month will be at that point.

\newpage