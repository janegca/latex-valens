\section{The Distribution of the Chronocratorships Starting with the Lot of Fortune and with Daimon.}

\textbf{/160K/} I will now append this truly powerful method: to begin the vital sector with the Lot of Fortune and with Daimon (which signify the \Moon\xspace and the \Sun).

Universally speaking, the \Moon\xspace is fortune, body, and
spirit, and when it sends its emanations to us from its position near the earth, it causes appropriate effects, since it rules our bodily constitution. 

The \Sun\xspace is the cosmic mind and divinity. It arouses men’s souls to action through its own energy and love-inspiring nature, and it becomes the cause of employment and progress.
So, if we are investigating the chronocratorships with respect to bodily existence, such as \textbf{/152P/} critical points of illnesses, hemorrhages, falls, injuries, diseases, and whatever affects the body with respect to strength, enjoyment, pleasure, beauty, or love affairs, then we must begin the vital sector with the Lot of Fortune. 

Whenever the <first> chronocratorship ends, at that point we calculate the sign, the stars in conjunction or aspect, how the stars are configured with respect to the overall houseruling star of the vital sector’s chronocratorships, and whether the rulers of the Lots are at angles or not.

If on the other hand we are investigating employment or rank, then we will begin the chronocratorships with Daimon as the apheta. We will make our determination according to the benefics or malefics in conjunction or aspect with it.

Note that if the Lot of Fortune or its ruler are badly situated, the Lot of Daimon will distribute both the bodily and the active qualities. Likewise Fortune will make the distribution of both qualities if the Lot of Daimon or its ruler is unfavorably situated, and the same is true of the controls and the houserulerships.

Whenever Daimon and Fortune are found in the same sign, we will derive forecasts of bodily constitution from that very sign, but the forecasts of activity from the sign immediately following. In addition we can <use> the same apheta for new or full moon nativities, since at those times the Lot <of
Fortune> and Daimon fall in the same sign, but when we investigate the chronocratorships in such nativities with respect to physical health, we will start the vital sector at that very sign, but the chronocratorship with respect to activity at the one immediately following the Lot. This is particularly true for night births or for those nativities which have the new moon at IC and as a result have the angles
<Ascendant Descendant> square with the Lots. 

The results of a new moon are better than those of a full
moon because at a new moon the Lots are in the Ascendant, \textbf{/161K/} at full moon they are in the Descendant. It also happens that if the luminaries are square with each other, the Lots are in opposition to each other, and under this configuration some astrologers allot the chronocratorships for activities beginning with the signs immediately following <the Lots>. This however does not seem right to me, because
<usually> the Lot of Fortune is found at a different place from the Lot of Daimon, although they are at the same place for new and full moon nativities.

In addition, for male nativities, the vital sector is usually found to begin at Daimon, since these nativities share in activities consisting of discourse, giving and receiving, and trusts. For female nativities it begins at the Lot of Fortune because of their bodily functions. (But even men happen to accomplish things through bodily activities, i.e. handwork, athletics, and bodily motion, and women accomplish
things through buying and selling.) Similarly for infant \textbf{/153P/} nativities it is necessary to begin the vital
sector with the Lot of Fortune until <the time when> the nativity can show evidence for its full development or its occupation. Bodily excellence accompanies these <infant> nativities at birth, i.e. beauty, loveliness, size, elegance, fine proportion, or—which is more usually the case—the opposite: injury, disease, rashes, eruptions, pustules, or congenital defects such as birthmarks and hernias. The
active and intellectual qualities come into play later.

For example: assume that the Lot of Fortune or Daimon is located in \Aries. The overall houseruler <of \Aries> is \Mars. Determine if \Mars’ successors are or are not configured properly. \Mars\xspace itself allots 15 years first, and from this period it assigns itself 15 months. Next (because of \Taurus) it assigns 8 months to \Venus, next (because of \Gemini) 20 months to \Mercury, next (because of \Cancer) 25 months to the \Moon, next (because of \Leo) 19 months to the \Sun, next 20 months to \Mercury, next 8 months to \Venus,
next (because of \Scorpio) \Mars assigns itself 15, next (because of \Sagittarius) 12 to \Jupiter, next (because of
\Capricorn) 2 years 3 months to \Saturn. Next it assigns to \Aquarius the remaining 11 months to fill out the 15 years. 

Now \Venus\xspace receives from \Mars\xspace the overall chronocratorship for 8 years and assigns years to
each sign as described. \textbf{/162K/} Because of \Gemini, \Mercury\xspace receives 20 years after \Venus\xspace and assigns the years to each sign. Next is the \Moon\xspace with its 25 years, then the \Sun\xspace with its 19. 

It is necessary to assign the years in the order <of the stars> to whatever date the nativity extends <=lives>.

Now since the circle of the 12 signs has comprised 17 years 7 months, we will allot the remaining time using the signs in opposition: since \Gemini\xspace allots 20 years, if the vital sector begins there and if 17 years 7 months have been assigned, the remaining 2 years 5 months are allotted beginning with \Sagittarius, giving \Sagittarius\xspace itself 1 year <=12 months>, the rest to \Capricorn\xspace to complete the 20 years.

In a similar manner, if we find the vital sector beginning with \Cancer, \Leo, \Virgo, \Capricorn, or \Aquarius, after allotting the 17 years 7 months (not counting the intercalary days), we will allot the rest in order, beginning with the sign in opposition.

Some astrologers allot the remaining chronocratorships beginning with the sign in trine, but this does not seem scientific to me. Just as in the universe the four \textbf{/154P/} elements are in sympathy with each other, and each becomes alive and grows when linked with another, so in the distribution <it is necessary to> make the transmission from sign to sign, according to the harmony among them. For instance, since fire and air are upward-trending elements, they mingle with each other. Fire, a dry element, is nourished
by the mildness of the air, while on the other hand, fire does not allow the air to take on an icy or dark nature, but renders it warm and mild. In the same way it is logical that \Leo, the fiery sign, transmits the period remaining from its chronocratorship to \Aquarius, the airy sign, with which it is in sympathy, and in turn, \Aquarius\xspace transmits its period to \Leo.

Another instance: earth, a dry element, is nourished by moisture and gives birth to everything, while water, distilled from the earth and thus born from it, maintains the sympathy <between the two>. So it is logical that \Cancer, a moist sign, and \Capricorn, an earthy sign, mutually transmit to each other, as do \Virgo, an earthy sign, and \Pisces. The rest of the signs show the same interchange with the sign in opposition. 

And so the sequence of distributions is written in the order of zodiacal signs: \Aries—fiery, \Taurus—earthy, \textbf{/163K/} \Gemini—airy, \Cancer—watery, and the signs trine with these are of the same nature. Consequently if we make the connection within the triangles, we will find the nature of the transmitter and the receiver to be the same. No blending will be found, and each element will be overpowered by itself. If we use the other method, we find that the \Sun\xspace begins its course at the equinoctial tropic in \Aries\xspace and makes the days long for one hemicycle. Then, making the connection in \Libra, it begins to shorten the days. <The \Sun> in \Cancer\xspace stops the pattern of lengthening days, and when it is in \Capricorn, it causes this to happen to the night, making its change in the sign in opposition. Likewise the \Moon\xspace becomes new, waxes, then in its cycle makes the connection <=full moon> in the sign in opposition. As a
consequence I believe we should use the method described above for making connections.


\newpage