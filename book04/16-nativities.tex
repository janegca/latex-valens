\section{Nativities of Varying Fortunes.}

First of all it is necessary to take into account the basis and rank of each nativity and to coordinate the influence of the stars and signs with this basis, \textbf{/175P/} so that the forecast for average and for exalted nativities will not be different, not the same. 

Each star and each sign has an average influence for benefit or harm, as well as an extreme influence for good or bad, and they are sometimes the cause of great good, sometimes of great evil. So, if we find a distinguished and illustrious nativity <whose basis> is guaranteed by an aspect of benefics in the temporal transmission, but the malefics have the chronocratorship or have come in their transits to the angles or operative places, we say that the nativity will suffer nothing unexpected: the native will put his household completely in disorder, will endure scandal and criticism, and will be disturbed and fearful. 

If the \Sun\xspace or the full moon is afflicted, the native will act unlawfully and violently, will suffer upsets and much-talked-about dangers, as well as revolts of cities and of enemies, because of which he will live assailed by disturbances. For such nativities, \textbf{/185K/} it will be necessary to determine the conjunctions and aspects of the benefics in order to see if the causes might go beyond <what is expected>, and might foretell ruin or disgrace. If only the \Moon\xspace or the Ascendant is afflicted, the high rank of the native will not come to fruition because of his bodily afflictions and sudden illnesses, and it will become painful and grievous to the possessor.

If the \mndl basis of the nativity is found to be characteristic of one living an inactive, isolated life, varied and surprising activities during the transmission of the chronocratorships or in the configuring of transits should not be forecast. Furthermore, those who are entirely fortunate will not be harmed by malefics entering operative places, nor will the humble be helped by benefics—all because of the overall predisposition <of the nativity>, which partial influences cannot change. 

Now there are many nativities which fall from great fortune and rank to low rank, while others rise from mediocre fortune and base descent to prosperity and prominence. <In view of this,> if the nativity is found to be rising to high position judging from the overall situation<?>, and if benefics have the chronocratorship in the detailed configuration of years, brilliant prospects, benefits, and success will follow. If malefics have the chronocratorship, uncertainties, disturbances, and bodily afflictions will follow, but the underlying <favorable> basis will remain unchanged. 

On the other hand, if the nativity falls to low rank (again judging from the overall basis), even if benefics receive the yearly chronocratorship or are in transit, their influence for good will be quite weak and they will permit malefics \textbf{/176P/} to harm the nativity. \mnbm So, not in every case do benefics have a benefic role nor  malefics a malefic role, but they \mnmb interchange in the detailed configuration of years according to the overall basis of the nativity,becoming benefics <at one time, malefics at another.

In addition, it is necessary to examine the activities of each nativity to see if it gets its impulse from \Mercury, \Mars, \Venus, or \Saturn, or from the \Sun, the \Moon, or \Jupiter, and whether its basis is found to be distinguished. If each star is favorably configured in the transmissions or is coming to a transit, the year will be beneficial and productive of glory in <the star’s> type of activity. For example, if \Saturn\xspace comes into the places of the \Sun\xspace or the \Moon, the native will be harmed in those matters which the \Sun, the \Moon, or their places naturally indicate. Likewise if the year is with \Saturn\xspace \textbf{/186K/} or comes to it (i.e. where it was located at birth or where it is at the transit), we will make the forecast according to <\Saturn’s
nature>. When the rest of the stars, plus the \Sun\xspace and the \Moon, are in the Leisurely Place\footnote{The Schmidt translation has ``the idle place'' (the 8th house) (VRS4 41).} and are chronocrators, they become inactive; when in opposition <to the Leisurely Place>, they cause disturbances.

For any nativity, if the year is transmitted from the \Sun, the \Moon, and the Ascendant and gives any indications, those indications will be unchangeable, whether good or bad: it is good if the transmission goes to \Venus, \Jupiter, or operative places; it is bad if it goes to \Saturn, \Mars, or afflicted places; if the transmission goes to both, the nature of the stars and places will indicate what will happen that year. If the three figures <\Sun\xspace \Moon\xspace Ascendant> indicate incompatible events, the year will be complicated and
subject to ups and downs. 

It is better if malefics transmit to benefics rather than the reverse. 

A star transmitting to another star in the same sign (i.e. a houseruler in operative places) brings a vigorous period.

The following procedure seems to be scientific: the apheta of the years should start from each indicative Place. We start from MC when investigating occupations, from the Place of Marriage when investigating wives, from the Place of Slaves when investigating slaves, and likewise from the Place of Children. If we find that benefics are in conjunction or aspect with the place we come to, or if the sign is operative, we can
say that the outcome will have success, profits, and the fulfillment of wishes\footnote{This sounds like profections but from the house of interest rather than always from the Ascendant}.

In any configuration it will be necessary to see in which sign the houseruler of the houseruler\footnote{The dispositor of the planet ruling the sign marking the house.} is located and how it is configured. If the houseruler of the sign is unfavorably situated and indicates some crisis, but its houseruler [\textit{dispositor}] \textbf{/177P/} is favorably situated, relief from this evil will come, as well as partial benefits or a successful outcome of expectations: the native will receive trusts and gifts from the great or from royal personages, if the overall chronocratorship is controlled by the \Sun\xspace and the \Moon\xspace or by benefics, and if the distribution is to a good place. 


The following is particularly <effective> in a nativity: if \Jupiter\xspace is in superior aspect to \Saturn, or is in square, trine, opposition, or conjunction; likewise if \Mars\xspace is found in trine, square, or in the Place just following the Descendant when \Jupiter\xspace is at IC—under these conditions the gifts to the native will be very great and most profitable. 

In the case of those who present gifts to others, who strive for public acclaim, and who spend money on the masses: \textbf{/187K/} if \Mercury\xspace is found to be in aspect with \Jupiter\xspace and \Venus\xspace at the nativity, but not with \Saturn\xspace and \Mars, the native will be acclaimed and will share much public repute and honor. 

If \Mercury\xspace has \Mars\xspace in aspect on the right, the
native will regret his actions, will experience criticism, upset, and the notoriety of scandal, even if he spends his money lavishly. 

If \Mercury\xspace has \Saturn\xspace in aspect <on the right>, the native will end up a starveling, notorious, endangered. If \Mercury\xspace also has benefics in aspect, both things will happen.

All \mn{Aspect strength} stellar aspects are powerful, but the aspects of square and opposition are considered especially so.

<Aspects> are considered <strong> in the signs of equal rising times\ldots

The transmission passing to \Taurus\xspace and \Virgo\xspace from those <signs?> indicates that the results will be
insecure, subject to delays and lawsuits, and completely ruinous. 

The transmission to \Sagittarius\xspace and \Capricorn\xspace brings mysterious and harmful results because these signs are imperfect.

If the distribution is to be scientifically based, it will be necessary not only to examine the transmissions in the natal chart, but also in charts for katarchai and for runaways. 
Determine the Ascendant, make the chart, then use it in the same way as for nativities.

If \Saturn\xspace and \Mars\xspace have a relationship with the \Sun\xspace and the \Moon\xspace or with the Ascendant (e.g.
opposition, superior aspect, or any other influence for bad)\ldots

Assume, for instance, \mndl the nativity of a child is presented for interpretation, a child to whom a forecast for the beginning of an occupation cannot apply. When the transmissions of the stars are found, the results will apply to the father and mother, occasionally to his master, until the time when the native, attaining the age of full development, is subject to the indications <of the forecast>. 

Forecast for a child only what can apply to him: gifts, legacies, adoption, dislocations, boils, etc. Sometimes surprising \textbf{/178P/} forecasts are made for such people, forecasts which become evident from the overall interpretation of the stars. 

It is also necessary to harmonize forecasts for each indicated period with the age of the native and the customs and laws of the country. If this is done, the operation will be considered irrefutable. 

In horoscopes for paired nativities—brothers, man and wife, relatives, other persons linked by friendship—it is necessary to make the forecast for the individual whom the horoscope best fits \textbf{/188K/} at the applicable time, and to say that such and such will happen first or second to this individual; then assign the outcome to the other individual in the second place. For example: if the horoscope causes A to benefit from something to do with a death, and causes B to benefit at some time from an inheritance, an inheritance from which A also expects to gain, the gain will not come in the chronocratorship of A, but in the chronocratorship of B, who is expecting the inheritance. (The forecast will be made from MC.) The
same forecast will be realized more quickly for one person, but more slowly for the other, because of travel, a trial, accusation, or some other crisis. The same applies to rank, gifts, buying and selling, association and affinities, travel, and to all other occurrences in life. Therefore events sometimes come to pass sooner or later than expected because of the sympathies and antipathies of nativities. 

Nature reveals her causative forces through the cycles of the stars just as if she had supplied us with a map, but she brings some things suddenly and unexpectedly, while delaying others and holding them under the power of Necessity, until the
star which is the most fitting cause of the matter receives the chronocratorship.

It is necessary to make a comparison between the universal motions and these matters: the \Sun\xspace moving through the tropic degrees does not always effect the same change of weather, but sometimes it brings the universal fabric to a mild condition before it is expected, and sometimes it passes through the winter tropic with clear weather, but later stirs up heavy squalls and fearsome gusts of wind. 

Nor does the \Moon <always> cause storms corresponding to its visible phases, nor does it <clear> the air after conjunction, but sometimes it storms and shows the effects of its nature before they are expected, and it causes an
extraordinary mixture of weather, at other times it partially manifests wintery conditions, \textbf{/179P/} but then
in that very phase, it brings a clear sky. Occasionally, when passing out of conjunction, it takes a wintery turn. 

In a similar fashion the other weather indications and settings of the stars will not <always> have the same results: they will show their phases sometimes early, sometimes late, sometimes not in full measure.

These variations will occur according to the \textbf{/189K/} risings of the year, the new and full moons, the eclipses and the quadrennia, the overall and the cyclical houserulers, and with reference to the interchange of periodic transits.

\newpage