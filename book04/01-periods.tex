\section{The Distribution of Periods}

We believe that we have set forth an appropriate, in fact, magisterial, explanation of the previous $<$theorems$>$. We will now reveal a topic investigated by many and hidden from view, namely the distribution of propitious and impropitious times. We must preface our discussion with the distributions
which have been proven by our own experience. The primary period is one-fourth of the minimum period, as follows:
\begin{table}[ht]
\begin{center}
\caption{Planet Distribution Periods}
\label{Table 4.1}
\vspace{0.5cm}
\begin{tabular}{cccc}
\toprule
\textbf{Star} & \textbf{Period} & 
\textbf{$\frac{1}{4}$ Period} & \textbf{Days/Years} \\
\midrule
\Saturn & 30 & $7\frac{1}{2}$ &  $<$85 \\
\Jupiter & 12 & 3$>$ & 34 \\
\Mars & 15 & 3$\frac{3}{4}$ & 42$\frac{1}{2}$ \\
\Venus & 8 & 2 & 22$\frac{2}{3}$ \\
\Mercury & 20 & 5 & 56$\frac{2}{3}$ \\
\Sun & 19 & 4$\frac{3}{4}$ & 53$\frac{5}{6}$ \\
\Moon & 25 & 6$\frac{1}{4}$ & 70$\frac{5}{6}$ \\
\bottomrule
\end{tabular}
\end{center}
\end{table}

Altogether, the “fourths” total 32 years 3 months.

\newpage