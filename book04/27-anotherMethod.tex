\section{Another Method for Years, from Seuthos. The School of Hermeios, Which Starts from the Sun, the Moon, the Ascendant, or the Lot of Fortune}

\index{distribution!Method for Years}
“These are the four Places from which the beginning of the year <=chronocrator> is made: the \Sun, the \Moon, the Ascendant, or the Lot of Fortune. The choice is made as follows: if the \Sun\xspace is at an angle, it is necessary to count from it; for night births count from the \Moon, if it is at an angle to the degree. If these are inapplicable, count from the Ascendant. If the Lot of Fortune is at an angle, providing the luminaries are inapplicable, make it the beginning of the year. 

For nativities which have the luminaries approaching the angles, it would be odd to start from <a point> out of its own sect<?>. “Using our system you will most definitely recognize in which signs the year of the nativity will be
operative as it proceeds chronologically. For day births <start counting> from the \Sun, if it happens to be
in the Ascendant or at MC; if not, count from the sign in the Ascendant. Then count the year starting with the ruler (at the nativity) of the place where the count stopped. 

For night births, start counting from the \Moon, if it is situated as was described for the \Sun, particularly if it is rising and increasing in longitude <=waxing>. For the two luminaries if it is not full and if it is decreasing \textbf{/194P/} in longitude <=waning>, <count> from the node, if it is at an angle alone. If it is not as described, count from the ruler of the place where you stopped, i.e. when you counted from the Ascendant. 

If you find the signs at the angles, acting favorably or effectively due to the stars which are in conjunction or which are beholding in the natal chart and in transit, it is clear \textbf{/204K/} that the year’s results will be good. If the places are bad, the opposite will result, since the beholding and rising stars provide the active <=beneficial> impulse,
while the setting stars provide unemployment and the ruin of what has been done—unless matters happen to be obscure/the chart holds something hidden.

\newpage