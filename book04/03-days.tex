\section{The Distribution of Days}
\index{planets!distribution of days}
\textbf{/151P/} Make the distribution of days as follows: if \Saturn\xspace is found to be the overall apheta, it assigns 7 1/2 years.

Now since it is necessary for all the stars to take part in the distribution of this <7 1/2 year period>, we will do as follows: multiply the 85 days of \Saturn\xspace by 7 1/2 to get a total of 637 1/2. This is the amount \Saturn\xspace will allot to itself from its 7 1/2 years. 

Now let us find the amount for \Jupiter: since it governs 34 days, multiply this 34 by 7 1/2 (since \Saturn\xspace is the apheta), for a total of 255. \Jupiter\xspace will have this many days of \Saturn’s chronocratorship. 

Next in order \Venus\xspace receives the chronocratorship\footnote{He's skipped \Mars, who should follow \Jupiter.}: since
it controls 22 2/3 days, we will multiply this amount by 7 1/2, and we will find the total 170. \Venus\xspace will control this many days of \Saturn’s chronocratorship. And so on with each star: if we multiply its days by 7 1/2, we will find its allotment <in \Saturn’s chronocratorship>. 

If the \Moon, on the other hand, controls the vital sector, we multiply each star’s days by 6 1/4 <to find its distribution>. Similarly for the rest.

\subsection{To Find the Days of Each Star}

The days of each star are found in this way: double the star’s period, then take one-half, then one-third of the period. 
After adding all these figures together, we will find the days. 

The period of \Saturn\xspace is 30 days; I double this for a total of 60. One-half of 30 is 15; I add this to 60 for a total of 75. 

One-third of 30 is 10; I add this to the 75 for a grand total of 85. \Saturn\xspace will have this number of days. Likewise for the rest of the stars.

\newpage