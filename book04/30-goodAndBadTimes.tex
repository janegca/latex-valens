\section{Propitious and Impropitious Chronocratorships with Reference to the One-Fourth-Part of the Cycle}

\index{distribution!method!One-Fourth-Part}
The distributions of propitious chronocratorships with reference to the fourths:

\begin{center}
\begin{tabular}{ccl}
\textbf{Star} & \textbf{Period in Years} 
				  & \textbf{One-Fourth of Period} \\
\Saturn & 30 & 7 1/2 \\\
\Jupiter & 12 & 3 \textbf{/206K/} \\
\Mars & 15 & 3 years 9 months \\
\Sun & 19 & 4 years 9 months \\
\Venus & 8 & 2 \\
\Mercury & 20 & 5 \\
\Moon & 25 & 6 years 3 months \\
\end{tabular}
\end{center}

Those were the minimum periods; following are the maximum:
\begin{center}
\begin{tabular}{cr}
\Sun 		& 120 \\
\Moon		& 108 \\
\Saturn 	&  57 \\
\Jupiter &  79 \\ 
\Mars 	&  66 \\
\Venus	&  82 \\
\Mercury	&  76 \\
\end{tabular}
\end{center}
\newpage

<Saturn's> days are as follows:
\begin{center}
\begin{tabular}{cl}
\Saturn		& 637 \\
\Jupiter		& 255 \\ 
\Mars 		& 318 \\
\Sun 			& 403 \\
\Venus 		& 169 18 hours \\
\Mercury		& 423 days 18 hours \\
\Moon			& 531 \\
\end{tabular}
\end{center}

Jupiter from its 3 years distributes the days:
\begin{center}
\begin{tabular}{cl}
Itself		& 102 \\
\Mars 		& 127 days 12 hours \\
\Sun 			& 161 days 6 hours \\
\Venus 		&  67 days 12 hours \\
\Mercury		& 170 days 12 hours \\
\Moon			& 212 days 6 hours \\
\Saturn		& 255 \\
\end{tabular}
\end{center}

Mars from its 3 years 9 months distributes the days:
\begin{center}
\begin{tabular}{cl}
Itself		& 159 days 5 hours \\
\Saturn		& 318 \\
\Jupiter		& 127 1/2 \\
\Sun 			& 201 days 19 hours \\
\Venus 		&  84 days 18 hours \\
\Mercury		& 212 days 21 hours \\
\Moon			& 265 2/3 \\
\end{tabular}
\end{center}
\newpage

The Sun from its 4 years 9 months distributes the days:
\begin{center}
\begin{tabular}{cl}
\Saturn		& 403 \\
\Jupiter		& 161 days 12 hours \\
\Mars 		& 201 days 20 hours \\
\Sun 			& 255 days 18 hours \\
\Venus 		& 107 days 21 hours \\
\Mercury		& 269 days 4 hours \\
\Moon			& 336 days 6 hours \\
\end{tabular}
\end{center}

Venus - 2 years:
\begin{center}
\begin{tabular}{cl}
\Saturn		& 170 \\
\Jupiter		&  68 \\
\Mars 		&  85  \\
\Sun 			& 107 days 12 hours \\
Itself		& 45 \\
\Mercury 	& 113 days 12 hours \\
\Moon			& 141 days 12 hours \\
\end{tabular}
\end{center}

Mercury - 5 years:
\begin{center}
\begin{tabular}{cl}
\Saturn		& 425 \\
\Jupiter		& 170 \\
\Mars 		& 212 \\
\Sun 			& 269 \\
\Venus 		& 113 \\
Itself		& 283 days 12 hours \\
\Moon			& 354 \\
\end{tabular}
\end{center}
\newpage

The Moon from its 6 years 3 months <distributes>:
\begin{center}
\begin{tabular}{cl}
\Saturn		& 531 \\
\Jupiter		& 212 \\
\Mars 		& 265 days 15 hours \\
\Sun 			& 336 \\
\Venus 		& 141 days 16 hours \\
\Mercury		& 354 days \\
\Moon			&  442 \\
\end{tabular}
\end{center}

Wherever the year stops, the ruler of the sign gives its period first:
\begin{center}
\begin{tabular}{cl}
\Saturn		& 85 days \\
\Sun 			& 53 days 20 hours \\
\Mercury		& 56 days 16 hours \\
\Venus 		& 22 days 16 hours \\
\Jupiter		& 34 days \\
\Moon 		& 70 days 20 hours \\
\Mars 		& 42 days 12 hours \\
\end{tabular}
\end{center}

Another method: the number of terms which a given star has in the 12 signs will be the number of years of that star. For example: the Ascendant in \Libra. The nativity is in its 28th year <=336 months>. \Capricorn\xspace gives the first 57 to \Saturn, then 76 to \Mercury, 82 to \Venus, 79 to \Jupiter, 65 <?> to \Mars, 70
<?> to the \Moon, and 6 hours to the \Sun.

Another distribution: multiply the minimum period of the star by 4, and give 25 to the \Moon\, and 6
hours to the \Sun. \Saturn\xspace will have 120 days because of \Capricorn, \textbf{/197P/} then \Mercury\xspace 80, \Venus\xspace 32, \Jupiter\xspace 48, \Mars\xspace 60, the \Moon\xspace 25, and the \Sun\xspace 6 hours. Calculate the year from the Ascendant for day or night births. 

Someone is in his 28th year <with a transmission> from \Libra\xspace to \Capricorn. \Saturn\xspace is in
\Libra. The <count of> 28 stops in \Leo <sign of the \Sun>. Therefore \Saturn\xspace transmits to the \Sun\xspace in \Libra, and the \Sun\xspace has the year. This method is approved by Egyptians, Babylonians, and Greeks. They distinguish the triangles in year 28 as follows: \Venus\xspace is the ruler of the triangle. From \Venus\xspace in \Cancer, the count stops in \Libra. For day births, the ruler of the triangle is \Saturn\xspace in \Libra. \Saturn\xspace receives the year from \Venus. This distribution is at odds.

\newpage
