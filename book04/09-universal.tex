\section{The Universal Year. The Year with Respect to the Distribution: How Many Days It Has and How It Must Be Calculated.}

Since the universal year has 365 1/4 days, while the year with respect to the distribution has 360, we subtract the 5 intercalary days and the one-fourth of a day, then we find the number of years. Only then will we make the distribution. (We calculated in this way for the previous nativity.) 

For example: a person in his 33rd year was born on Tybi 15; we are investigating Mesori 20 of his 33rd year. I multiply 30 years
times the 5 $<$intercalary$>$ days for a total of 150. Now I add 10 $<$intercalary days$>$ for the two complete years $<$31, 32$>$ plus one-fourth of 32 (=8) for a total of 168. Next I take the number of days from Tybi 15 to the day in question, Mesori 20 (=215), and I add this amount to 168 for a grand total of 383. From this sum I subtract 360, and the remainder is 23. So the nativity will be in its 33rd full year with respect to the distribution, plus 23 days. I consider this number of years and days when making the distribution of the chronocratorships.

\newpage