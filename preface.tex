\chapter{Preface}
This is a copy of the \textit{Vettius Valens Anthologies} that Prof. Riley translated and kindly made freely available on his \href{https://www.csus.edu/indiv/r/rileymt/}{website} and which was re-formatted as a PDF file and made freely available by \href{https://beyondtheheaven.files.wordpress.com/2019/02/valens-atnhologies-1st-ed.-green.pdf}{Ile Spasev}.

I have taken a number of liberties with the layout, primarily to make the text easier to read. I have kept Prof. Riley's angle brackets ($< >$) which mark his clarifications within the text. I have also kept his references to the pages of the original Greek texts edited by Wilhelm Kroll (1908 edition) (\textbf{/\#K/}) and David Pingree (1986 edition) (\textbf{/\#P/}). 

The public domain \href{https://www.ctan.org/pkg/starfont}{StarFont Sans} astrological fonts used in this document were created by Anthony I.P. Owen and made available as a Latex package by Matthew Skala. 

This document was built from Latex files created with the \href{https://www.tug.org/texworks/}{TexWorks} editor.

\section{Annotations}

Annotations have been added primarily as margin notes and footnotes. 

In addition, comments/questions/musings/summaries or other assertions appear in \textit{italics} between square brackets. For example, ([\textit{my interjections}]). 

Any Table of Content entries that appear in bracketed italics are not part of the original text. For example, the entry under Book I, \textit{[General Indications]} is a section title not found in the original text.

\section{Abbreviations}
The following abbreviations are used in the text:

\begin{description}
\item[CA]
	\textit{Carmen Astrologicum} of Dorotheus of Sidon trans. by David Pingree.
\item[DF]
	\textit{Definitions and Foundations} trans. by Robert H. Schmidt. The text contains translations of texts from Thrasyllus and Serapio who were near contemporaries of Dorotheus.
\item[GH] 
	\textit{Greek Horoscopes} by O. Neugebauer and H.B. Van Hoesen. Reference numbers refer to the book's chart numbers i.e. L50, L113, etc.
\item[LH]
	\textit{Liber Hermetis} trans. by Robert Zoller. An anonymous collection of early Greek horoscopy attributed to Hermes Trismegistus.
\item[PT]
	\textit{Ptolemy Tetrabiblos} trans. by F. E. Robbins.
\item[VRS] \textit{Vettius Valens: The Anthology} trans. by Robert H. Schmidt.	

\end{description}

\newpage