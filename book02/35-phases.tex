\section{The Eleven Phases of the Moon and The Influence of Their Effects (35K,36P)}

According to the physicists’ reasoning, there are seven phases of the Moon, but we find eleven listed elsewhere:

\begin{enumerate}
\item New moon;
\item First visibility;
\item Next the crescent \Moon, 45° from the \Sun;
\item Next the quarter \Moon\, at 90°;
\item Next the gibbous \Moon\, at 135°;
\item Next the full \Moon\, at 180°;
\item Next the second gibbous phase when it is 45° from full, i.e. 225° <from the \Sun>;
\item Next the second quarter at 270°;
\item Next the second crescent at 315°\footnote{Balsamic \Moon.};
\item Final visibility at 360°;
\item There is another phase as well, when it first begins to wane.
\end{enumerate}

 \subsection{\textit{[What Each Phase Indicates and What Effects It Has]}}

\textbf{/107K/}\footnote{Promoted the first line of this paragraph into the above subsection.} 

We will append how the preceding phases are to be taken in casting horoscopes and to which god they belong.

The new moon is indicative of rank and power, of kingly and despotic dispositions, of all public business concerning cities, of parents, \textbf{/102P/} marriages, religion, and of all universal, cosmic matters. The rulers of the new moon, of the latitude, and of the motion are indicative of the same things. 

The first visibility of the \Moon\, (which is also called its “light”) and its ruler are indicative of life, occupation, and future wealth; in addition, it strengthens the matters influences by the n[e]w \Moon. The ruler of the “light” indicates the overall influences in the same way that the monthly cycles and the universal cycles are observed by means of the first visibility. \Mercury\, adds its influence until day 4 of the \Moon’s motion.

The crescent formation is indicative of nurture and expectations in life, of wives and mothers. \Mercury\, adds its influence until day 8.

The quarter formation is indicative of injuries, diseases, and violent accidents; also of children, status, and good things to come. \Venus\, is configured with the \Moon\, until day 12.

The gibbous phase is indicative of prosperity, future success, travel, and the affinity of relatives. The \Sun\, works with the moon until day 14.

The full moon is indicative of fame and infamy, of travel and violent events, of those who fall from preeminence as well as those who rise from a humble state, of affinities, passions, political opposition, and the affinity of parents. This phase has the color of the sign in the Descendant.

The first ruler of the waning of the light is indicative of the diminishing of resources, of the chilling of occupations, of those who grow humble and lowly, and of sudden falls. This phase has the same influence as the sign which just follows the Descendant. \Mars\, is its ruler until day 21.

The second gibbous phase is indicative of travel abroad, of great activities, and of prosperity. It has the same influence as <the IX Place of> the God. \Jupiter\, is its ruler to day 25 of the \Moon.

The second quarter phase is indicative of old affairs, of chronic diseases, and of children. \textbf{/108K/} It has
the same influence as\ldots \Saturn\, is its ruler to day 30.

The ruler of the last crescent is indicative of a wife’s death, of unemployment or robbery. 

Finally, the last visibility is indicative of chains, imprisonment, secrets, condemnation, and infamy.

The preceding was the arrangement of the \Moon’s phases, their relationships with the five gods and the sun in the\ldots angles\footnote{The following subsection heading was inserted by Prof. Riley.}.

\subsection{\textlangle A personal comment\textrangle}

Since I wished to set out brief explanations of these matters, and since I deprecate all long-winded, mythological mystification, I have published these chapters, most particularly for those who are vitally interested in these matters, those who have spent much time in their studies, and who, because of this, can
make an equal contribution from their own insights. \textbf{/103P/} I believe that I have persuaded these students, in what I have written and will write, to put aside the hard-to-believe and easily-ridiculed parts of our art, to convict our opponents of folly and mad raving, and to display the immortal foreknowledge which is <now> in danger. 

Eager scholars, exercised in the mathematical, introductory disciplines by other men, will win the victory-prize of glory with the help of this treatise—although they themselves are not
unfamiliar with the mysteries of constructing and arranging astronomical tables\footnote{A reference to Ptolemy?} (a subject which I did not want to go into and then have to repeat). Even if we seem to be <merely> compiling and explaining the doctrines of the old astrologers, <even> for this we will win the prize of merit from our readers, because of the precision, clarity, and instructiveness of their methods.

Others have employed long-winded, elaborate schemes, and although thinking that they have explained, have really overturned their existing reputation for foreknowledge. Trying to exercise a pure Hellenic style in their writings, they have revealed a thoroughly barbarian mind. One might say that they
act like the Sirens, who attracted sailors with their treacherous, but harmonious, voices and with the music
of instruments and of baneful song, then destroyed them on the reefs of the deep. This is what some men suffer and have suffered, men who fall in with the sects of those <other astrologers>: beguiled from the start by their spectacular words and their spells, they have become lost in a trackless wilderness, and finding no exit, they perish not only in the depths, but even in a maze. Some who think they have escaped \textbf{/109K/} this danger fall into tormenting, soul-wearying agony and come to a bitter end. If someone uses Odysseus’
scheme and sails past these “Sirens,” he will bequeath <to others> knowledge sanctified by his life, knowledge with which he can live and associate always, enjoying his span of days, while repelling the malignant opinions of his opponents as if by magic. So then, saying farewell to these men, we will reach the glory which lies before us.

\newpage