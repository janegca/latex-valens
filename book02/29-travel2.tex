\section{Travel (29K,30P)}
\index{travel}
\index{lots!Foreign Lands}
\index{lots!Daimon!travel}
The Lot of Foreign \mn{Lot of Foreign Lands} Lands is found by determining the distance from \Saturn\, to \Mars, then counting that same distance from the Ascendant. 

These circumstances make changeable nativities. The occasions
and the times of travel for such nativities will be evident from the configurations outlined by Abraham. We will add the configurations observed by me personally; let no one reading this criticize us for adopting as our own the work and observations of others—as some do. We testify to the work of these men. Let us return to our subject.

\index{distribution!timing method}
\index{lots!Daimon}
\index{timing!from Daimon}
To distribute the operative chronocratorship according to Abraham\footnote{This technique has not been explained as yet.}, i.e. those which are allotted starting with Daimon (for he does allot in this way, starting where the Lot of Daimon is located at birth): first look at the ruler of the sign where the Lot is found; then determine how many years \textbf{/97K/} its shortest cycle happens to be and divide that amount among the 12 signs starting from Daimon itself, counting through the signs in order.

Next, when that cycle is completed, look at the ruler of the next sign after Daimon, determine how many years its cycle happens to be, and divide this <among the 12 signs>. Do the same in the successive signs, if the nativity has any years of life remaining\footnote{Assumes a length of life calculation has been made.}. If the sign where the chronocratorship happens to be located has a place indicative of travel or the Lot of Travel either <in conjunction>, in opposition, or square, or if the stars (especially malefics not at an angle) which are in the signs that receive the allotment from the original sign, have more years than the <nativity’s> basis, then they cause travel.\footnote{This sounds as if you establish a sequence of signs ruling the years after birth beginning with the sign occupied by the Lot of Daimon and then for the year in question, see how the sign relates to the Lot of Travel.}

If the ruler of the sign which has received the allotment is not at an angle, is turned away from the sign, or is a malefic, it causes travel. Even if it is at an angle, it will do the same. 

If malefics have the allotment and are in the signs or are square with them, they cause travel. 

If a benefic receives the allotment, but is found in opposition to the nativity, it will cause travel and movement for the native. 

Again, whenever the rulers of the signs which have the chronocratorship or the distribution happen to be turned
away from their signs, or are in opposition or in inferior aspect, or are not at an angle, these signs cause
travel. Malefics in opposition, especially when beholding the luminaries in the places of the \Sun\, or \Moon,
\textbf{/93P/} also cause travel. 

If the ruler of the sign which has the chronocratorship is not at a center, or if it is in opposition to the sign, it makes movement or travel, provided that the Lot of Travel is located in the same place or in opposition or square with it. If it is in its proper place or is found in the squares, it does
not cause travel.

\index{planets!\Mercury!travel} \index{planets!\Venus!travel}
\Mercury\, and \Venus\, do not cause distant travel, but rather swift returns. 

\index{lots!Fortune!travel}
If the two Lots, Fortune and Daimon, fall in the same sign, and if the Lot of Travel is in opposition or square with this sign, and if some malefic is in this place, the native will be involved in travel. Likewise if the Lot of Travel is in opposition to the star which is the chronocrator or which is in conjunction with the Lot of Fortune, and if the two Places, Fortune and Daimon, are in opposition, it is the cause of movement and makes travel for the nativity, especially <if these Lots are> not at centers. Even if they are at angles, \textbf{/98K/} or else if the signs which have the allotments also have the Places of Foreign Lands or the Lot <of Travel> in opposition or square, they cause travel for the nativity. Likewise if they are at IC, they make men fond of travel. 

Again, if the Lot of Travel is located in the Ascendant at MC or just following MC—even if the nativity is not naturally inclined to travel or does not have the configurations mentioned previously—it still makes them travel not a lot, but a little, especially if no malefics are in opposition.

If the signs above the earth have the chronocratorship or its distribution\footnote{Is he talking about either the sign of the yearly profection or the sign found from the Daimon being a sign that falls in places above the horizon?} (apart from the XII or the IX Places), they do not cause travel, provided that the Lot does not cause it and that no malefic is in opposition to or conjunction with the sign with no benefics associated. 

If the signs below the earth have the chronocratorship, they do cause travel, especially when the Lot of Travel is located in the region below the earth. If the Lots of Fortune and Travel have malefics in conjunction or opposition, they cause frequent
travel. 

If the ruler of the Lot of Foreign Lands happens to be in opposition to the sign which has the chronocratorship, it will make the nativity travel. 

If the Lot of Travel and the Lot of Fortune are together at
IC, they cause much travel, especially when beheld or controlled by malefics or by a luminary. If it falls in
one of the signs which has been assigned either the allotment of the chronocratorship or the monthly period, it causes motion, especially if it has a malefic in opposition or if the luminaries are similarly situated and at the Places above the earth which precede the angles. If the allotment is less or in
opposition, \textbf{/94P/} the nativity will have intermittent travel. 

The stars in superior aspect to the moist signs under the earth which have the allotment cause travel, especially if these signs have the luminaries or malefics in conjunction. The configurations under discussion will be particularly influential if the current year <=chronocrator> of the Place has travel for the native or if it produces travelling nativities because of <the nativity’s> fundamental nature. 

Wherever the allotment of the overall chronocratorship or its distribution may be located, the ruler of that sign—whether at an angle or not, provided that no malefic is in opposition\ldots and that the ruler is not in one of the signs of the luminaries, \textbf{/99K/} departures will occur. If they have them at or just preceding an angle, they cause foreign departures.

Malefics in conjunction with signs that just precede an angle, or that have the chronocratorship or its distributions, cause travel, and the year has special movement. 

Whenever the star which has the chronocratorship or the ruler of the Place of Foreign Lands is found to be in the Place or Lot of Foreign Lands, it causes travel, especially if a malefic is square or opposed to the Unlit Place. Likewise if the sign which has the allotment of the chronocratorship is in opposition to the Lot of Travel, especially when it just precedes an angle, it causes travel. If the allotments of the chronocratorship just precede an angle and if the signs do not have <the Lot of Travel?> in opposition or in superior aspect, they do not cause travel; rather the native will nervously anticipate travel and will have unfulfilled intentions to travel. 

Whenever a benefic is in opposition or in superior aspect with malefics which have the chronocratorships or their allotments, or if the benefic is with <such a malefic> at IC and a travelling “year” occurs, they make delays and obstacles for departures occur.

If Fortune falls in the Place of Foreign Lands, or if the Lots are in opposition, with a malefic in conjunction or opposition, travel occurs, provided that no benefic is joined to any of them or in opposition. If the ruler of Foreign Lands is in opposition to it or to them <?>, with no benefics in aspect, and if \Mars\, is in opposition to the Lot of Fortune or is located in the Lot of Foreign Lands, or is situated in one of them, this makes the native travel extensively. On the other hand, \Mars\, as ruler of both Lots, even though turned away from the “signs” that cause travel or in moist signs, causes traveling nativities. 

If \Mars\, is turned away from the Lot <of Fortune> or is in the Lot of Foreign Lands, and it is the ruler of neither Lot, it does not cause travel, but it will cause the native to live mostly in his homeland, \textbf{/95P/} experiencing
only the threat of travel. Likewise if the Lots have benefics in conjunction, they do not make men who are subject to travel, but instead, those who rarely travel.

For any nativity it is possible to find configurations that do not <?> easily bring travel, because most nativities are subject to travel, some constantly and everywhere, others rarely \textbf{/100K/} and briefly, because in some <nativities>, the configurations which cause travel are in the majority, in others they are not. Therefore some men become much travelled, others are rarely, only briefly, subject to travel. Concerning those people who have a few configurations indicating travel: if in the original horoscope or in a later
recasting, the Lots of Travel and Fortune are located near benefics, they do not cause departures, especially if the year has no impulse toward travel. if the nativity has the configurations I mentioned above, they do cause travel. 

If the Lot of Travel is turned away from Fortune, especially if one of the Lots has a benefic in conjunction, it will make men spend most of their lives in their homeland, rather than be subject to travel.

\Mars\, turned away from the Lot of Foreign Lands causes short trips. The Lot of Fortune does not cause travel if it has benefic stars in conjunction and if they are above the earth.

If the Lot of Fortune is at MC and is turned away from the Lot of Travel, and if it does not have a malefic or a luminary in opposition, it causes nativities to stay in their homeland rather than to travel. If the two Lots are in conjunction and in the Place just preceding MC, separated from <\Mars> and having no malefic in opposition or conjunction in another sign, the nativity does not readily travel. But if the two Lots have malefics in conjunction or opposition, they cause nativities to be subject to travel, especially when the Lots are in moist signs. 

The Lot of Fortune, when well situated, having neither malefics in superior aspect nor the luminaries nor the Lot of Travel (especially when \Mars\, is turned away from the two Lots), does not cause travel. Even if the native wants to travel, he does not go. If \Jupiter\, transits these “signs,” he prevents departures. 

The native will have remarkable travels if the year falls just before an angle (Ascendant) and in moist signs, especially
if a benefic is not in conjunction (either in transit or at the nativity). If the ruler of the chronocratorship transmits the year to the ruler of the Lot of Travel, especially if a malefic beholds it, or vice-versa if the ruler of the Lot <transmits> the year to the ruler of the chronocratorship, <it causes travel>\ldots


\newpage