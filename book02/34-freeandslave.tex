\section{Free and Slave Nativities (SAN) (34K,35P)}
\index{SAN}
\index{rank}
Differences in birth, i.e. servile and noble nativities, are determined from the phases\footnote{Robert Hand says ``phase'' here refers to the syzygy, the New or Full Moon that occurred before birth (VRS2 p70 footnote 3).} of the nativity.

If \textbf{/106K/} the sign of the phase or the ruler of that sign is unfavorably situated or is beheld by malefics, the
native will be base. Even if he attains high rank and a position of trust, he will be ruined. 

If the phase is found at an angle, and its ruler has benefics in aspect, the native will become noble and famous. 

If the place <of the phase> is in operative signs, but its ruler is unfavorably situated or beheld by malefics, \textbf{/101P/} then the native is free-born and begins life well, but he later falls into reversals, servitude, and want. 

If \mndl the ruler is found in operative signs, but the place itself is unfavorably situated, the native will fare very ill in his first years in servile roles and will be unsettled, but later he will live easily, be elevated, and gain freedom, success, and a family <name>, especially if benefics incline. 

If the place <of the phase> and its ruler are unfavorably situated and both are beheld by malefics, the native will be exposed as an infant, or will become a captive and experience servitude. If under these conditions, benefics come close or are in coaspect, the native will be released from servitude after the chronocratorships of the malefics and will become
a man of property. 

If the place is guarded by malefics, but its ruler by benefics, the native, although of servile birth, will be raised as a free-born man, or will rise by being adopted or taken in <to be reared>. If the opposite is true, i.e. the place is guarded by benefics and the ruler by malefics, the native, although freeborn, will be reduced to slavery or will hand himself over, in middle age, into slavery for lack of food or as a means of gaining a position and occupation.

\newpage