\section{The Triangles \textit{[Triplicities]}}

When the zodiacal circle is subdivided according to similarities and differences, we find two sects, solar and lunar, day and night. 

\marginnote{Fire \\ \Aries-\Leo-\Sagittarius}[0.2cm]
The Sun, being fiery, is most related to \Aries, \Leo, and \Sagittarius, and this triangle of the \Sun\xspace is called “of the day-sect” because it too is fiery by nature. The \Sun\xspace has attached \Jupiter\xspace and \Saturn\xspace to this sect as his co-workers and guardians of the things which he accomplishes: \Jupiter\xspace as a reflection of the \Sun\xspace and as his successor to the kingship, a partisan of good, and the bestower of glory
and life, \Saturn\xspace on the other hand as a servant of evil and of downfall, and a depriver of years <of life>. Therefore the \Sun\xspace is the lord of this triangle for day births; for night births \Jupiter\xspace succeeds to the throne; \Saturn\xspace works with both.

\marginnote{Earth \\ \Taurus-\Virgo-\Capricorn}[0.2cm]
Next the Moon, being near the earth, is allotted the house rulership of \Taurus, \Virgo, and \Capricorn, a triangle earthy in nature and the next in order. It has \Venus\xspace and \Mars\xspace as members of the same sect: \Venus\xspace
(as is reasonable) acts as a benefactor and distributes glory and years; \textbf{/56K/} \Mars\xspace acts as the bane of
nativities. Therefore for night births the \Moon\xspace has preeminence; in the second place is \Venus; in the third
is \Mars. For day births \Venus\xspace will lead; the \Moon\xspace will operate second; \Mars, third.

\marginnote{Air \\ \Gemini-\Libra-\Aquarius}[0.2cm]
\textbf{/55P/} Next is the airy triangle of \Gemini, \Libra, and \Aquarius. For day births \Saturn\xspace will rule this;
\Mercury\xspace will operate second; \Jupiter, third. For night births \Mercury\xspace will lead; \Saturn\xspace will come second; \Jupiter, third.

\marginnote{Water \\ \Cancer-\Scorpio-\Pisces}
In the same fashion, next is the moist triangle of \Cancer, \Scorpio, and \Pisces. \Mars\xspace will have the house rulership for night births; in the second place is \Venus; in the third the \Moon. For day births \Venus\xspace will lead; after it comes \Mars; then the \Moon. 

\mndl[0.2cm]
Note that Mercury is common and works with the two sects to a special degree to accentuate the good or the bad, and to accentuate the individual characteristics and configurations of each star.

\newpage