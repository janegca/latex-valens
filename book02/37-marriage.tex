\section{Marriage, Wedlock, and Happiness. Various Theorems and Configurations (37K,38P)}

\textbf{/109P/} I have published, with explanations, all the methods which seem from my experience to be true. Now
I will explain the Place/topic of Marriage, complicated to be sure, but easily grasped by those who pay attention.

Scientifically speaking, the Place of Marriage is considered to be the VII Sign from the Ascendant. It is also necessary to look at the location of Venus, the other stars in <\Venus’s> sign, as well as its aspects and rulers. 

If Venus is operative in a tropic or a bicorporeal sign, especially for night births, it makes men oft-married and promiscuous, particularly if \Mercury\xspace is in conjunction or—even more so—if \Mars\xspace is in aspect: then the native also rushes into the embrace of boys. If the sign is virile, the native is successful in gaining the object of his passions. 

If Venus is operative, \textbf{/115K/} and its ruler is either setting or in <the XII Place of> Bad Daimon, or if it is a malefic and afflicts \Venus, or if it is wretchedly situated, it makes men unlucky in marriages and relationships. 

If a malefic nullifies \Venus\xspace or—more particularly—
its houseruler by being in aspect <with it>, it will cause deaths, injuries, or other problems to wives. If they are well situated at the nativity, they bring inheritances; if they are badly situated, diseases and pains.

If Saturn beholds the setting \Venus, in most cases it makes unmarried, unsociable men. If \Venus\xspace is in the sign or terms of \Saturn, or if \Venus\xspace has \Saturn\xspace directly on the line of opposition, with neither \Mars\xspace nor \Jupiter\xspace in aspect nor \Mercury\xspace in conjunction with \Venus, the native will be a widow or a virgin. 

In all cases Saturn, when in opposition to \Venus, brings sickly or barren wives—or for wives, brings sickly and
sterile husbands. 

If Saturn is at MC and is in opposition to \Venus, it brings wives who are slaves. 

If Venus is in the house of \Saturn\xspace and has
\mn{Applying and Separating Aspect} \Jupiter\xspace in aspect, or if \Venus\xspace is leaving \Jupiter\xspace and making contact with \Saturn, or becoming attached to \Saturn, and is beheld by \Mars, in these cases the native will lie with his nurse, with the wives of his tutors, with stepmothers,or with uncles and aunts. 

If the Sun is also in aspect with them or with the \Moon, the native will be involved in perversion even more, \textbf{/110P/} especially if the \Moon\xspace is in aspect with them or is aspected by them. 

Venus in conjunction with \Saturn\xspace in the
Descendant or at IC brings the native a marriage below his station and causes him grief in this marriage. 

Generally speaking, all those who have \Venus\xspace in conjunction with \Saturn, as a houseruler of \Saturn, or in superior aspect with \Saturn, and who have \Jupiter\xspace in aspect with both, are united with prominent or elderly women. If the native is a woman, the same applies to her.

The Moon and \Venus\xspace at the same angle unite the native with brothers or sisters, especially if \Jupiter\xspace
and \Mars\xspace are also in aspect. The \Moon\xspace and \Venus\xspace square or in opposition make men jealous; \Mars\xspace in aspect as well intensifies the jealousy. 

The Moon and \Venus\xspace trine in their own houses, especially at angles, cause the marriage of relatives; even more so if \Mars\xspace and \Jupiter\xspace are in aspect. 

The Sun in its own house or exaltation and in conjunction with \Jupiter\xspace and \Venus\xspace \textbf{/116K/} causes marriage with the father’s relatives. 

Venus in its own house or exaltation <or> terms and in conjunction with \Mercury\xspace and the \Moon\xspace
causes marriage with the mother’s relatives. 

Venus at IC with the \Moon, or \Venus\xspace and the \Moon\xspace in opposition, particularly with one at MC, the other at IC, causes marriage with siblings or relatives.

In all cases Saturn in superior aspect, in opposition, or in conjunction with \Venus, or as \Venus’s houseruler, chills or contaminates marriages, especially if \Mercury\xspace is in aspect. 

Saturn in aspect with \Venus\xspace at or just following an angle causes shameless, degrading, rebellious marriages, those involving the low-born or slaves, for whose sake the native is snowed under with trouble, unless some star intercepts and cancels the malign influence. If \Jupiter\xspace is in aspect, most of the marriage’s irregularity will be hidden and there will be no shame; the native will lie with prominent women, with women of high status. He will not have many children; his partners will be barren or conceive only with difficulty, and if they do conceive, they will miscarry. Apply similar reasoning to female nativities.

If Saturn is in aspect with \Venus\xspace or in \Venus’s terms, and if \Venus\xspace itself is configured with \Jupiter\xspace and \Mars, the native will achieve success with the help of children or females and will see \textbf{/111P/} prosperity, but \mn{Planet Dignity} he will fail utterly in the end, unless the stars happen to be operative in their own houses or
exaltations. 

If \mn{Rays, Figure, Aspect} the Moon is struck by \Jupiter’s rays or if it is configured with \Jupiter, and if \Saturn\xspace is in aspect along with \Jupiter, the native will live with a low-born, purchased women\footnote{``Rays'', ``configured'' and ``aspect'' are all used in a slightly different sense, why?}(\Saturn\xspace harms social standing.) In this configuration <concerning marriage>, if \Venus\xspace is in its exaltation and has \Jupiter\xspace in aspect, the native will become successful and propertied, and will be acknowledged on the part of great men, because of \Venus. Again in this arrangement, if \Mercury\xspace is in the configuration as well, the native will be vigorous, shrewd, intelligent, and charming; he will also be promiscuous and unstable in his marriages. 

Generally speaking, Jupiter in aspect with \Venus\xspace from the right, being familiar with \Venus, or in agreement to the degree\mn{Exact Degree}, causes sociable men, those helped by women (or for women, those helped by men). Even if \Venus\xspace is afflicted, <\Jupiter> helps so that not the native is not entirely ruined.

Venus at an angle (especially in the Ascendant or at MC) and unafflicted by \Saturn\xspace makes men happy
in their marriages. 

\textbf{/117K/} Venus with \Jupiter\xspace in aspect restrains any malign influence so that no disaster occurs, and it causes affinity and marriage. 

Venus in <the XII Place of> Bad Daimon, in its own house or exaltation, with \Jupiter\xspace in superior aspect, or beheld by \Jupiter\xspace in trine, makes a good marriage, but it will bring the grievous death of a good wife. 

If Venus and Saturn are in Bad Daimon and \Jupiter\xspace does not behold them, the native becomes a widower or unhappily married, distressed by death and desertion. If in the preceding configuration (i.e. \Venus\xspace in Bad Daimon without \Jupiter\xspace in aspect) a malefic like \Mars\xspace is in aspect, the native becomes an adulterer or a victim of adultery, a dirty, unlovable man and consequently drawn into difficulties. 

In all cases, <malefics> in conjunction with or opposition to \Venus\xspace cause separations, deaths, or grief-producing unions—or even worse, if they afflict the \Moon\xspace as well.

The Moon setting under the rays of the \Sun\xspace is not good for marriages. 

Mars in conjunction with \Mercury\xspace causes adultery, whoring, lechery; if their sign is tropic or bicorporeal, it causes even worse: the native sins more often, he casts his eyes everywhere but does not attain his desires. Sometimes he lies with people like himself and suffers the terrible things at their hands that he had done <to others>. Even
worse happens if \Mercury\xspace is in aspect with them. (The same happens in the case of feminine nativities.) If \Saturn\xspace is also in aspect, even more occurs: \textbf{/112P/} the native is treated ungratefully even when he is kind to women, so much so that he plots against them as a result of their ill-treatment. Wives also suffer this at the hands of their husbands.

If Mars and Venus are setting under the rays of the \Sun, they cause sneaking adulterers and secret sins.
If these stars are rising or at angles, the sins are more public. If \Mercury\xspace is in conjunction and rising with
them, the adultery and the public outcry will be rather dangerous. If \Jupiter\xspace is also in aspect, the native
escapes; if not, he will be seized and murdered, if he is fated to this sort of death. If he is not, then he will
avoid death by paying a great ransom. 

If Venus is unfavorably situated with \Mars\xspace in <the XII Place of> Bad Daimon, and if both of them are operative, not in their own sects, or if they are in the Descendant or in the house of another member of the same sect—for a native with this chart, the ruin will be more terrible,
the adultery will be even more hazardous, the outcome will be murderous. 

If Mars and Venus are in divorce, unpleasantness, jealousy, and anger, and they bring in succession even more plots and dangers. Because of their aspect with \Mercury, rebellious sins follow. The native is united to slaves and servants, is promiscuous, whores around, and becomes notorious. He is seduced by friends, slaves, and enemies, and is involved in riots and murder by poison.
\Jupiter\xspace in conjunction or in aspect with \Venus\xspace causes the above mentioned effects, but they are secret. The native makes progress toward greater property—especially if \Jupiter\xspace is at morning rising or is at an angle.

Whenever Mars is in aspect with \Venus\xspace and in harmony with it, the native is united as a result of adultery. 

Whenever Mercury is in aspect with Venus rising, with \Saturn\xspace having nothing in common with <them or> the houseruler, the native is joined to a virgin or to a young women. If \Mars\xspace beholds, this is even more true. If \Jupiter\xspace beholds, this is positively certain. 

It is generally true in all cases that Mercury in aspect with \Venus\xspace involves and unites the native with those who are young and of a lower class. Men and women with this nativity <do the same thing>. 

If Mars is together or square with \Venus, it makes
adulterers, lechers, involvement with the base-born, criticism, divorces, and the deaths of mates. It is worse if \Saturn\xspace is in opposition: this configuration unites the native with elderly or barren women; if \Jupiter\xspace <is in opposition>, with women of high rank. 

If Saturn is configured with Jupiter while \Jupiter\xspace is in conjunction with \Venus, the native lies with prominent women or noblewomen. (The same apples to
women, but in addition, \textbf{/113P/} whenever \Mars\xspace and \Mercury\xspace are estranged from \Venus, the women are
spinsters, marry late in life, and are abstinent and chaste.) 
If \Saturn\xspace and \Jupiter\xspace are in conjunction or trine with \Venus, these results are more certain. 

Those masculine nativities which have Venus rising as a morning star can command women; those which have \Venus\xspace under the rays of the \Sun\xspace are commanded by women. The reverse is true for feminine nativities.

Calculate the Marriage Lot\mn{Marriage Lot} as follows: for day births, determine the distance from \Jupiter\xspace to \Venus\xspace (for night births, from \Venus\xspace to \Jupiter), then count this distance from the Ascendant. \mn{Lot of Adultry}The point in opposition to this Lot is indicative of Adultery. If the ruler of the Marriage Lot is found in opposition, and if the ruler of the Lot of Adultery is in the Marriage Lot, the native will constantly commit adultery, then be reconciled, then having \textbf{/119K/} reconciled, be separated, then again rejoin his mate in the course of his adulteries. 

If the ruler of the Marriage Lot is at morning rising, the native will marry at an early age; if it is at evening rising, he will marry late. 

If the ruler is operative while setting, the native will have a jealous or an illegal marriage. 

The ruler of Marriage causes the first marriage, the benefics in harmony with the Marriage-bringer or its ruler also cause marriages, especially if the signs of the stars in aspect or of the Marriage-bringer itself are bicorporeal.


\newpage