\section{Propitious and Impropitious Periods. The Length of Life Calculated from the Angles and the Signs Following the Angles (27K,28P)}
\index{timing}
The periods of good or bad fortune, of failure or success, must be determined by using the rising times of each sign or the cyclical period of each star\footnote{Valens does not give example charts here but in the Schmidt translation Hand says the timing methods described here can be found in examples in Book VII.}.

\index{length of life}
When investigating the length of life, it is necessary to pay attention to the Ascendant and the \Moon, or to the signs in which their rulers are located. 

\index{rank}
With respect to occupation and rank, it is necessary to pay attention to the Lot of Fortune, to Daimon, to the \Sun, to the new or full moons, and to the Exaltation and its ruler. 

The stars which are in the Ascendant (viz. the most important relationship), begin to rule over the first period of life <=first chronocratorship>. The stars at MC, at the Descendant, or at IC <rule over the subsequent periods>. If these places happen to be empty, then the stars just following the angles <rule>. If these too are empty, then the stars just preceding the angles <rule>. Even though they are not too strong, they will regulate affairs. \textbf{/95K/} (The stars inclining away from <=just preceding> the Ascendant or MC make the allotment first, then the stars preceding the
other angles. They cannot allot their entire rising times or periods, but only an amount proportional to the amount of the sign that they control.) 

Those stars which are in their proper place and at angles or just following an angle, and which are found to be rising, especially those which have some relationship with the business of the nativity, whatever that may be—whenever they control the previously mentioned places, they allot the rising times of their signs and their own periods, or the rising times and periods of the signs in which their rulers are located. \mndl (Likewise, when investigating the remaining Places and their masters, it is necessary to interpret the chronocratorships (e.g. concerning livelihood, brothers, parents, children, \textbf{/91P/} etc.), the harmful and helpful stars, and whatever influence each <star> can produce. We mention this so that we do not write too often about the same matters. Their natures have been explained; we will remind you of them in the rest of this work.)

\index{lots!Fortune}
It is necessary to allot first the minimum period of the ruler <of the sign> and of the star in conjunction, next the rising time of the sign <itself> or of the sign in which its ruler is located. In addition, examine the houserulers of the triangle, as we mentioned above. If both are well situated, the chronocratorship will be noteworthy and beneficial. If the indications are mixed, the results will be the same. If they are badly situated, the nativity will be irregular from beginning to end, involved in griefs and dangers. But if the Lot of Fortune or its ruler is configured in its proper place, it will give the nativity prosperity and a high rank suitable to the <nativity’s> basis. 

If two or more stars happen to be in the same sign, the period of each, distributed consecutively, will be operative, but the effects will result from a mixture of the two or three stars. Likewise the rising time of the sign, distributed consecutively with the period of the star in conjunction (or its ruler), will be operative. 

If the chronocratorship derived from the rising times and the periods of benefic and malefic stars and signs coincide, then both good and bad together will happen at that particular time. \textbf{/96K/}

\newpage