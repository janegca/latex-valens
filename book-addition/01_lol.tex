\section{The Length of Life, With Reference to the Tables of Chapter 12. The Determination of the Degree in the Ascendant Using the Two Tables of Valens in Chapter 12}

First of all, determine the Sun’s degree-position accurately. Then it will be necessary to examine the preceding new or full moon. See how many days and hours there have been from the preceding new or full
moon, and see what fraction this is of 15 days (the period from new to full moon or full to new). 

Having found this fraction, note it separately, and investigate whether it should be added or subtracted in the following way: it is necessary to take the distance from the Sun’s degree-position to the Moon’s (in the
order of signs), with reference to the rising times of the correct klima. Register the total number of degrees as the solar gnomon. 

Next, again enter the table of the rising times for the correct klima, and examine the hourly magnitude of the Sun’s degree-position. (This is for day births; for night births examine the magnitude of the point in opposition to the Sun.) Multiply this by 12, then multiply the result by the hour/time of birth. If the result exceeds 360°, subtract a 360° circle, and see if the result <=the horoscopic gnomon> corresponds to the solar gnomon. If it does, the reported hour will be in agreement with the facts and should be used. 

If the horoscopic gnomon exceeds the solar gnomon, subtract from the reported hour a fraction of the hourly magnitude of the Sun, which fraction is derived from the period counted from the new or full moon to the day and hour of the birth (i.e. the calculated fraction of 15 days).
For example: a nativity at the third hour of the day; there are five days from the new or full moon to the day and hour of the birth, a figure which is one-third of 15 days. The hourly magnitude of the Sun is
16. Subtract one-third from this magnitude, which is then… You make it 2 hours total. In this way you will calculate the Ascendant. 

If the solar gnomon exceeds <the horoscopic gnomon> then add to the
reported \textbf{/350P/} hour and make it 31/3 hours. In this way you will calculate the Ascendant according to hourly magnitudes and rising times.

Another method: calculate the distance from the new or full moon to the Moon’s present position. If it is less than 180°, multiply it by 12, and see what fraction the result is of 15 days. If it is found to be greater than 180°, subtract 180° and \textbf{/365K/} see what fraction the remainder is of the Moon’s motion. Deduct this from the hourly magnitude.

Example 2: Diocletian year 147, Tybi 14/15, the third hour of the night, klima 4; the \Sun\xspace in \Capricorn\xspace 19° 2'; \Moon\xspace in \Taurus\xspace 23° 30'; the <preceding> new moon in \Capricorn\xspace 9° 29'. From the new moon to the day and hour of birth (Tybi 14/15) are 10 days, i.e. two-thirds of one hour\footnote{Rising times from the sun to the moon=21. - marginal note}. I investigate how many rising times there are <=what the difference is between the hourly magnitudes> from the Sun to the Moon in klima 4, and I find 90 time degrees. I record these 90 time degrees as the solar gnomon. 

Next I take the hourly magnitude of the point in opposition to the Sun, \Cancer\xspace 19° (because it was a night birth). Its hourly magnitude is 17;53. I multiply this figure by 12 and find 215. I multiply these 215 time degrees by the 3 hours of the nativity for a total of 645. I subtract a 360° circle for a remainder of 285 time degrees. I record this as the horoscopic gnomon. 

Now since the horoscopic gnomon exceeds the solar gnomon, the
third hour (the time of birth) requires subtraction. So I subtract two-thirds of an hour, because there were 10 days from the new moon to the day and hour of the birth. Do not calculate using the third hour, but
using 21/3 hours. Proceed as follows: \Sun\xspace in \Capricorn\xspace 19° 2'; the hourly magnitude of \Cancer\xspace <19°> (because of the night birth) is 17;55. Multiply by 21/3 hours for a total of 41;48. Add the rising time of \Cancer\xspace (93;7) for a grand total of 134;55. With this figure I enter the table for klima 4, and I find the Ascendant to be \Leo\xspace 23° 0'. With this 23° I enter the table for \Leo\xspace and I find written there 22 sixtieths, which is 1/3 1/30 and which equals 73 years 11 months. 

I investigated the proportional part as follows: the Ascendant in \Leo\xspace 23°, \textbf{/351P/} the hourly magnitude 16;45. I multiplied this by 12 for a result of 201. I multiply this by 1/3 1/30: 1/3 times 201 gives 67 years; 1/30 times 201 gives 6 2/3 1/30 years. The
grand total is 73 years 2/3 1/30 years <73;42>.

Example 3: the reign of Valentinianus, klima of Spain <=4>. The native was killed in his 36th year. Year 135 of Diocletian, Epiphi 8, the beginning of the first hour. The \Sun\xspace in \Cancer\xspace 7° 11'; the \Moon\xspace in \Aries\xspace 22° 30'; the Ascendant in \Cancer\xspace 7° 20'. The <preceding> full moon was at \Gemini\xspace 18° 40', at the seventh hour of the day on Payni 29. From the full moon to the day of birth were 71/2 days, which is one half of 15. Since the solar gnomon exceeds the horoscopic gnomon, I add this fraction, 1/2, to the initial hour of birth, the first hour, and calculate as follows: the hourly magnitude of the \Sun\xspace <in \Cancer\xspace 7°> is 18;5, half of which is 9. The rising time of the Sun is 79;7, for a total of 88 time degrees. With this figure I enter the table and I find the \textbf{/366K/} Ascendant very nearly at \Cancer\xspace 14°. I enter the table at this figure, 14°, and I find 34 years 10 months written there, next to the figure 10, which is 1/6 of 60. I now take the hourly magnitude of \Cancer\xspace 14°, and I find it to be 18;0. Multiplying this figure by 12, I find 216;0 time degrees. One-sixth of this is 36 years, his length of life.

\partialsecbr

<The next chapters are to be added here>

\newpage