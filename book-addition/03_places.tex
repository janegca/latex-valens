\section{The Zodiacal Places of the 12 Signs; Their Employment For The Length of Life with Respect to the Minimum, Mean, and Maximum Factors (7P)}

\textbf{/367K/} Most <astrologers> assign the chronocrators for every nativity using the 7-zone system. They begin by giving the rulership first to Saturn, then to Jupiter, to Mars, and to the Sun, after which comes
Venus, Mercury, and the Moon. Similarly in the rotation of chronocrators, they examine the ruler of the week and of the days. 

I do not like this procedure because the same chronocrators are found in most nativities. One should (as was described) put the aphetic point at the Sun or the Moon, or at the star immediately following the Ascendant, then <assign the rulership> in the order that the stars are situated by sign and degree in the particular nativity.

\textbf{/359P/} A handy way of making the allotment is as follows: reduce the months of each ruler of the 10-year-period <=129 months> to days, and divide them by 129. Multiply the result of this division by the months of each ruler, and you will have the days applicable to each one, viz. the days of the rulership of each star (which were reduced to days and divided by 129). 

For example, Saturn rules 30 months. Make the distribution to it and to the other stars as follows: take the 900 days <=30 months times 30 days> and divide it by 129. You will find the result of the division will be very nearly 7 <10-year-periods>. Now it is necessary to multiply <by 7> the number of months of each star and to find the days which Saturn will give to itself and to the other stars from its own allotment:

\begin{tabular}{ll}
210 to Saturn & 56 to Venus \\
84 to Jupiter & 140 to Mercury \\
105 to Mars  & 175 to the moon \\
133 to the sun & \\
\end{tabular}

When Jupiter is calculated, the result of the distribution of its 360 days <12 months times 30 days> divided by 129 is 2 1/2 1/3. This figure is multiplied by the number of Jupiter’s months (and by those of
the others) in order to find the days which Jupiter allots to each, as follows:

\noindent approximately 7 to Saturn \\
2 1/2 1/3 <=2 5/6>to Jupiter \\
3 1/2 to Mars \\
1 1/2 1/3 to Venus \\
4 1/12 1/21 to Mercury \\
5 1/12 1/14 to the Moon \\
4 1/2 to the Sun. 

The explanation is quite clear: 
the ratio 129 days:30 days : : 30 months:210 days <=7 months>
holds <for Saturn>. Do likewise for the rest of the stars.

Now if we investigate the distribution of these very 210 days, again we apply the factor 129. There will be the ratio 129 days to 210 days, and so for Saturn the 210 days will be 1 2/3 of 129. So if we
calculate 1 2/3 of 30, we will have 50 days for Saturn. 

1 2/3 of Jupiter’s 12 will give us 20 days for Jupiter, i.e. the days which Saturn assigns to Jupiter. 

1 2/3 of Mars’ 15 gives 25, and so on.

Next let us make a distribution of Saturn’s 50 days: as 129 is to 50, so Saturn’s 30 will be to some figure, Jupiter’s 12 will be <to some figure>; for the others stars, the correct period <will be derived from> 129.

\textbf{/360P/} The subdivision to shorter time periods for these stars (when you get down to fractions) becomes difficult. The other method, which distributes the days and the hours to each star according to the ratios of months, is easy and has logic behind it. Take for example Saturn: in the 129 months of the 10-year-period, Saturn is ruler of 30 months. Similarly in any given 129-day-period during the 30 months which it rules, Saturn will be the \textbf{/368K/} ruler of 30 days. Again in the 30 days (which equal 720 hours) which it rules, Saturn will be the ruler of 30 hours. As 129 months are to 30 months, so 129 days are to 30 days, and 129 hours are to 30 hours. 

For example: let the <aphetic> luminary be the Moon. Let the Moon have the first calculation and the first 10-year-period. Assume that following it in the nativity come Saturn and Jupiter in that order, then
Mars, the Sun, Venus, and finally Mercury. Now since the moon is the aphetic point, it has 25 months, which equal 760 days <25 months times 30 days plus 10 intercalary days>. If the infant is 40 days old, 25 days must be given to the Moon, then the next 15 to Saturn, because Saturn has the days until day 55. We say, “The Moon has transmitted to Saturn.” 

Now if the time in question is more than 55 days, e.g. 60 days, then Jupiter has the days <from 55> to day 67, because it rules 12 days. 

If the time is question is 70 days, Mars, next star, must be examined; Mars rules 15. So Mars receiving from Jupiter has the days <from 67> to day 82. 

If the time is question is more than this, e.g. day 95 (if Venus were the next ruler) then the star following Venus would be ruler, because Venus rules the next 8 days. 

If the Sun is next, it has the days <from 82> to day 101, because it rules 19 days. 

If the time in question were day <120> and if Venus were after the Sun, then Venus’ days would have passed, because Venus rules to day 109.

Consequently Mercury rules <from day 109> to day 129. 

If the time in question is more than 129 days, the Moon receives control again.

So the completion of 129 days in called a “period.” It cycles through the 7 stars, comes back to the first one again, and keeps the same distribution, through the second 129-day period, the third, the fourth,
however long it takes to complete the moon’s 760-day period. This 760-day period is completed after the fifth period plus 115 days into the sixth. As a result 14 <additional> days remain in the sixth period, of which Saturn is the ruler, being the lord (following the moon) of the next 30 months in the monthly periods, and of the next 30 days in the daily periods. 

\textbf{/361P/} So when the Moon has completed its 25 months, Saturn succeeds with its 30 months and gives to itself—not as some say, 210 days=7 months (for why should we apply a distribution of different months, and not a distribution of days derived from the
months?) No, it gives itself 30 days, derived from its 30 months, then to Jupiter after it 12 days, then to Mars (if it is the next star) 15 days, then to the Sun (which happens to be next) 19 days, then to Venus 8 (if Venus is next), then to Mercury 20, and finally 25 to the Moon.

Since the first 129-day period is completed, Saturn next after the Moon again receives 30 days of the second 129-day period, then the other 7 stars in order in their correct sequence. After the second period is completed, Saturn will again receive the 30 days first, then the other stars in order until the third period is completed. And so on for the fourth, fifth, sixth, and seventh, until the completion of 30 months, i.e. 910 days. Only 7 days remain after the completion of the seventh period.

Jupiter, the next star after Saturn, then receives 12 monthly time periods. From this period it gives itself the first 12 days, then to Mars, then to the Sun, which is next, then to Venus, then to Mercury, next to the Moon, and finally to Saturn. 

In the same way Jupiter gives the first 12-day interval of the second
period to itself, then distributes to the rest. When the second is completed, the third does not finish \textbf{/369K/} but runs for only 107 days. So <in the third period> Jupiter receives the first 12 days again, following Saturn, then Mars next in order receives 15. The Sun receives 19 following Mars, then Venus 8, Mercury 20, and the Moon 25. Finally Saturn receives the final 8 days to complete 107.

After this comes the monthly distribution of Mars, 15 months, which equals 455 days when broken down to the daily distribution. First Mars will have 15 days, then the rest in order. Do likewise for the
second and third periods. There will be 68 days left in the fourth period. Of these, Mars receives 15 days first, the Sun 19, Venus 8, Mercury 20, and the Moon the final 6. 

In all cases, when the year has 6 intercalary days <rather than the usual 5>, add one day to the star which ends <the sequence>; in this case we add one day to the Moon’s 6 days.

After the monthly and daily distributions of Mars are completed, the distribution of months from the same 10-year-period passes to the next star, the Sun. It \textbf{/362P/} will receive 19 months, which equals 575 days when broken down as before. If the year is a leap year, one more day must be added to the 575. 

Of this period, the Sun receives 19 days first, then Venus, Mercury, the moon, next Saturn, Jupiter, and finally Mars. After the first period of 129 days is completed, the Sun again allots itself 19 days, after which the other stars in their order at the nativity receive their days. There are 4 periods. Since there are 59 days left in the fifth period, it is obvious that the Sun receives <the first> 19 days of this period, then Venus 8, then Mercury 20; the Moon receives the remaining 12 (or 13 if a leap year occurred in the 19 months).

Proceed in the same manner, giving the months and the days derived from them to the stars in order, until the first 10-year-period, that of the Moon, is completed. After that, Saturn will receive the second 10-year-period, since it is next in order following the Moon. Saturn will begin the second 10-year-period by giving itself 30 months and 30 days of the 30 months, then the appropriate number of months and days to the next star, and so on until the daily, monthly, and 10-yearly periods of Saturn are completed. Then the next star begins another 10-year-period, with the appropriate number of days and months. From this period
it will allot <days and months> in the preceding pattern until the end of the nativity’s years. 

For example: the lifespan of a nativity is 45 years, 9 months, 25 days. Assume the Sun receives the first 10-year-period, with the Moon following next, then Mars, Mercury, Jupiter, Venus, and finally Saturn. 43 years are four 10-year-periods. The remaining 2 years, 9 months, 25 days belong to the fifth 10-year period, which is ruled by Jupiter, following the Sun in the fifth place. From its monthly period, Jupiter gives 12 months to itself and 8 months to Venus. 13 months and 25 days remain until the day in question. Saturn, the next star after Venus, has 30 months, of which 13 months, 25 days remain, i.e. 420 days, which Saturn has received from Jupiter. After \textbf{/370K/} we break down Saturn’s 30 months to 910 days, let us subtract three daily periods \textbf{/363P/} of 129 days each (totalling 387 days) from this amount <420 days>. There now remain 33 of the 420. Saturn gives the first 30 days to itself, the final 3 days to the next star, the Sun. 

So Jupiter is the ruler of the fifth 10-year-period; Saturn is the ruler of the months (succeeding Venus); the Sun is the ruler of the three days (succeeding Saturn), <and its rule would extend> for 16 more days. Consequently there are three distributions: yearly, monthly, daily. 

Some <astrologers> make a fourth type, an hourly distribution, by multiplying the days of each star by the 24 hours of the day-and-night period: for example, the 19 days of the Sun become 456 hours. From this amount the sun assigns:

\noindent 19 hours to itself 12 to Jupiter \\
25 to the moon 8 to Venus \\
15 to Mars 30 to Saturn \\
<20> to Mercury

In the second and third 129-hour periods, the Sun distributes in the same way to itself and to the other stars in their order <total 387>. Of the remaining hours of the 456 (i.e. 69), the Sun assigns to itself 19, then 25 to the Moon, 15 to Mars. 

Mercury is the ruler of the final <10 days> needed to complete the 69 days of the Sun. We break down the first 3 days <of this 10 day period>, the days which Mercury has received from Saturn (These are the remaining days of the 45 years, 9 months, 25 days) into 72 hours. Mercury grants the first period, 19 hours to the Sun, until the seventh hour of the night. After the Sun, it assigns 25 hours to the Moon, from the eighth hour of the night to the eighth hour of the next night. After the Moon, Mars is ruler of 15 hours, from the ninth hour of the night to the eleventh hour of the day. After Mars, Mercury receives a 13 hour period, from the twelfth hour of the day to the completion of the 72 hours, i.e. to the first hour of the <next> day.

It is necessary to examine the influences in the transmissions: Mars, when operating in the monthly and daily distributions and in the hourly sub-distributions, has the same influence which it has when transmitting to Mercury in the 10-year-period, and it is from these influences that the daily alteration of affairs can be understood. Much indeed <can be learned> from the presiding star and its successor.

After much experience in casting Initiatives, I have found this method of distribution genuine; it has erred not in the slightest with respect to persons or actions. This method, which is found in Valens as well is simple and true, and it does not introduce fractions of months and days, fractions which do not admit of an exact ratio. For it is necessary, if we should wish to subsume everything under the same ratio, \textbf{/364P/} to take 1/5, 1/30, and 1/360 of the period of each of the stars, and having done so, to transform the
fraction (1/5, 1/30, 1/360) to a fraction appropriate to the second distribution of each star in order to multiply. (Multiplication is easier than division.) 

For example: Saturn has 900 days or 30 months. One fifth of the days is 180 and of the months is 6 (which is also 180 days). The fraction that one day-period is of the other is the same as that of one month-period to the other. Similarly 1/30 of 900 days is 30 days, and 1/30 of 30 months is one month, which is also 30 days.

Since the number of the days of each star is 30 times the number of months, then clearly 1/30 of each month is one day, and so 1/30 of 30 months is 30 days, and 1/30 of 12 months is 12 days <i.e. for Jupiter>. In general the days of each star have the same number as the months when the months are multiplied by 1/30.

\textbf{/371K/} Now 1/5 has the same ratio for the days and months of each star: 

For Jupiter, 1/5 of 360 is 72 days; 1/5 of 12 months is 2 1/3 1/15 months, which is again 72 days. 

For Mars 1/5 of 450 days <=15 months> is 90 days; 1/5 of 15 months is 3 months, which is also 90 days.

For the Sun too, 1/5 of 570 days is 114 days; 1/5 of 19 months is 3 2/3 1/10 1/30 months, which is again <114 days.

For Venus 1/5 of 240 days is 48; 1/5 of 8 months in 1 1/3 1/5 1/15 months, which is also> 48 days.

Likewise for Mercury 1/5 of 600 days (or 20 months) is 120 days or 4 months; <1/5 of 20 months> is 4 months, again equalling 120 days.

Likewise for the Moon 1/5 of 750 days (25 months) is 150 days (or 5 months).

Since then 1/5 of the days results in the same figure as 1/5 of the months, and 1/5 of the number of months implies 1/5 of each month, which is 6 days, and since the 6 days are 6 times one day, it is clear that when the number of months of each star are multiplied by 6, the same number results as when the number of days are multiplied by 1/5.

So as to understand this more clearly: if we multiply the months of each star by 6, we will get the days of each star. For example:

\noindent For Saturn, 6 times 30 <months> gives the same 180 days again.\\
For Jupiter 6 times 12 <months> gives 72 <days>.\\
For Mars 6 times 15 gives 90.\\
For the Sun 6 times 19 gives 114.\\
For Venus 6 times 8 gives 48. \\
For Mercury 6 times 20 gives 120.\\
For the Moon, 6 times 25 gives 150.

Besides this, from 30, without any multiplying, we find the days which have the same number as the months of each star: 30 days are 1/30 of Saturn’s 30 months; 12 days are 1/30 of Jupiter’s 12 months; and so on. \textbf{/365P/} But taking 1/360 of a number is difficult:

\noindent For Saturn 1/360 of 900 is 2 1/2. \\
For Jupiter 1/360 of 360 is 1. \\
For Mars 1/360 of 450 is 1 1/4. \\
For the sun 1/360 of 570 is 1 1/2 1/12. \\
For Venus 1/360 of 240 is 1/2 1/6. \\
For Mercury 1/360 of 600 is 1 1/2 1/6. \\
For the Moon 1/360 of 750 is 2 1/12. \\

Therefore we will transform this fraction, 1/360, to something smaller and easier. Since the number of months is 1/30 of the number of days in those months, the following ratios obtain: 1 to 30, 2 to 60, 3 to 90 and so on on, the ratio of the months to the days. 

Correspondingly, the number of days is 30 times the number of months, so that if I select another fraction for the months instead of 1/30, 30 times the selected fraction will be the fraction of the days. Since the 360 of the days is 30 times the 12 of the months (i.e. 30 times 12=360), if we use 1/12 for each star’s months, this will be the same as 1/360 of their days:

So for Saturn 1/12 of the 30 months is 21/2, which is the same as 1/360 of the 900 days.

For Jupiter 1/12 of its months is 1, which is the same as 1/360 of its days.

For Mars 1/12 of 15 is 11/4, which is the same as 1/360 of its 450 days.

And so on for the rest. Consequently we can use 1/12 of the months instead of 1/360 of the days. 

We can transform this into the 24 hours of the day-and-night period by multiplying:

For Saturn 21/2 times <24> gives 60 hours, which is twice 30, the same number as the months and the days.

For Jupiter 1 times 24 hours <gives 24>, which is twice 12, the number of Jupiter’s days and months.

For Mars 11/4 <times 24> gives 30, obviously twice 15, the number of Mars’ days and months. \textbf{/372K/} 

For the Sun 1 1/2 1/12 gives 38, which is twice 19, the same number as the sun’s days and months.

For Venus 2/3 gives 16 hours, twice 8, the same number as Venus’ 8 days and months.

For Mercury 1 1/2 1/6 gives 40 hours, twice 20, the same number as Mercury’s 20 days and 20 months.

For the Moon 2 1/12 days gives 50 hours, twice 25, which are the moon’s days and months.

When for each star the number of months are changed to the number of days, \textbf{/366P/} 1/5 times 1/6 <times the number of days> shows the number of months; 1/30 is also equal to the number of months. It is therefore obvious that 1/5 <of the days> plus 1/30 <of the days> will be 7 times the same number <of
days>. 

For example: for Saturn, 1/5 of 900 is 180, which is 6 times 30, but 1/30 <of 900> is also equal to 30. 210 is 7 times 30, the number of Saturn’s months. For Jupiter, 72 <1/5 of 360> plus 12 <=1/30
of 360> is 84, which is 7 times 12. And so on for the rest.

Since each star is ruler not only of its own recurrent years, but of a number of days equal to seven times the number of months, it is clear that, when it must make a distribution to the seven stars, it gives to each star the same amount it gives to itself, and that there will be no remainder except 1/60<?> which is itself shown to be true since it is double the proper number of the months of each star.

Assume we are to find, using this procedure, which stars control a given time. Let the age be 18 years, 4 months, 13 days. As in the example above, let the Moon control the first 10-year-period of the
220 month total. There are 91 months left for the second 10-year-period. Let Mars be located following the Moon as the ruler of the second 10-year-period. Since this period is not complete, Mars will give itself 15 months; Mercury, the next star, gets 20 months; Jupiter next gets 12; Venus next gets 8; Saturn next gets 30. The Sun receives the remaining 6 months and 13 days, which total 193 days. 

Since the Sun does not have its complete number of days (because of the given time), it is necessary to break down the partial period, the 193 days. So in this period the Sun gives itself (using this procedure) 163 <?should be 133?> days. Next the Moon, since it cannot receive its complete period of days (175), takes 30 as incomplete and breaks it down to 720 hours (=360 doubled). In this period, it gives itself 25, Mars 15, Mercury 20, Jupiter 12, Venus 8, Saturn 30, the sun 19. 

Next after this 129 (doubled) hours, the Moon distributes, beginning with itself, another 129 (doubled), and the remaining 102 <to make 360>. 

Again beginning with itself it makes the distribution in order, and 22 are left, which Saturn rules, since it rules 30 (doubled). So Mars is the ruler of the second 10-year-period, the sun is the ruler of the months, the Moon is the ruler of the days, and Saturn is the ruler of 44 hours—because you must restore the 22 (doubled) to the correct number of hours. The distribution will be of months, the subdistribution will be of days, and the sub-subdistribution will be of hours.

The presiding star and the following star will clearly indicate the changes in each day. It is necessary to examine the rulers of the 10-year-period, of the months, and of the days, to see the nature of their transits and configurations. When they are beheld by benefic places and stars, they indicate that the <period> is also benefic; when beheld by malefics, it is malefic.

END



\newpage