\section{The Lunar Degree-Position with Reference to its Hourly Motion; A Compelling Method Which I Have Discovered}

I have not wished to hide any of the methods which I have previously worked out, and now I am generously bestowing another scholarly gift on you devotees of such matters, as if you were my children. The tables described below are on two rectangular sheets. As before <end of Book VIII>, they are arranged in equal intervals across the page. The first sheet is the daily motion of the Moon, from the minimum
factor of 11 <hours>, going from one square to another in the first row. The same series of intervals is in the bottom row, extending to the 15 hours/squares of the maximum factor. From the top to the bottom are 48 rows. So there is an increase of 5 minutes <per row> from the 11 hours <of the first row> to row 48 <15 hours>, i.e. 4 full hours, the excess <of 15 over 11>. When \textbf{/347P/} 60 minutes are completed, there is a red mark to indicate the beginning of another cycle. In this way the motion of the moon is charted.

The second rectangular table shows the length of the hours for each of the 7 klimata. This one has the numbers for <every> klima arranged \textbf{/362K/} from least to greatest, i.e. from 10;30 entered in the first row to 19;30 entered in the outside <?> row. I write each entry in order from 0;30 in the first row to the blank/zero in the bottom row. Therefore, the increase is 5 minutes <per row>. I find the bottom row ending with 15, which indicates one equinoctial hour. And so, continuing from 15 at the bottom, add 5 minutes going up the table to end with 19;30 in the first row.

The horizontal lines separate the two-hour periods, and each pair of hours equals an equinoctial hemisphere <30>. (For night births, calculate using the point opposite <the Moon>.) A double line
separates these. The ratio that night length has to day length is the same as that between the lengths of the night-hour and the day-hour: for example 17 to 13. 12 times 17 is 204; 12 times 13 is 156. The total is 360°. Both tables show the degree-positions of the Moon and its phases, so if we want to know the Moon’s degree-positions at a nativity with reference to its hourly motion, this is how we operate. 

First we must enter the table of klimata, holding the compass with legs apart. At the Sun’s degree-position in the night hemisphere<?> and after we have determined the length of the hours, we place one leg of the compass right there. Then we open the compass until the other leg reaches the hour in question <of the night>. The 12 hours of the night are tabulated <to allow this>. If the nativity was during the day, note the extension of the compass legs in the night hemisphere and extend this distance to the hour in question of the day.

Now having measured out the total number of hours in the way described, move the compass to the lunar table. Set one leg of the compass at the figure equal to the motion, then see what degree-position the other leg touches. The degrees will be obvious from the chart of its motion, and these must be added (if the nativity is after sunset) to the degrees previously determined for the Moon; add the difference due to klima as well. Having added, consider this to be the Moon’s degree-position. It is necessary to know the hemispheres accurately, particularly \textbf{/348P/} the night hemisphere. It is also possible to measure off the degrees remaining until sunset in the day hemisphere and to apply them to the Moon’s hemisphere. Treat it as having that distance and being at that degree position. \textbf{/363K/} But if we proceed in such a manner, the whole width of the lunar table will not be used, only a part. Therefore use the night procedure.

We will give a clearer explanation by using examples: a nativity on Hadrian year 3, Athyr 30, the fourth hour of the day; the \Sun\xspace in \Scorpio\xspace 7°, the \Moon\xspace in Virgo\xspace 30° late in the day. Its motion was 14;15. The length of the day (using day-degrees) was found to be 11 hours 42 minutes. Entering the column of klimata at 11, I find 11;42, and I use this row. Since this is the day hemisphere, I go to the opposite point (i.e. in the night hemisphere) at 18;18 so that 18;18 plus 11;42 might total 30. Now I place one leg of the compass at 18 and I extend it over the other hours of the day, along the line in the row for 11;45. Keeping the compass at this extension, I move to the previously mentioned table of the Moon’s motion. For the nativity at hand, it has run 14;15. I enter the table at that point, place one leg on the compass there, and see which square in the whole chart the other leg touches. I find it in the tenth square, around the fourth part. Since each square indicates one degree, I add this amount to what was determined <to be the position> at sundown, \Virgo\xspace 30°. The Moon is found to be approximately at \Libra\xspace 11°. Alternatively, I enter the day hemisphere at 11;42, the <hourly> magnitude of the Sun’s degree-position, and I extend the compass to the 8 hours still remaining until sunset (for this nativity), and the same magnitude <results>…

\newpage
