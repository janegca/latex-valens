\section{Critical Times (4K, 5P)}

\textbf{/325P/} As is proper, the contacts and unions of the stars, the Sun, and the Moon become very active for good and for bad. It is also necessary to examine the periodic chronocratorships of each star to see if they come into effect in the chronocratorships of benefics or malefics, and what numbers they consist of. 

For example: the period of Saturn is 30 years. I am investigating at what other numbers this 30-year period is operative. I proceed as follows: I begin with 4 and I factor in the next numbers in order: 4 and 5 make 9; then 6 and 7, for a total of 22; next 8, for a total of 30. So the 30-year period is completed by starting with 4. There will be a Saturnian critical point every 4 years, then every 30, its own period.

Jupiter acts as a benefic and brings rank every 3 years: 3 plus 4 plus 5 total 12.

Mars every 4: 4 plus 5 plus 6 total 15.

Venus is found to be unassociated; it will act every 8 years.

Mercury acts every 2 years: 2 plus 3 plus 4 plus 5 plus 6 total 20.

The Moon fills three-year-periods: 3 plus 4 plus 5 plus 6 plus 7 total 25.

The Sun fills 9-year-periods: 9 plus 10 total 19. The sun also \textbf{/339K/} controls 20-year-periods: this period comprises the unit and the 19 year period, since the Sun travels 1° in a day-and-night period, i.e. in 24 hours it traverses 4 phases, the first from sunrise to noon, the second from noon to sunset, the third from sunset to midnight, the fourth from midnight to sunrise. If we add the degrees of the phases (first=6 hours, second=12 hours, third=18 hours, fourth=24 hours), the total is 60. In addition, since the Sun allots 120 years as its maximum period (one-half of which is 60), the semicircle <of the Sun> becomes 60. 

The <astrologer> must make the type of forecast which is operative in accord with the positions of the stars and the configuration and nature of the signs.

\newpage