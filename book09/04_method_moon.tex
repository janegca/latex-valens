\section{A Method Concerning Propitious and Impropitious Times and Lifespans with Respect to the Moon (3K,4P)}

\textbf{/337K/} Through experience I have discovered another aphetic method which uses the zones of the stars.

Zoroaster spoke of this method in riddles. Beginning with the Moon we count upwards, giving the following to each star:

\begin{tabular}{ll}
9 to the \Moon & 9 to \Mars \\
9 to \Mercury & 9 to \Jupiter \\
9 to \Venus & 9 to \Saturn \\
9 to the \Sun & \\
\end{tabular}

Now we count in the order of signs until 108 years are completed, the maximum period of the Moon. He proposed this universally as a model, as have the King and many others. 

Everyone uses this as described, but I propose to make the following allotment: count off the number of completed years, beginning with the \Moon’s sign and subtracting first the amount which the Moon controls. Allot the remainder until 9 years are completed. Then give 9 to each sign, proceeding in the order of signs. 

If the <last> sign does not receive a full 9, from that point give the months to each sign until you arrive at the time in question, noting first the sign which was assigned the 9-year-period during which the starting point of the 9-month-period began. At the point where the count stops, \textbf{/324P/} determine (using the previously described system) whether it ends at an active or at a weak spot. 

If the 12 aphetic points have the same influence, they will bring the maximum period to the life-aspects, and they will indicate vigor in the activity-aspects. Allotting in this way 9 years to each of the 12 signs, we will complete 108 years. 

If, having allotted periods to certain signs, we make a second assignment to them, we will find an excess and a deficiency with respect to the stars.

In addition, it is impossible and unseemly for the many nativities that occur at the same time to have one and the same chronocrator. Therefore, the zodiacal aphetic point for each nativity, viz. the Moon,
which changes its position in each nativity, causes an extraordinary variation. 

As I have said before and must now make clear, it is necessary to examine the star of Jupiter to see if it is in aspect from the right with the Ascendant (to the degree), i.e. whether it casts its rays within its (the Ascendant’s) degrees or beyond them. 

If Jupiter is ahead <of the Ascendant> and is found to have a retrograde configuration, its beneficial effect will be strong because it is being carried \textbf{/338K/} towards the position of the Ascendant. 

If it is behind <the Ascendant, i.e. to the right>, it will be naturally better. 

If it is turned away from the Ascendant, it is bad. 

In so far as it beholds any aphetic place or the place of a star or sign at the change of the chronocratorship in question, it must be considered a benefic. 

Whenever it leaves a sign or degrees in either a direct or a retrograde phase, it becomes malefic and harmful.

It is also necessary to observe the Moon’s relationship with the degree-position of the Sun and the angles—i.e whether it is square, trine, or in opposition. Not only that, but also with the intervals of 15° or half the sign's rising times, for then it seems to make a motion/phase. Particularly when it passes through the two nodes in any chronocratorship, the doom will be certain: the determination of the fatal cycle will be made from the lunar and solar degree-positions, just as we have explained using both <the nodes> and the procedure of cycles<?>, or using some other powerful procedure which takes the nodes into account…


\newpage