\section{A Method for the Length of Life, with Reference to the Sun and the Moon}
\index{length of life}
Although I have already engaged in many contests and have been involved in struggles without number since I was led by my zeal and my ambition to face my rivals, \textbf{/345P/} and although I had
anticipated the attainment of my investigations and was already putting an end to this treatise, just like a noble athlete in the sacred games, the Olympic contest, still I have reversed my intentions because of the
multifaceted operations of nature. 

\textbf{/360K/} The one who brings or lifts these operations into the light of day gains eternal honor and fame. He turns his previous struggles to pleasure and delight. He strengthens his powers and certifies his deeds. He dismisses vain chatter and turns his enemies into friends and associates, however unwilling they may have been. He takes to his bosom many devotees.

Therefore we must return to the cosmocrators, the \Sun\, and the \Moon: it is necessary to calculate the rising time and the hourly magnitude of their degree-positions according to the klima of the nativity. Then subtract 360° circles. Treat the remainder as a powerful degree position derived from the equatorial times, and which is the “complement” of the Ascendant. Four degree positions in all are operative, and two of the four remain as potent. Occasionally five places are operative, or perhaps two.

\index{Ascendant!degree}
It is necessary to examine the sign in the Ascendant and the signs on both sides. It is from these that the operative degrees allow the chronocrators to be calculated, particularly when there is some error in
placing the Ascendant or if one gnomon is used in the Ascending sign, while the other gnomon does not fall in the same sign<?>…The sixteenth degree equals 11/30…

It is necessary to take into account the variations due to the klimata, because often a place or a nation which seems to be at the beginning or the end of a “parallel” is really in another area or is divided between
two parallels and the time is in error by some addition/subtraction factor. 

Additionally, as is obvious to those devoted to such matters, they have different astrological attributions. As you can see, this art and its methods are sacred and infallible for those who attend to the details.

\newpage