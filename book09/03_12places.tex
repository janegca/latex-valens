\section{The XII Places and Their Relationship to Propitious and Impropitious Times (2K, 3P)}

\index{Egyptians}
These matters were thus arranged according to their cosmic harmony in ancient times. The Egyptians, although they had received them in simple form from antiquity, locked them up with complex and interwoven distinctions, and they used sophistic talk and approaches. 
Having walled in this art with a myriad of bulwarks and with “bars of unbronzed bolts,” they then departed. \textbf{/321P/} As a result, those who enter these precincts are like blind men: they wander at random because no gates have been placed there or because they do not chance upon the location of these gates. They have the misfortune of making the same discovery a thousand times. I have knocked down a section of this barrier-gate and have shown the entrance, like a gate, to those who wish it. 

Now \mn{Places \& Derivatives}I return my thoughts to the subject at hand; let this discussion concern the XII Places. \index{Asclepius@\textit{Asclepius}}Asclepius, beginning with this topic, composed the most; then many Egyptians and \index{Chaldeans}Chaldeans did likewise\footnote{The same is true for the Eight-Place system.-marginal note [Riley]}.
The Places starting from the Ascendant are as follows:
\index{places!derived}
\begin{description}[labelindent=0em , labelwidth=1em, labelsep=1em, leftmargin =!]
\item[I.] \index{places!01@1st}
	Life, the basis of years, the psychic spirit—i.e. the Ascendant itself.
	
	Relative to the III Place of Brothers this is the Good Daimon and the Place of Children and Friends. 
	
	Relative to the IV Place of Parents it is the Place of Action. 
	
	Relative to the VII Place of Women it is the Marriage-bringer. 		
	
	Relative to the V Place of Children it is the IX Place.
	
\item[II.]\index{places!02@2nd}
	 Livelihood, income from property. 
	 
	 Relative to the III Place of Brothers it is the Bad Daimon and the Place of Slaves and Enemies and of afflicting crises. 
	 
	 Relative to the IV Place of Parents it is the Good Daimon and the Place of Friends. 
	 
	 Relative to the V Place of Children it concerns action and rank. 
	 
	 Relative to the VII Place of Women it is the Place of Death. 
	 
	 If the ruler of the new or full moon is found in this Place or in the Place in opposition, it indicates exile. The new or full moon is observed for similar indications.

\item[III.]\index{places!03@3rd}
	Concerning \textbf{/335K/} the life of brothers. 
	
	Relative to the IV Place of Parents, it concerns enemies and slaves. 
	
	Relative to the VII Place of Women it is the IX Place [concerning rank, occupation, and childbearing]. It is also the Place of the Goddess, of the Queen, [and of occupation].

\item[IV.] \index{places!04@4th}
	The Place concerning the life of parents, concerning religious and secret matters, estates, property, and treasure-troves. 
	
	Relative to the III Place of Brothers it concerns livelihood. 
	
	Relative to the VII Place of Women it concerns rank and occupation. 

\item[V.] \index{places!05@5th}
	The Place concerning the life of children; the Good Fortune. 
	
	Relative to the III Place of Brothers it is the Place of bastard- and step-brothers, of the Goddess and the Queen. 
	
	Relative to the VII Place of Women it is the Good Daimon.
	
\item[VI.] \index{places!06@6th}
	Concerning injuries, illness, and afflicting crises. 
	
	Relative to the IV Place of Parents it concerns brothers. 
	
	Relative to the III Place of Brothers it concerns step- and suppositious parents. 
	
	Relative to the VII Place of Women it concerns enemies and slaves.

\item[VII.] \index{places!07@7th}
	 The Marriage-bringer of a nativity; concerning the life of women. 
	 
	 Relative to the III Place of Brothers it concerns children and is the Place of Good Fortune. 
	 
	 Relative to the IV Place of Parents it concerns parents, estates, property, treasure-troves, and religious matters.
	 
\item[VIII.] \index{places!08@8th}
	 Likewise for the nativity this Place concerns death. 
	 
	 Relative to the III Place of Brothers it concerns injuries and diseases. 
	 
	 Relative to the IV Place of Parents it concerns bastard children. 
	 
	 Relative to the VII Place of Women it concerns livelihood.
	 
\item[IX.] \index{places!09@9th}
	 Concerning Foreign Lands, the God, the King, \textbf{/322P/} prophecy, and money matters. 
	 
	 Relative to the III Place of Brothers it is the Marriage-bringer. 
	 
	 Relative to the IV Place of Parents it concerns injuries, diseases, and afflicting crises. 
	 
	 Relative to the VII Place of Women it concerns brothers. 
	 
	 Relative to the II Place of Livelihood it concerns death.
	 
\item[X.] \index{places!10@10th}
	Concerning occupation and rank. 
	
	Relative to the VII Place of Women it concerns estates, property, religious undertakings, and the Place of Parents.
	
\item[XI.] \index{places!11@11th}
	The Place of the Good Daimon, the Place concerning friends and desires and acquisition. 
	
	Relative to the III Place of Brothers it concerns the God, the King, prophecy, and money matters. 
	
	Relative to the IV Place of Parents it concerns death. 
	
	Relative to the V Place of Children it is the Marriage-bringer. 
	
	Relative to the VII Place of Women it concerns step-children.
	
\item[XII.] \index{places!12@12th}
	 Concerning enemies, slaves, and afflicting crises. 
	 
	 Relative to the III Place of Brothers it concerns occupation and rank. 
	 
	 Relative to the IV Place of Parents it concerns travel, the God, the King. 
	 
	 Relative to the V Place of Children it concerns death. 
	 
	 Relative to the VII Place of Women it concerns injuries and disease.
\end{description}


The precise distinctions between the things indicated by the Places are explained elsewhere. 

\subsection{\textit{[Delineation]}}\index{nativity!delineation}
After charting these Places in the order of the zodiac for interpretation, it will be necessary to examine which stars, whether benefic or malefic, are in the Places or are in aspect; which stars’ signs they coincide with; and whether these signs are tropic, solid, bicorporeal, moist, dry, lewd, thievish, etc. 

\index{distribution!planet order}
Likewise determine the rulers of the Places, i.e. which ruler of which sign is in which Place. In the proper determination of chronocrators, determine from which Place to which Place the chronocratorship is passing, and count off the years of each star from \textbf{/336K/} each Place\footnote{Handing off from Place to Place ordered by domicile rulers?}. The XII Places, when compared in circular order with each other in this way, will make the results and the type of result obvious\footnote{Hand offs will vary if the place ruler hand offs define the order with the type of places involved determining if the result is good or bad.}.

\index{places!equal}
First \mn{\tiny Equal Houses} of all, it is necessary to calculate the positions of the Places in degrees: count from whatever point has been determined to be the Ascendant until you have completed the 30° of the first Place; this will be the Place of Life. Then proceed until you have completed another 30°, the Place of Livelihood. Continue in the order of signs. Often two Places will fall in one sign and will indicate both qualities according to
the number of degrees each one occupies. Likewise examine in which sign the ruler of the sign is and which Place it controls (according to its degree-position in the horoscope). With these procedures, the
Place can readily be interpreted. 

\index{places!whole sign}
If it is calculated that each Place exactly corresponds to each sign in the chart as a whole (a circumstance which is rare), then the native will be involved in confinement, violence, and entangling affairs. If the star of \Mercury\, is associated with these chronocrators (i.e. with the sign of the \Sun\, or with the signs belonging to the star of \Mars), then this circumstance indicates that the attack or the confinement
occurs because of documents. And so on.

\index{transits} \index{distribution!periodic}
Be aware of the transits of the stars and their changes of sign at the various chronocratorships, as I have described. It is necessary to calculate \textbf{/323P/} as follows: add a number of days to the birth date equivalent to the age (in years) of the native. Then, having first determined the date, whether in the following month or in the birth month itself, cast a horoscope for that day\footnote{Essentially modern Secondary Progressions where one day of life equals one day in the ephemeris.}. <See> which star, if any, is in the Ascendant or is coming into conjunction with another star, and whether it is moving from an angle to a point following or preceding an angle, or from a point <following or> preceding an angle to an angle, or whether it was rising at the date of the delivery but is now setting or coming to some unrelated phase, or to something better. You may consider these to be the periodic forecasts.

\index{transits}
The following procedure seems valid to me: we add the age in years to the birth date and calculate in which month the new date falls. Then chart the <transits> of the stars of the current year and make the forecast as described\footnote{Compute the Secondary Progressed chart for the year in question and compare real-time transits to that chart?}. As for the previously explained <previous paragraph> method for the stars: we will not find much change in position for \Saturn, \Jupiter, and \Mars. These stars have an imperceptible motion and stay in the same place. In the latter method <this paragraph> we will find that they come to be in square, trine, and in opposition.

\newpage