\section{Preface}

Valens sends greetings to Marcus. The information set out by the divine king Nechepso in the beginning of his XIII Book has been exhaustively treated in our previous compilations and in the labors of others. Now I will compile the following material, falling short in no respect. 

It is obvious that the King made his explanations with mystic intelligence and that he has also been the guide—even for us—in our
approach to this art. His willingness to confess, and then to correct, his early errors is a sign of a nobility and wisdom on his part which gave him the intelligence to know when to change his mind. The fact that
he despised his kingship and power and devoted himself to these matters <astrology> is a sign of his experience and persuasiveness, qualities which reveal this art’s alluring and attractive face to his successors. No necessity for making a living and no trickery caused by greed affected him—as these traits have affected so many now-a-days. As a result this man must be taken as a model.

The very wise Critodemus, in the vital work attributed to him, the \textit{Horasis}, made such a beginning of great mystery, to wit: “Already having traversed the seas and having crossed great deserts, I was thought worthy by the gods to reach a safe harbor and a secure resting place.” Timaeus, Asclation, and many others have said the same. These men were carried away by the beauty of words and by reports of marvels, and they did not produce works which fulfilled their promise, nor were these works complete and lucid, but \textbf{/330K/} rather they left their readers in the lurch many times and at all times were warped, begrudging, withdrawn, and deceptive. They never travelled one road, but they piled scheme on scheme and wrote
books which could be prosecuted because they are proofs \textbf{/317P/} of fraud, not of truth. This Critodemus, although he had inherited a mass of theorems, had developed others himself, and was able to interpret clearly, still obscured the truth because of the appearance of his tables. I on the other hand in my previously compiled books, have composed an oeuvre which does not consist of vain and empty babble, nor have I included questionable solutions using someone’s mere opinion or purely qualitative non-numerical writings. 
Approaching what seemed to be the truth, he <Critodemus> wandered off into endless inquiry and criticism. 

One who wishes to write treatises must <proceed> as if wishing nothing else; if he does <have ulterior motives>, he will import error into his work because of his ignorance and spite. Therefore having traversed the sea and having crossed many lands, I have surveyed many climes and nations, have been plunged in long toil and trouble, finally to be thought worthy by God of attaining secure foreknowledge and a safe harbor. Not everything that men attain is corruptible and burdensome; there is in us some divine, divinely crafted element. The circumambient air, that incorruptible and all-pervading substance, imbues us with this momentary influx of immortality at appointed and fixed times. Each of us in our daily activity strives to receive or give forth this lifegiving spirit. As the divine Orpheus says:

“Man’s soul takes root in the aether.”

and

“When we draw in the air, we harvest the divine soul.”

and

“The immortal and unaging soul comes from Zeus.”

and

“Of all things, the soul is immortal, the body mortal.”

Therefore in so far as we posses soul, we move, we associate, we perform, we contrive, we do actions fit for the gods. When our debt <to Fate> soars into the air, our body will lie dead and silent, having given up its spirit in succession to another, an empty \textbf{/331K/} artifact of Destiny, \textbf{/318P/} perceiving nothing. Its nature then being dissolved, the mortal frame is then examined in its own place.

Now with the help of God, I have discovered these matters which have been treasured up in darkness. For my part, my plan—generous from the start—has been to preserve my exposition as secret and hidden,
because of the multitude of the unworthy. But so as not to seem to be an accuser and to fall into greater criticism and to excite accusations from others, I have decided to mystically set forth in this book the
chapters necessary for completing the previously outlined topics—not in an arcane and obscure way, but with direct clarity. I can count on the great intelligence of my listeners. I do this so that my heretofore
ignorant listeners and those who fight against the Gods may gain faith (with the help of these <chapters>), may become friends of the truth, and may receive this pre-existent and revered science.
\newpage