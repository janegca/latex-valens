\section{A Method for the Length of Life with Reference to the Sun and the Moon.}

This method seems scientifically appropriate: combine the remaining degrees of the Sun and Moon, subtract the distance between them in signs, and treat the remaining degree position as powerful and as the
complement of the Ascendant. 

\textbf{/346P/} Consider the remaining <number> in the 30 as the lunar gnomon, using an addition/subtraction factor derived from the equinoctial times. Add the lunar gnomon to the solar, divide by two, and treat the answer as the length of life. 

Moreover, often the solar gnomon \textbf{/361K/} by itself (or the lunar), when the addition/subtraction factor derived from the equinoctial times is used…one-half of the sum on the two [places] indicates the length of life, depending on its temporary or operative states<?>.

So it is best to calculate the intervals of degrees and signs using rising times. Moreover, if the gnomon is operative by itself, calculate the times of it alone.

\newpage