\section{A Method Concerning the Degree in the Ascendant (10K, 11P)}

\index{Ascendant!degree}
Since many scholars take delight in all sorts of <astrological> systems, I will append yet another method which has been transmitted by some of my predecessors in a riddling manner. I do this so that scholars can become familiar (through our help) with the systems held in honor by others, can gather together the potentialities of these systems, and can award us eternal fame. It is odious and disagreeable to test others’ opinions, especially those which have not been received through written books or compelling dialogues—as \index{Petosiris@\textit{Petosiris}}Petosiris and \index{The King}the King mystically published books on many subjects. For the compiler knows the beginning and the potential; he then makes the end agree. He intentionally publishes many systems for the initiates and for the ignorant, systems whose power will be easily grasped by attentive
students. Some of these have been written out privately, \textbf{/352K/} others secretly, and they are despised by their readers since these readers fail to recognize their power. 

It is as if a man trod on a piece of ground which held a treasure: he does not see what is under his feet, but walks on blindly because of \textbf{/338P/} his ignorance. If, however, someone were to inform him of the treasure, he would excavate, would find it, and would feel an extraordinary delight.

In every case it is necessary to take the distance from the \Sun’s degree-position to the \Moon’s in the order of the signs using the rising times<?>\footnote{Take the degrees in the ascensions of the birth place.}, and to mark the resulting number of degrees as the solar
gnomon. 

Next consult the table of rising times under the klima of the nativity and see what fraction is entered at the \Sun’s degree-position. (This is for day births; for night births, look at the point in opposition to the \Sun.) Multiply this by 12; then multiply the result by the hour of the day at the delivery. If the result exceeds 360°, subtract 360° and see if the remainder corresponds to the previously determined gnomon. If it does, the hour which was reported will be accurate and should be used. If,
however, the remainder greatly exceeds the solar gnomon, subtract from the <reported> Ascendant an amount equal to the excess. (Do this by figuring what fraction the excess is of the <hourly> magnitude,
and subtract that.) 

If the solar gnomon is greater, add to the <reported> Ascendant an amount corresponding to the excess. Then determine the fraction <of the hour> and enter the table of rising times. 

Continue by calculating the full hours and the fraction. Add the years and note at what degree of the zodiac the \textit{enklima} falls. Consider that point to be the real Ascendant. \index{Thrasyllus@\textit{Thrasyllus}}Thrasyllus used this method, made a scientific beginning, and fashioned a forecast of the end.

\newpage