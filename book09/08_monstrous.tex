\section{Male and Female Nativities; Monstrous or Animal-like Nativities (7K, 8P)}

For any nativity it is necessary to see where the dodekatemorion of the \Moon\, is located. If it and its ruler are in female signs, forecast a female nativity as a rule. If the nativity chances to be male, calculate
some addition and subtraction factor for the Ascendant such that the male dodekatemorion can be in the Ascendant when calculated using the rising times. Examine the point in opposition to the dodekatemorion as well: if the dodekatemorion, the point in opposition, or their rulers, fall in a theriomorphic sign\footnote{A sign having the form of a beast.}, predict a monster or an inviable creature.

For example: the \Moon\, in \Pisces\, 19°. The dodekatemorion is in \Libra, a male, man-like sign, and the nativity was male. The ruler <of \Libra>, \Venus, was in \Sagittarius. The Ascendant requires neither addition nor subtraction since it fell in a man-like sign. I take then the rulers of \Libra\, and \Aries\footnote{The dodekatemorion of \Pisces\, 19° is \Aries, the dodekatemorion of the point opposite \Pisces\, 19° is \Virgo\, 19° whose dodekatemorion is in \Libra.}, \Venus\, and \Mars. \Mars\, was in \Virgo, \Venus\, in \Sagittarius. After <re->calculating the Ascendant from the table of rising times using the reported Ascendant, I find it to be in \Leo\, 15°. I start counting the dodekatemorion from this point to \Mars\, \textbf{/328P/} and \Venus, and I will consider the one nearest the degree in the Ascendant (i.e. \Leo\, 15°) to be the ruler of that monomoirion: from \Leo\, 15° to \Mars\, in \Virgo\, is a distance of 50 <?>. From 15°, I count two additional dodekatemoria to \Scorpio; the result is 56. \Venus\, is in \Sagittarius, and
if we give a part of the dodekatemorion to \Sagittarius, the result will be 12. Therefore this degree-position will be closer to \Leo\, 15°. Since \Venus\, has the closest dodekatemorion, it is necessary to investigate the monomoirion of \Venus\, at \Leo\, 11°, 12°, and 13°, and to consider that to be the Ascendant.<?paragraph?>\footnote{Can't make head nor tails of the numbers 50 and 56.  There are 7 dodekatemoria from \Leo\, 15°, to a \Virgo\, dodekatemoria in the beginning of \Virgo\, and counting two from that does give a \Scorpio\, dodekatemoria. Each dodekatemoria is 2.5° which gives 7 x 2.5 = 17.5, and 9 x 2.5 = 22.5 so numbers don't appear to be degree based. Looking at rising times, \Leo\, rises in 35°, \Virgo\, in 38°, neither of which correlate to 50 or 56 as far a I can see.}

Another procedure: the \Sun’s position always indicates the length of day and night, depending on which sign it is in. The Ascendant is determined from the hour, the degree is determined from the
Ascendant, \textbf{/342K/} and the precise degree of the Ascendant is determined from this degree, since they arise, are governed, and are corroborated by each other—particularly for day births. (For night births, it is necessary to take the remaining degrees of the \Sun\, plus the degrees included in order of those attributed to
the signs<?>. If some inaccuracy is suspected in the Ascendant’s position, it is necessary to enter the table of rising times using to the data of the horoscope and to examine the hourly magnitude to see what numerical ratio it has. Then add this to the total rising time, and subtract an amount proportional to the apparent error. The resulting position is to be considered the Ascendant. This exact point will be the
standard, because of the differing inclinations of each nativity.

Next the relationship of the \textit{horimaia}\footnote{The \textsl{horimaia} are planetary hours, the hourly times of a planet with the day hours equaling 1/12th of the planets diurnal arc, the night hours, 1/12th of the planet's nocturnal arc (AATPG p.1135).} is mystically made clear: not every nativity has the same eastern <=aphetic> point, nor an equal length of life. Some are extended through a great space, others through two or three signs, others through not <even one> entire sign. If anyone wishes to foreknow the point of the Ascendant, he can discover it from the degree determined by the methods set forth in the previous book <VIII>. 

Now let us add some clearer notes to this topic. Determine the number of degrees from the <preceding> new moon to the \Moon’s position at the birth. Now double the number. The first number, counted upwards from the new moon, shows the eastern point
<=Ascendant>. The second number, counted in the order of signs from the new moon shows the western point <=Descendant>. Now it is necessary take the distance from the eastern point to the western, and to see what fraction of 360° the distance is. The nativity will survive this fraction of time with respect to the maximally allotted time. In a like manner, \textbf{/329P/} count off the number of months from the degree-position of the Ascendant, in the order of signs and in the opposite direction with respect to the eastern and western hemispheres. 

Another procedure: each of the Lots, by itself, shows the degree-position of the Ascendant\footnote{This is a redundant method, you can only find the Lots if you know the Ascendant degree to begin with.}. For example, if the Lot of Fortune is located within a sign and if the degree-position of the \Moon\, is known, count the degrees from the \Moon\, to the \Sun, then count the resulting number upwards starting from the Lot. The Ascendant will be where the count stops. Daimon will be similarly useful for day and night births: when compared <with the Lot of Fortune it gives the result> by sign; when used with the Sun, <it gives results> to the degree. For day births, count from the \Sun\, to the \Moon\, \textbf{/343K/} and take an equal number of degrees upwards from Daimon. For night births, count from the \Moon\, to the \Sun\, and take an equal number in the order of signs from Daimon. (Or from the \Sun\, to the \Moon\, and the same number upwards from Daimon—both procedures will have the result coming at the same point.) This degree position will be considered operative.

The determination of the <Ascendant> in question using the method at hand will be made from the tables of rising times and of the inclination \textit{[of the ecliptic]}: by adding to or subtracting from the total rising time the amount by which the length of the hour varies, one can calculate the difference in degrees. Then after adding or subtracting that distance to/from the previously determined operative degree, one must consider the new degree-position to be the Ascendant. Therefore the layout of the XII Houses, which are arranged differently depending on the inclination of the ecliptic in different <geographical areas>, cause an extraordinary variation <in fortune>. Those born in Rome will not have the same lifespan as those born in Babylon, and vice-versa. Sometimes a very small variation is found, sometimes a very great one, sometimes an
enormous one. If the difference of a fraction of an hour or of a day has such an effect, why not the same for one klima with respect to another—because of the <different> shadow lengths on the <different parts of the> earth, as well as the ascents and stations of the \Sun\, relative to the ecliptic? But these matters are difficult of
approach to the multitude and are considered madness; to the wise, however, the results of the forecasts are proof of what I have said.

Another procedure: add the degree-positions of the \Sun\, and the \Moon. Treat this figure as the operative gnomon and the vital degree, one active according to the determination made by the directing <sign?>. Then begin with the first degree of the sign that appears to be in the Ascendant, and position the two Lots \textbf{/330P/} in such a way that the positions of the \Sun\, and the \Moon\, are <correct>. It is absolutely necessary that two degrees (occasionally three) in any sign be in the Ascendant, whenever the total number
of the \Sun\, and the \Moon\, falls in the beginning of a sign or in the middle or at the end. The addition or subtraction occurring at the equinox will show the ratio of the degrees<?>. The same is true for the \Sun\, and the \Moon: the \Moon\, passes through a sign in 2 1/2 days. The gnomon, when halved, yields the same ratio. Nature, sending to us emanations from her immortal elements, creates and fashions piece-by-piece the universal structure of everything, unalterable and invariant. 

Nature directs the universe without exceeding the bounds of law. \textbf{/344K/} She supports the cosmos, awakening and recycling it to immense ages. Sometimes she destroys, expends, and brings to oblivion the tribes of men and beasts, and the kinds of plants and crops; sometimes she begets, nourishes, and rejuvenates others. No earthly thing is everlasting or extremely long-lived, nor is any <totally> destroyed or desolate, thus causing the bereft earth to assume a formless character. 

No, the earth is piloted by the heavenly bodies, glorified by the good things in it, made splendid and transfigured by the different colors, and takes on its lovely shape—for none of the elements is unshapely. These elements rejuvenate the sea which is exercised by the winds and tides, and because of its needs <?> the elements restore it with streams and springs pouring down out of the earth. Filled at all times, <the sea> never comes to an end, nor indeed does it flood over beyond its appointed limits. Although spreading widely with its thousand billows, it seems to stand still, although not motionless. The sea nourishes the multitude of fish swimming in its whirls and eddies, and it has made some for whales, some for the use of men, some for food for other fish.

Not even the apparently empty air is to be considered useless and inactive: it is reined in by the winds and alters its direction so quickly that it seems ever-changing. Unperceived by us, it creates our vital spirit with its mildness. By directing the many tribes of winged birds sporting in the air, it provides stimulation and delight at the sight of them.…<fire>\textbf{/331P/} 

In such a way each of the elements changes one to the other according to natural law, transforming itself and taking on its own beauty and its own value to make manifest the universal structure. An element that stays in its own form to encroach on the other elements is nothing but useless and harmful. But when blended with another, it creates a temperate state, and when permeating everything, it is not destroyed by anything. In our view, the earth seems to referee the other elements, controlling the universe as its creator.

\newpage