\section{A Method for the Length of Life Using the Apogonia (8K, 9P)}

All of the preceding methods are effective and easily understandable to those who study them, and they result in the same answer \textbf{/345K/} i.e. the same degree-position, but not the same number of years. Consequently those who wish to discover the <correct number of years> must approach the calculation with all eagerness and zeal, because the one who is willing to work gets what he desires. 

Toil and constant thought accompany every business, whether good or bad: royal, governing, or ruling affairs; matters of wealth and poverty; and the arts and sciences as well. Moreover, neither pleasure nor enjoyment directs matters in a way that lacks care and grief. Rather these two lead to decline, conceit, and endless mental pain. I have set a table rich in learning and I have invited guests to the banquet. Let those who wish to feast act with the physical assistance of the body, which helps them to use the nourishment not in a greedy or insatiable way, but only in so far as the victuals can provide reasonable pleasure. (What is consumed beyond the bounds of nature usually causes harm.) Now if any of the guests should wish to continue living unharmed, let him eat one or two courses, and he will be happy. To make a comparison: when small scraps of wood come in contact with fire, they make a great, towering blaze which is very overpowering and bright, but which falls in on itself rapidly and becomes dim. The glow of the fire is quenched, and a billowing, thick smoke and a strong, tear-producing stench surrounds the bystanders. A thick haze surrounds those who are farther off. In the same way, if anyone spends some time on one or two of the preceding methods, he will find his goal to be easily grasped, and he will spend his time in pleasure and delight and will enjoy great repute. If, however, anyone is slow to understand what he reads, yet wishes in one \textbf{/332P/} day to run through two or three books, he will not discover the truth. Instead, he will be like a storm-fed river, rolling its burden along, worthless and profitless to the onlookers, and sinking back quickly to its useless state. Nor does a racehorse running in a desert place, outside of a stadium or a battle, win any prizes. If the river carries profitable loads, men will readily leap into it to win the profits, even if the river is swift and dangerous. Or again, a ship running swiftly on course gives great joy to its sailors. The horse who runs with determination delights in the praise <showered on him>, \textbf{/346K/} attracts many admirers, wins much attention, and gains prizes by his labor. It is just so for those who enter these mysteries with keen intelligence: they are worthy of the prize, and they do gain pleasure and profit for themselves. On the other hand, those who enter perfunctorily rapidly come to jeer at this art, because fate has not granted them ready understanding and immortality. It was not enough for Nature to make it possible for men to know the epicyclic theorems of the stars, with their stations and invariable passages. In addition, Nature fitted everything together by means of this
circular motion. As a result, mortals’ affairs are destined to be intelligible. An exact knowledge of the things administered or planned <by Nature> on earth is difficult of attainment by men, e.g. the dimensions of the klimata and nations, the boundaries and the depths of the sea, since this smallest and weakest of creatures (as one must guess until one comes closer) does not have the power of seeing afar. Nevertheless, men can share in immortality and in anticipation can be associates of the Gods through their investigation of the celestial circle, the motions of the stars, the courses of the Sun and the Moon, the subdivisions of the years, months, and hours, the tropics and the variations of weather, the contacts and separations, and because of the resulting foreknowledge.

If indeed it is true as the Poet says: Meanwhile Kyllenian Hermes was gathering in the souls of the suitors, etc. <Odyssey 24.1> then obviously this god <Hermes> has a nature partaking of earth and of heaven, and he conducts the souls of men aloft, around the astral regions of the cosmos, surrounding the souls (particularly of those who are immersed in these matters) with inspired, scientific, intellectual forces. \textbf{/333P/} Those wicked men who are blind to these matters not only miss their share of immortality, but even their humanity: they are herded together like brute beasts, they later pay a justly deserved penalty for their greed and their rash reasoning, and they do not escape the law.

\textbf{/347K/} It is then perfectly obvious that the gods can attend to men and can supply them with the finest and most respected benefits. Wishing men to keep the laws which they have made, the gods do not nullify the Fates; rather they confirm their effective control of human affairs with unbreakable oaths. For there is among the gods a fearsome and respected oath “By the Styx,” an oath which is accompanied by a steady cast of mind and unalterable Necessity. The Poet is a witness to this when he says: A golden chain reaching down from Heaven, etc. <Iliad 8.19>

The poet portrays Zeus making this threat, a god who can do what he says, but the Poet also mentions that Zeus does nothing to transgress the law nor does he do wrong among the gods. The following verses are said in a mystic fashion, not as some have taken them, when the poet reminds us, in connection with the \textit{aristeia} of Hector, that as long as Hector’s basis of years remained, he was unconquerable, killed many men, advanced beyond the tomb, broke down the gates, and burnt the enclosure: He raged like destructive fire or like spear-shaking Ares. <Iliad 15.605> and he seems to be helped by Apollo. But when his doom faced him and his years were fulfilled, he was struck by Achilles and He went down to death, and Phoebus Apollo forsook him. <Iliad 22.213>

The same is true of Achilles: he filled the plain of Troy with blood and the river Xanthus with corpses. He appeared to battle the gods—all with the support of the gods. Then he was deserted by Athena and killed by Paris, while his goddess mother stood by. 

In the case of Diomedes and Odysseus too, Athena in sleep made the Trojans drunk so the Rhesus, the Thracian king, might be killed. In so doing Athena glorified Diomedes and Odysseus. Other proofs of this point have been collected and transmitted by many writers. With respect to those men who have strength of body, who are helped in their effective actions by the chronocrator, and \textbf{/334P/} who seem to associate with and stand among the gods, one should infer that even \textbf{/348K/} the gods are agents of the Fates: at times they become helpers of men, at other times enemies. For nothing is accomplished among men, for good or for bad, without reason. In addition, Fate is preparing the future by means of agreements, friendships, rank, and existing associations; also through hatreds, injuries, disease, secret and religious matters, death, etc. All this being given, the discussion of the power of the \textit{apogonia} must begin. 

Critodemus created the basis, but I myself previously discovered an approach, explained it in other books, and now, having made a more detailed investigation, I will expound it further. I must make clear that the approach to any method at the beginning is as yet incomplete, but when researched at length, it becomes more solid. Now if anyone should debate which is better, the clever and penetrating compiler, or the discoverer of solutions, he would declare (in my opinion) for the discoverer. For a musical organ does not produce praise for its builder, but rather for the person who can skillfully produce a musical tone through the air’s action. Likewise every type of instrument making—or the compilation of procedures—which does not have an expert performer of the activity is considered to be empty, useless, and vain. If someone could distinguish music by modes and recognize their effects, he would provide not only pleasure and delight <to his listeners>, but also profit and fame <for himself>. On the other hand, I have heard of many learned men who depreciate certain compilations because of their obscure and recherché quality—but we should turn our attention to the matter at hand.

In this method, the Sun and the Moon are mutually supportive. (It is not necessary to prove at length their mutual cosmic sympathy and harmony, since these have been discussed previously.) The explanation
is complicated and I will outline first the opinions of others. 

First it is necessary to determine the \Sun’s degree-position. Then consult the table of \textit{apogonia} (under the appropriate klima) and multiply the degree-number entered there by the degree-position of the \Sun. Note the result. Next consult the table of rising times at the \Sun’s degree-position. Multiply the <hourly> magnitude entered there by the time (in hours) of the nativity. \textbf{/349K/} Without adding <?>, multiply this number by the result of the previous multiplication, \textbf{/335P/} viz. that of the \Sun’s degree-position times the figure in the table of \textit{apogonia} expressed in minutes. Divide the resulting total, either 10,000’s, 1000’s, or 100’s, by 30 and note the remainder of the division. See what fraction of 30 this figure is, then add this to or subtract this from the remaining part of the 30 or the magnitude of the sun’s degree-position. Now having found the degree-position, consult the table of \textit{apogonia}, and make the forecast according to the years entered there, using the three factors.

This method did not seem worth keeping after I made a clearer explanation of it. The error in the figure not only obscures the intention <of the forecast>, but also makes one forget the previously calculated
magnitudes. If someone were willing to enumerate erroneously in the second or third step, he would be terribly in error, because such a complication would confuse even me—and I am active and eager in such matters.

For the most part, I did my calculations from the Sun and the Moon, and I acknowledged that the calculation of lifespans and ends is derived from these stars, and I directed my attention to the mystic three sign
system (which I have set forth previously). Thus it is necessary to determine the Sun’s and the Moon’s hourly motion, expresse[d] in minutes. (I say this frequently so that I will not seem to be in error.)
Then count from the Sun to the Moon or from the Moon to the Sun: the same position will be mystically arrived at. 

Having summed up the total number, divide by 30. Determine what fraction of 30 the remainder (which will be less than 30) is, and take this fraction of the equinoctial times. This fraction of the remainder must also be added to the Sun’s degree-position. The result will show the \textit{apogonion}, i.e. the degree in the Ascendant. Alternatively, add to or subtract from the equinoctial times, depending on the
nearest hour.

Every nativity has gnomons at two degree-positions: if we investigate the ratio of the two Lots using the method appended by me, <just as we did> for the Sun and the Moon in their degree motions, it will not be different, but we will find it in the same ratio, not being greater \textbf{/350K/} or lesser than the equinox <=30>, because the equinox is the universal gnomon, the commander of the klimata, and the just mediator of day and night.

\textbf{/336P/} So, when the degree is found, we enter the table of \textit{apogonia} and research the years using the first, second, and third factors. If the degree is in a sign of short rising time, but the basis of the nativity is receptive to a number of additional years, it is necessary to add the years of the third factor to itself, then combine the third factor with the first and second. Having done this, make the prediction. Similarly in the signs of long rising times: if this <great> number of years is <not> in effect, calculate the entire factor, then apply the calculated degree-position to the second factor. Alternatively, moving up <in the table>, add the years entered at the degree in the first factor.

The result will be discovered to be quite accurate if the investigator tests it according to the previous account, counts it out, then finds the years which correspond to the \textit{apogonia}. One method combining with another contributes great certainty. For “one” by itself attains nothing and has a vague and evanescent utility, since it is unsupported and aided by nothing. We see that infants and the very old have a precarious gait, and in the same way the blind make their way leaning on a staff. Moreover nature has not created for men anything useful which is at the same time self-contained and complete: night accompanies day, death life, black white, dry moist, bad good, bitter sweet, etc. Each marches with and is completed by the other. For some it indicates good hopes for life, livelihood, and safety, and brings hope for survival; for others it indicates only despair and a wish for death, because of its afflicting crises and their compulsions.
…
When the Ascendant falls at the beginning of a sign, it is necessary to investigate the preceding sign with respect to the \textit{aphetic} point. When it falls at the end of a sign, see if the following sign is in agreement. Critodemus has used the Sun in this method also, another procedure which he has not published.


\newpage