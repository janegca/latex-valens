\section{The Determination of the Sign and the Degree in the Ascendant (6K, 7P)}

The sign in the Ascendant is found (for day births) by counting the number of the \Sun's degree-position from the \Sun's sign, giving 1° to each sign. \textbf{/327P/} The sign where the count stops is in the Ascendant for the nativity—or the sign corresponding to it, in both the diurnal and nocturnal hemispheres. 

If the degree is located in the nocturnal hemisphere or if the nativity is diurnal, the sign in opposition or the sign square with it will be in the Ascendant. 

For night births, count in the same way the degree-position of the \Moon\, from the \Moon’s sign. 

Alternatively: it is necessary to start counting the \Sun’s dodekatemorion from the sign trine to the left. The sign where the count stops will be the Ascendant—or the corresponding sign.

Alternatively, using a compelling method: in the cases when the \Sun\, is in the sign of the new moon while the \Moon\, is traversing the sign just following the new-moon sign, the Ascendant will be in the sign of the new moon, or in the sign sextile, trine, or opposition, the inclination of the \Moon\, determining which one. 

When the \Moon\, is with the \Sun\, and at an angle, if the \Sun\, leaves the sign of the new moon while the \Moon\, \textbf{/341K/} is still traversing its first cycle <=phase?>, the Ascendant will be found in the sign square with the \Moon\, or in a sign which is unaspected by the \Moon. 

If <an astrologer> knows whether the birth was day or night, but does not know the hour, he must draw up <a horoscope> with two signs in the Ascendant

\newpage