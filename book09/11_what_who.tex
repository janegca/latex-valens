\section{What Tables Should Be Used; Who Should Be Followed; That Nothing Is In Our Power (11K, 12P)}

I believe I have compiled the powers of the preceding <methods> in a sufficient, even generous, fashion. With this being done, there is something I wish to leave to scholars for their investigation and reflection. I am not speaking now to the uninitiated, but to those who are keen about these matters, so that they too can become aware of this multifarous and complex art, which reaches its peak by means of its many paths, its ins \textbf{/353K/} and outs; in so doing, they may seem to associate with the gods. 

It is clear from what has been and will be said that this art by itself has an everlasting, irrefutable, and eternal foundation. \textbf{/339P/} It is also obvious that it is occasionally in error, in view of the weakness of the practitioners of the science and the fact that they are not experienced in the variations from one astronomical table to another. I will omit any mention of those who construct tables of rising times and of the variations in their diagrams and numbers, in the motions of the \Sun, the \Moon, and the other stars which these astronomical tables show. Even the \index{year!length}length of the year has been fixed at different values: \index{Meton@\textit{Meton}}Meton the Athenian, \index{Euctemon@\textit{Euctemon}}Euctemon, and \index{Philip@\textit{Philip}}Philip fixed it at 365 1/5 1/19; \index{Aristarchus@\textit{Aristarchus}}Aristarchus of Samos at <365> 1/4 1/162; \index{Chaldeans}the Chaldeans at 365 1/4 1/207; the \index{Babylonians}Babylonians at 365 1/4 1/144; and many others at various values. If then in the four-year cycle, one day coming around shows the precise degree in the astronomical table of the \Sun, why then would it not necessarily be correct to determine the exact degree position <of the \Sun> by adding the appropriate motion to the day in question, using whichever year-length one had calculated?

I had reasoned to myself that the previously mentioned men were aware of the power of calculation, but had not discovered the determination of the length of life. If they had researched this, they would certainly have added this missing part to their astronomical tables. So I myself have tried to construct a table of the \Sun\, and \Moon\, using the eclipses: but since \textbf{/354K/} time prevented me from bringing this to a conclusion, I was brought to say, along with \index{The King}the King, 
\begin{quote}
“Others have beaten these paths, and because of this I omit mention of them.” 
\end{quote}
I thought it best to use \index{Hipparchus@\textit{Hipparchus}}Hipparchus for the \Sun, \index{Soudines@\textit{Soudines}}Soudines, Kidynas, and \index{Apollonius@\textit{Apollonius}}Apollonius for the \Moon, in addition to Apollinarius for both bodies (if one applies the addition-factor of 8°, which I believe to be correct). He however calculated quite well the tables with respect to the observed motions, but he confesses (being mortal) to have erred by one or two degrees. (Absolute accuracy and precision is for
the gods alone.) 

For the rising times they used the \textit{proenklima}, and they calculated the the 14 klimata.

First of all, it is necessary to observe with all accuracy the numbers of the \Sun, the \Moon, and the five stars, with the time’s/hour’s relationship to them being the referee of their mutual aspects, because it is from this that the Ascendant is known and the XII Places are positioned by degree. \textbf{/340P/} If the investigation appears accurate in the way described, it will make the forecasters famous, it pleasantly confirms good and bad for the connoisseurs, and it brings eagerness, encouragement, and belief in the words of those who wish to make such an investigation.

If it were generally true that the rich man never became poor, or that the man who by good fortune has attained the kingship, rule, fame, or any pinnacle whatever always continued secure in his good fortune, or
that the strong man continued hale and hearty, or that the man lucky in business never went bankrupt, or that the sea captain never was swamped by waves or sailed off course in his voyages, or that the doctor never was sick, the seer never suffered, or the prophet gained eternal possession of the good which the gods give to men—if all this were true, then the \index{astrology!prognostic art}prognostic art would not be useful. Each man would keep what he was allotted, would occupy himself with his portion, and would live his span of years without anticipating anything new. But as it is, all of man’s affairs are insecure and unsound. They are seen to be shaky, ready to turn into their opposites: the king becomes \textbf{/355K/} a prisoner and a slave; the rich man becomes poor and needy; the strong and powerful man becomes crippled and helpless; and so on.

Everything that is beautiful and fine in life, everything concerning health, beauty, fame, and business, changes into something else and gives men the “opportunity” of suffering what they had not suffered before. Rarely does anyone conclude a life free of reproach and care. Most men, according to the basis of their own nativity, experience vicissitudes from day to day in their fortunes. For this very reason and due to the information derived from my forecasts, I know myself, I know the foundation which my \index{Fate}Fate has assigned me, and I know that it is impossible for anyone, contrary to Fate, to become different from what he is. Therefore I have not become a lover of positions of command, rule, or any other prominent rank; or of lavish wealth, of possessions, or of numerous slaves. I have not become a slave of desire, an impious flatterer of the gods and of men, hoping to gain what the Godhead does not want to grant. Just as an intelligent slave of a bad master knows his master’s character and his daily behavior,
and therefore he does his duties in an orderly manner: he does not contravene the master’s orders, and in acting thus, he considers his station to be free from pain and suffering. In this same way, I do not view my service as labored and strained. I have abandoned all vain hopes and thoughts, and I have kept the laws of Fate.

\textbf{/341P/} If someone who loves inquiry and who has strengthened his intelligence wishes to learn about what is and what will be from a learned man, he will despise vulgar matters and will become a devotee of those things that suit the foundation <of his nativity>. Transforming himself day by day, he will obliterate any fear of the evil he must suffer. The bad will be blunted and worn away by his contentment, and he will bear voluntarily and in good order the end of his life, acting under his own self-control just as if he were under the command and control of another. If anyone wishes to learn from experience how this can be so, let him compare <this state of mind> with the thoughts of an unlearned man, and let him do the opposite.
I mean, if he is poor, let him become rich; if lowly, a commander; if inactive, successful, uncriticized, without grief or care. (For all \textbf{/356K/} men are by nature lovers of good things.) If someone attains all this, he will despise Fate. But it is impossible that whatever he wishes to be accomplished should remain unchanged to the end. For that reason it is advantageous for Fortune to be…and to remain unsteady,
because men do not bear fortune’s favor <indefinitely>. Just like those maddened by stinging gadflies, governed by many masters, and suffering the goads of desires and passions, they pay an appropriate penalty, even though unwilling.

To some simple minds it seems right to say, “Everything is under our own control.” Being unable to prove this by experience, they resort to saying that this is partly true: “Some things are under our control,
some under Fate’s.” Going this far, they impudently move on to circular and inappropriate conclusions, saying: “Leaving my house is in my power, as is bathing, going where I wish, carrying on some business,
buying, meeting friends” and other matters. 

Now I declare to these men that the opposite is the case, that not even these trivial matters are under their control. Their very choices go to the contrary because of some unforeseen cause. I myself for example have often wished to do some business or to meet with a friend; having chosen a propitious time for the meeting, I did not attain my goal nor get to where I was going.

On the other hand, when I did not desire this, the very thing has happened\footnote{It would always be necessary for the time to be appropriate for the thing that is about to happen. -marginal note [Riley]}. For this reason, an intelligent man should follow where God wishes to lead him (for God readies <man’s mind> for what he wishes). Or the intelligent man should choose propitious times \textbf{/342P/} and after casting an Initiative for the business, taking into account the universal motion, he must examine the forecast resulting from the current stellar positions and the position of the Ascendant.
 
\newpage