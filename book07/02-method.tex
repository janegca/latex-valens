\section{A Method for Propitious and Impropitious Times and Length of Life, with Respect to the Houseruler, the Rising Times, and the Ascendant (1K,2P)}

In the previous books we explained the rising times and what effect they have on the distribution of the chronocrators. Now we must \textbf{/263K;252P/} add additional clarification. 

For \mndl every nativity, after the stars  have been accurately charted, it will be necessary to examine how the houseruler is configured, which stars
it has in aspect, whether it is rising or setting, and if it has a proper configuration or is out of its sect. <It is also necessary> to inspect the star which is allied with it, the Lot of Fortune and its ruler, and of course, the status of the nativity.

If the houseruler is found to be at an angle or operative, it will begin the chronocratorship; the rule will then pass to the stars next in order in the order of signs. If these stars are in the same sign, their positions
in degrees must be calculated, since the one in the leading position will rule first. 

It is of primary importance to inspect the Ascending angle and to see how its houseruler is configured: if they happen to be together or appropriately configured in different signs, the houseruler will have a good effect on the length of life and on mental activities. They have a bad effect when setting, in opposition to each other or inappropriately configured, because when the houseruler completes the rising time of the sign in the Ascendant, the rising time of the sign in which it is located, or the period of the star in whose sign it is located, it will abdicate from the chronocratorship; in so doing, it will make the native short-lived, or it will allot months instead of years, days instead of months.

If the ruler of the Ascendant is not in conjunction, but another star is, then the latter star will begin the chronocratorships. If several are in conjunction, they all will share <in the chronocratorship>, with the star closest to rising or most related to the <Ascendant> sign ruling first. Similar forecasts can be made from the nature of the star and of the sign. If they are not found at the angles but are just preceding them, they will rule the chronocratorships briefly, or for as long as they transit a part of the sign (in degrees), and they will fall short of their own numbers and rising times.

After examining the Ascendant, it is necessary to investigate MC and to determine its ruler; then examine the Descendant and IC in the same way. If <their rulers> are not found at the angles, the rulerships which follow the angles must be examined; if the stars are not in those signs either, the signs which precede the angles must be inspected. (Even if such positions do not possess a great deal of influence over activities and do not allot the maximum length of life, they are nevertheless active.) 

It is also necessary to take into account the contacts and aspects of the Moon and the ruler of its sign, because if it is found at an angle and has the configuration of a distributor, it will be the first chronocrator.
Following it, \textbf{/265K/} the stars closer to the angles or passing through a phase will rule. Moreover, rising stars differ from evening stars.

\textbf{/253P/} After \mn{basis} these matters have been researched accurately, it remains necessary to know the general basis of the nativity, i.e. to what high or low rank it belongs, the harmful and the helpful stars, which star has been assigned the rulership of which place and type of forecast (i.e. activity, rank, wife, children, father, mother, brothers, and whatever else applies to the body, the mind/soul, and the livelihood) so that the result of the distributions might be very clear.

But we have explained these matters in the preceding books. However, the same stars often have mastery over many matters which they bring about in their own times of rule. If they happen to be together at the same time, they make their influences obvious then too. Their times of rule must be determined from the rising times of the signs and from the periods of each star. 

If \mn{timing} the stars are at or following the angles, they allot the total rising time and period of each star. If the period of the star is found to be greater than the rising time of the sign, the star which has the chronocratorship allots its period. If the rising time
is greater than the period, we allot the rising time. Occasionally they allot their total years <rising time+period>\footnote{Schimdt translates this as ``But they will distribute their entire allotment of their years in either case'' (VRS7 p.10).}. If many stars are together, and if the periods of all exceed the length of the rising time, they allot the periods. If the periods are less than the rising time, they allot the rising time. 

<Not \mn{Planet Min/Max Times} all stars allot their maximum periods>, only those which have some relationship with the allotting sign. Those which have no relationship (e.g. those which precede the angles) allot their minimum period\footnote{Only those planets that are rulers of the sign by domicile, exaltation, or triplicity (or term? face?) allot their maximum periods IF the sign is angular or succedent. If the planet is peregrine in the sign it will allot its minimum years, as will a planet in a cadent (delcining) sign.}.

If \mn{rising times} a star happens to be in its proper face\footnote{Schmidt takes this to mean a planet is in its own decan believing the other definition of proper face, that ``the \textsl{z\={o}dion} of the planet has the same relationship to the \textsl{z\={o}dion} of the \Sun\, and/or the \Moon\, as the domicile of the the planet has to the domicile of the \Sun\, or \Moon'' is solely derived from Ptolemy  (VRS7 p.10).} or is in an appropriate sign\footnote{For Schmidt, an ``appropriate sign'' is one the planet rules by exaltation or triplicity (VRS7 p.10).}, it will allot only the rising time of its sign. The results will come to pass when the rising times or the periods are completed. 

If \mn{No time allotted} the stars are found to precede the angles, they allot neither the full rising time of their signs nor their full periods.

It \mn{malefics} is necessary to be aware in advance that malefics are not always harmful; they can be helpful and the cause of life and rank. In the same way benefics can bring dangers and harm—all of which can be foreseen through the appropriateness of their configurations or their oppositions. I have come to understand by experience that malefics \textbf{/266K/} are really malefic for average or humble nativities: these nativities are involved in all sorts of troubles, and even if the benefics seem to help momentarily, later they take away what was gained and render the native unsuccessful and unfortunate. Sometimes however they do give these men good physical condition, and make them excited by food, indiscriminate or cheerful about sex, and laboring with pleasure.

With respect to \textbf{/254P/} lofty nativities, malefics make men active, renowned, successful. Still, even then these stars do not cast off their own nature: these men have a bold, terrifying, dictatorial, greedy, and destructive nature; they desire others’ goods and disgrace themselves with lawless and violent deeds; their rank is easily lost and their fortune easily changed. As a result, such men do not direct their offices, their position of leadership, or their lives in an orderly manner, but they are involved in upheavals and plots, attacks or revolts of enemies and the masses, violence. While they are in office, plagues, famines, crop failures, cataclysms, earthquakes occur—depending on the configuration of each malefic. As a result, such men live wretchedly, suffering what each malefic portends. They are involved in violent or unexpected deaths and pay the penalty <for their crimes>. The fortune of such men becomes a subject of story, and
their end is far-famed. 

So \mndl as to make our explanations concise: the stars, when acting as chronocrators, will cause the results indicated by the configurations which they are found to be making with each other at the nativity, whether good or bad, taking into account the basis of the birth. Moreover, those stars which are in superior aspect or which are opposed to each other are most powerful; those which are in their own or in opportune signs and which rule the chronocratorship are most vigorous and active in producing results. 

Let the opposition of \Saturn\xspace and \Jupiter\xspace be considered productive and beneficial, particularly when these stars are in their own
signs, provided that this configuration is not harmed by another afflicting influence. Likewise, the mutual configuration of \Jupiter\xspace and \Mars\xspace is good, if it happens to be in appropriate signs. 

One star in another’s sign and having some relationship with it is productive and beneficial during the applicable chronocratorship. 

Each star will be operative at the end of its maximum period.

\newpage