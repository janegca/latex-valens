\section{Hostile Stars and Critical Places. The First Table of Critodemus (8K,6P)}

It is necessary to investigate the hostile places and stars, not only with respect to the other stars, but also for the Ascendant, the Sun, and the Moon. When they come into opposition, they indicate critical periods and death. 

Take Saturn for example: note which god controls the degrees in opposition to the position of \Saturn, as given in the table. The native will die when \Saturn\xspace is there, is square with the Ascendant, or in signs with the same rising time, depending on when the chronocratorships coincide. The
same must be done for the other stars, because the rulers of the terms of the degrees in opposition are hostile. These stars indicate destruction when they come to these places or to the places with the same rising times as the Ascendant.

For example: \Saturn\xspace in \Cancer\xspace 21°, terms of \Venus. The point in opposition is \Capricorn\xspace 21°, terms
of \Mars; \Mars\xspace was in \Taurus\xspace 27°. The native will die when \Saturn\xspace is in \Virgo\, because it was square, as calculated by degrees\footnote{\Virgo\, is square to \Gemini\, and \Sagittarius, neither of which are involved here. Possibly Valens means \Libra\, which is square to both \Cancer\, and \Capricorn, and which is ruled by \Venus, \Saturn's term ruler.}.

\Jupiter\xspace in \Scorpio\xspace 14°, terms of \Saturn. \Taurus\xspace 14° <the point in opposition> is also in the terms of \Saturn, and this star does not become hostile to itself\footnote{In the Egyptian bound system 14 \Scorpio, is ruled by \Mercury, and 14 \Taurus, by \Jupiter. Valens is using his own term table here and in the other examples in this section. He describes the table in the next section.}. \Leo\xspace has the same rising time as \Scorpio, and \Leo\xspace 14° is in the terms of the \Sun\footnote{The \Sun\xspace itself is not allocated terms (neither is the \Moon) in the Egyptian or other terms but Valens does allow it in his term table.}. So \Jupiter\xspace is the anaereta when it comes to the places of the \Sun.

\Mars\xspace in \Taurus\xspace 27°, terms of the \Sun. The same position in \Scorpio\xspace <the point in opposition> is also in the terms of the \Sun. Since the star does not become hostile to itself, I then investigate \textbf{/136P/} \Leo\xspace 27° or the sign of equal rising time <with \Taurus>, which is \Gemini\xspace according to the hourly intervals.

\Gemini\xspace 27° is in the terms of \Venus. The native will die when <\Mars> is in \Scorpio\xspace or <\Pisces>, which have the same rising times <as \Leo\xspace and \Taurus>, or in the signs square with them. If anyone calculates \Leo\xspace 27°, he will find it to be in the terms of \Saturn. \Saturn\xspace was in \Cancer. So the native will die when \Mars\xspace is in \Cancer, \Sagittarius, or the signs square with them.

\Venus\xspace in \Scorpio\xspace 27°, terms of the \Sun. The point in opposition is \Taurus\xspace 27°, terms of the \Sun,
This star does not become hostile to itself, \textbf{/144K/} so I investigate \Scorpio\xspace 27°, of equal rising time,
which is in the terms of \Mercury. The native will die when \Venus\xspace is in \Virgo, where \Mercury\xspace was, or in the signs square with it. The same procedure should be followed with respect to \Mercury.

In casting horoscopes for patients struck down by illness, it is necessary to examine the places in opposition, the stars in the hostile places, and the stars causing the monthly, daily, and hourly critical periods, with respect to the degree-position/sign of the Moon in which the opposing star is found.

\ldots

The vital sector will be considered as starting from the Sun, the Moon, or the Ascendant, or from the star found following the Ascendant, then the other <stars> in sequence in order of sign and degree at the nativity, making the determination <of the chronocrators> by the 10 year 9 month system.


\newpage