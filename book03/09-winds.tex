\section{Winds and Turns (9K,6P)}

The arrangement some have made of the terms using the seven-zone system, i.e. 8, 7, 6, 5, 4 (and they are not in agreement even as to that) does not seem correct to me. I prefer the arrangement derived from houses, exaltations, and triangles, to wit:

\Leo\xspace is the house of the \Sun, \Aries\xspace is its exaltation, \Sagittarius\xspace is the $<$other member [o]f its$>$ triangle. The total is 3, and so in each sign the \Sun\xspace has 3 terms.

\Cancer\xspace is the house of the \Moon, \Taurus\xspace is its exaltation, \Virgo\xspace and \Capricorn\xspace are the $<$other members if its$>$ triangle. The total is 4, and so in the same way the \Moon has 4 terms in each sign.

\Capricorn\xspace and \Aquarius\xspace are houses of \Saturn, \Libra is its exaltation, \Gemini\xspace is the $<$other member [o]f its$>$ triangle. The total is 4, and so \Saturn\xspace has 4 terms in each sign.

\Sagittarius\xspace and \Pisces\xspace are houses of \Jupiter, \Cancer\xspace is its exaltation, \Aries\xspace and \Leo\xspace are the $<$other members [o]f its$>$ triangle. So \Jupiter\xspace has 5 terms in each sign.

\Aries\xspace and \Scorpio\xspace are houses of \Mars, \Capricorn\xspace is its exaltation, \Pisces\xspace and \Cancer\xspace are the $<$other members [o]f its$>$ triangle. So \Mars\xspace has 5 terms in each sign.

\Taurus\xspace and \Libra\xspace are houses of \Venus, \Pisces\xspace is its exaltation, \Virgo\xspace and \Capricorn\xspace are the $<$other members [o]f its$>$ triangle. \textbf{/137P/} So \Venus\xspace has 5 terms in each sign.

\Gemini\xspace is a house of \Mercury, \Virgo\xspace is its exaltation, \Aquarius\xspace and \Libra\xspace are the $<$other members [o]f its$>$ triangle. The total is 4, so its terms in each sign will be 4.

$<$The Order of the Terms.$>$

\Aries\xspace \Leo\xspace \Sagittarius\xspace by day: \Sun-3, \Jupiter-5, \Venus-5, \Moon-4, \Saturn-4, \Mercury-4, \Mars-5. Total 30 by night: \Jupiter-5, \Sun-3, \Moon-4, \Venus-5, \Mercury-4, \Saturn-4, \Mars-5. Total 30.

\Taurus\xspace \Virgo\xspace \textbf{/145K/} \Capricorn\xspace
by day: \Venus\xspace \Moon\xspace \Saturn\xspace \Mercury\xspace \Mars\xspace \Sun\xspace \Jupiter\xspace
by night: \Moon\xspace \Venus\xspace \Mercury\xspace \Saturn\xspace \Mars\xspace \Jupiter\xspace \Sun\xspace

\Gemini\xspace \Libra\xspace \Aquarius\xspace by day: \Saturn\xspace \Mercury\xspace \Mars\xspace \Sun\xspace \Jupiter\xspace \Venus\xspace \Moon\xspace by night: \Mercury\xspace \Saturn\xspace \Mars\xspace \Jupiter\xspace \Sun\xspace \Moon\xspace \Venus\xspace

\Cancer\xspace \Scorpio\xspace \Pisces\xspace by day: \Mars\xspace \Sun\xspace \Jupiter\xspace \Venus\xspace \Moon\xspace \Saturn\xspace \Mercury\xspace by night: \Mars\xspace \Jupiter\xspace \Sun\xspace \Moon\xspace \Venus\xspace \Mercury\xspace \Saturn.

So that you will see the accuracy of this arrangement of terms—you can also recognize it in the nature of the winds. If the \Sun\xspace is transiting its own terms, with the \Moon\xspace or the ruler of the \Moon’s terms in aspect, the physical nature of the star will manifest itself in the wind. For instance: if the \Sun\xspace traverses its own terms with the \Moon\xspace in aspect, it will blow westerly. If \Venus\xspace traverses its own terms, it will blow southerly and dry. If \Saturn, it will blow westerly and bring moisture. If \Jupiter, it will be northerly and wet. If the \Moon, it will be northeasterly. If \Venus, southeasterly and there will be shifting winds and storm clouds. If \Mercury, there will be westerly and northerly winds, shifting and bringing heavy rains, thunder, and lightning. 

If any of the stars are in aspect with the \Sun\xspace and the \Moon, it is necessary to watch closely—in addition to the nature of each luminary—which phase the \Moon\xspace is passing through, i.e. full or new, and in whose terms the phase is located. Then make your forecast with reference to the ruler of the terms and of the stars in those terms or in aspect.

\ldots \textbf{/138P/}
\newpage