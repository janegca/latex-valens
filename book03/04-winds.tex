\section{The Winds of the Stars, their Exaltations, and their Steps}
\subsection{\textit{[Exaltations and Depressions]}}
\index{planets!exaltations}
\index{planets!winds}
\index{winds}
Having established this, it is necessary to append the winds. First it is necessary to examine the \mn{Exaltation degrees} degree in which each star is exalted; from these the determination is made:

\begin{table}[h]
\caption{Planet Exaltation Degrees}
\vspace{-1em}
\begin{center}
\begin{tabular}{cccc}
\toprule
\textsc{Star} & \textsc{Degree} 
	& \textsc{Exaltation} & \textsc{Depression} \\
\midrule
\Sun 		& 19°		& \Aries 			& \Libra			\\
\Moon		&   3° 		& \Taurus			& \Scorpio  		\\
\Saturn		& 21°		& \Libra			& \Aries			\\
\Jupiter		& 15°		& \Cancer			& \Capricorn		\\
\Mars		& 28°		& \Capricorn		& \Cancer 		\\
\Venus		& 27°		& \Pisces			& \Virgo 			\\
\Mercury	& 15°		& \Virgo			& \Pisces			\\
\bottomrule
\end{tabular}
\end{center}
\end{table}
\vspace{-1em}
Each star has its depression at the point in opposition to its exaltation.

The point square with the exaltation and preceding it is called northern; the point square and following it is called southern\footnote{The Exaltation degree  is considered to be the planet's most elevated point and is equated with the north wind. The degree opposite the exaltation degree is considered the ``depression'' or lowest point and is equated with the south wind. The planet is seen to be completing a cycle going from its most exalted position to its lowest position and then rising back up to its most exalted position.} For example: the \Sun\xspace is exalted in \Aries 19°, and the point square with it and preceding is \Capricorn 19°. If the \Sun\xspace is found there, we say it is ascending north and the exaltation is exalted. From \Aries\xspace 19° to \Cancer\xspace 19° it is descending north. \textbf{/133P/} From \Cancer\xspace 19° to \Libra\xspace 19° it is descending south. From \Libra\xspace 19° to \Capricorn\xspace 19° it is ascending south\footnote{Essentially, a planet in a right square (10 signs from the exaltation sign) with its exaltation degree is moving up North, towards its exaltation degree. A planet in a left square (4 signs from its exaltation sign) is moving down south, towards its depression or fall.}.

\begin{table}[h]
\caption{Planet Winds}
\vspace{-1em}
\begin{center}
\begin{tiny}
\begin{tabular}{c r c c c c c c c c c}
\toprule
&   &   	& 
			& \textsc{Left} 
			&  
			& 
			& 
			& \textsc{Right} 
			& 
			& \\
&   &  	&
			& \Square
			&
			& \Opposition
			&
			& \Square
			&
			& \\
\toprule
&   		& \textsc{N}
			& $\Downarrow$ \textsc{N} 
			& \textsc{E}
			& $\Downarrow$ \textsc{S} 
			& \textsc{S}
			& $\Uparrow$ \textsc{S}
			& \textsc{W}
			& $\Uparrow$ \textsc{N} 
			& Ex \\
\toprule
\Sun 		&	19° 	& \Aries 	 	& $\Rightarrow$ 
							& \Cancer		& $\Rightarrow$ 
							& \Libra		& $\Rightarrow$
							& \Capricorn	& $\Rightarrow$	
							& \Aries		\\
\Moon 		&	3°		& \Taurus		& $\Rightarrow$
							& \Leo			& $\Rightarrow$
							& \Scorpio 	& $\Rightarrow$ 
							& \Aquarius	& $\Rightarrow$ 
							& \Taurus 	\\
\Saturn		& 21° 	& \Libra		& $\Rightarrow$
							& \Capricorn	& $\Rightarrow$
							& \Aries		& $\Rightarrow$
							& \Cancer 	& $\Rightarrow$
							& \Libra \\									
\Jupiter		& 15° 	& \Cancer		& $\Rightarrow$
							& \Libra		& $\Rightarrow$
							& \Capricorn	& $\Rightarrow$
							& \Aries		& $\Rightarrow$
							& \Cancer		\\		
\Mars		& 28°	& \Capricorn	& $\Rightarrow$
							& \Aries		& $\Rightarrow$
							& \Cancer		& $\Rightarrow$
							& \Libra		& $\Rightarrow$
							& \Capricorn 		\\	
\Venus	 	& 27°	& \Pisces		& $\Rightarrow$
							& \Gemini		& $\Rightarrow$
							& \Virgo		& $\Rightarrow$
							& \Sagittarius	& $\Rightarrow$
							& \Pisces	\\
\Mercury 	& 15°	& \Virgo		& $\Rightarrow$
							& \Sagittarius & $\Rightarrow$
							& \Pisces		& $\Rightarrow$
							& \Gemini		& $\Rightarrow$
							& \Virgo		\\										
\bottomrule
\end{tabular}
\end{tiny}
\end{center}
\end{table}
\vspace{-1em}
\subsection{\textit{[Winds and Steps]}}
If we seek the step of the wind, we find it as follows: since each step has 15°, we find the distance of the star from each degree <listed above>, then divide it by 15. For example: the \Sun\xspace is in \Aquarius\xspace 22°. I find the distance from \Capricorn\xspace 19° to \Aquarius\xspace 22°; this is 33°. I subtract 30° (2 times 15°), which is equivalent to 2 steps, with a remainder of 3°. So the \Sun\xspace is ascending north in the third step of that wind.

We have given this to use as an example. Note that the northern and southern hemispheres must be calculated when the rest of the winds are determined, as must the wind itself and the step also.

\subsection{\textit{[Winds and the Apheta]}}
\index{delineation!winds}
For each nativity it is necessary to note whether the \Sun, the \Moon, or the Ascendant is the apheta, and which wind it has. Then examine the other stars. If any have the same wind as the apheta, they will be related and associated, especially in their own chronocratorships\footnote{When they are ruling a time period.}. \mndl In fact they will be stronger and more effective if they are rising, \textbf{/141K/} at an angle, proceeding with their proper motion, and in their own sect.

If any star has a wind opposite to that of the apheta, it will oppose the apheta and will be malefic, especially at the transmission of the chronocratorship. 

\index{planets!benefic}
\index{planets!setting}
\index{planets!retrograde}
If the star is setting or proceeding with a retrograde motion, it will be harmful and hazardous. It will not be considered a benefic at all, even if it happens to be at an angle during this period. 

If any star has a configuration with the apheta which is related in some ways, unrelated in others, it will be variable, not entirely helpful or harmful. 

If the Ascendant is found to be the apheta, it will be necessary to examine the ruler of the terms, and to note which wind it has and whether it is at an angle, rising, or proceeding with its proper motion, then to compare it with the other stars.

Now some astrologers think that this procedure is useless; I say that it is most scientific and effective. In their astronomical tables, astrologers have worked out this topic in various ways, but they have not brought it to perfection.

\newpage
