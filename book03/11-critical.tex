\section{The 7-Day and the 9-Day Methods for the Critical Period (11K,8P)}

We will now report what the King has revealed about the critical point: take the number of days from the rising of Seth\footnote{Riley: The Dog Star <Sirius> in Egyptian - marginal note.} to the date of birth\footnote{In a later paragraph Valens makes it clear that we are using days from \textsl{the start of the year until the birth day} for determining the number of days. The Egyptians used the rising of Sirius as the start of their year.}. \textbf{/148K/} Divide the total number of days by 52 1/7\footnote{Total days of the year, 365, divided by 7 equals  52.143 or 52 1/7.}. Now multiply the <integer> answer of this division by the remainder, and examine the result of this multiplication to see if it is less than the particular number of the critical point. As a sample, let the number of days be 220\footnote{Presumably the number of days between the day Sirius rose and the birthday.}. I divide 52 1/7 four times <into 220> for a result of 208 4/7 days, with a remainder of 11 3/7 \ldots <I multiply 11 3/7 times 4, and the result is 45 5/7.> \ldots We call this the critical 7-day number.
<The King> tells us to examine this number to see if the number of the critical point is less than the number of years. 

\begin{mdframed}[backgroundcolor=cyan!5]
\tiny
The Critical 7-day Number calculation:
\begin{longtable}[h]{l l l}
a. & Days from birth div by weeks in year &
		$220 / 52.143 = 4.219$		\\
b. & Integer value of (a) & 4 \\		
c. & Days in full weeks from birth & $52.143 * (b) = 208.572$		 \\
d. & Days in remaining week & $220 - 208.572 = 11.428$	\\
e. & Critical 7-Day Number & $11.428 * (b) = 45.712$		\\
\end{longtable}
\end{mdframed}

Take the example just cited: if the basis of the nativity’s length of life is 47, but the critical point falls in year 45, the native will be destroyed then, since the value of the critical point was less than the years of the basis. <The King> has said that if the 7-day number is a factor of 9, the situation will be inharmonious. For example: this 7-day number <45> contains the factor 5, the result of a division by
9. When this is multiplied by 9, it yields 45. Therefore the 7-day number is a factor of 9. But if the value of the critical point is <not> less than the number of years <of the basis>, and if the previously mentioned considerations are true, it will not be able to reduce the allotted time. Instead the native will live the assigned period. If the situation is harmonious, the critical point will take precedence over <the
number found by> the preceding method\footnote{This text is not clear but based on the example of 45 as the crisis year and 47 as the basis years, Valens appears to be saying that if the critical 7-day year is evenly divisible by 9 and less than the basis years, it will shorten the basis years; however, if it is not divisible 9 the basis years will take precedence while if the critical 7-day year is greater than the basis years and \textsl{not} evenly divisible by 9 it will take precedence over the basis years.}.

We claim that the method of multiplying by 5 1/4 and then proceeding using this factor in the same way is more economical than multiplying by 52. Having found the number, one should investigate the 9-day number, to be sure that the 7-day number is not a factor of it.

\textbf{/141P/} (Some astrologers do not like to start the year with the rising of Sirius. It is possible to use any given starting point in citing an example, since we see that men begin the year differently in the different latitudes. Still, let us assume that the system in which the calculation starts with the rising of Sirius and proceeds to the birth date is more scientific. Most use this as the beginning of the year. Let this method of determining the 7-day week be the most efficient.)

Some think it proper to investigate the 7-day and the 9-day week for night births, the 9-day week for day births. The results are similar for both methods since the 7-day weeks will be with reference to Mars, the 9-day weeks with reference to Saturn. In either method, \textbf{/149K/} they will have an exchange of critical points. Saturn will be the beginning of the 7-day week because of the \Sun\xspace and \Moon. Mars will be the beginning of the 9-day week. This is because \Capricorn\xspace and \Aquarius\xspace (\Saturn’s houses) are opposite, in the seventh place, to \Cancer\xspace and \Leo\xspace <houses of the \Sun\xspace and \Moon>, and \Aries\xspace <\Mars’s house> is the ninth sign from \Leo, and \Cancer\xspace is the ninth sign from \Scorpio\xspace <\Mars’s house>. But it would be more scientific to derive these from the exaltation of the \Moon\xspace in \Taurus: the beginning of the 7-day week would be \Mars, because of \Scorpio; the beginning of the 9-day week would be \Saturn, because of \Capricorn.

An example: the nativity was in Hadrian year 3, Athyr 27 in the Alexandrian calendar\footnote{Riley: Athyr 27, Phamenoth 11 in the Alexandrian calendar; November 22, March 27 in the Greek calendar; Tybi 1, Pharmouthi 11 in the Egyptian calendar.
I think this is a recasting of a natal horoscope of Athyr 27, Alexandrian years. An Alexandrian month has 30 days. If you subtract 28 days (i.e. 4 weeks) from each month, there are 2
days left. So, there are 3 days left of Athyr, 2 days each for Choiak, Tybi, and Mechir (viz. 6 left after subtracting 12 weeks). The total so far is 9 plus 11 days of Phamenoth, for a total of 20. The intervening 35 years make 8 quadrennia, which he calls “intercalary.” Total 28. This topic is most clearly presented in Dorotheus’ Epic, Book V, Chapter 138. - marginal notes}. Investigate the subsequent date Antoninus year 17, Phamenoth 11. 

I take the full years, 35, plus the 3 remaining days in the birth month <Athyr 27 to 30>, plus 2 days for each month from Choiak to Mechir <3 months>. <The total is 44.> With the whole weeks <4 weeks=35 days> subtracted, the remainder is 9. Now add the 11 days of Phamenoth (total 20), plus the 8 intercalary days. The grand total is 28. Let Phamenoth 11 be a critical day in the 7-day week system\footnote{It appears that that since the calculation of days from birth to Phamenoth 11 yields a total of 28, which is evenly divisible by 7, it, Pamenoth 11, is a critical day. This would imply that if the number of days after birth divided by 7 equals zero then the day is a critical.}. According to the sequence of days, Phamenoth falls in \Scorpio. Examine which stars are in aspect with this signs and with the \Moon.

The 9-day week is found as follows: I multiply the full years by 5 1/4, since each year contains forty 9-day weeks with 5 1/4 days left over\footnote{40 x 9 = 360, 365 1/4 - 360 = 5 1/4.}. For each month I add 3, since each month has three 9-day weeks with 3 days left over. I total the remaining days until the day in question and I divide by 9 (making sure that the remainder is less than 9.) Now the result will be the number of the critical day, just as in the 7-day week system. It will be necessary to calculate the month and year in the same way.

\textbf{/142P/} In the 7-day week system, the critical signs are \Aries, \Libra, \Cancer, \Capricorn; in the 9-day
week system \Taurus, \Leo, \Scorpio, \Aquarius. Common to both systems are \Gemini, \Sagittarius, \Virgo, and \Pisces. \textbf{/150/}

\newpage