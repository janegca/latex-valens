\section{The Control}

Various astrologers have handed down various teachings about the basis of the <nativity’s> length of life. Since this topic seems quite complicated and complex, we will clarify it using methods proven by our own experience. The first topics of discussion will be “control\footnote{In the Schmidt translation, ``the control' is ``the predominator'' (VRS3 p27). The Medieval astrologers call it the \textsl{Hyleg}.},” “projection of rays,” and “houserulership.” 

First \mndl let the control with respect to the \Sun\, and the \Moon\, be investigated. Some give day births to the \Sun, night births to the \Moon, but I say that the \Sun\, controls night births and the \Moon\, day births, if they happen to be configured advantageously. If both are, assign control to the one which is more appropriately configured in its own sect or triangle. Then the houseruler\footnote{The Schmidt translation simply calls it the ``rulership'' (VRS3 p27); the Medieval astrologers called it the \textsl{Alchocoden}.} is found from the terms of the controlling star. 

If both <the \Sun\xspace and the \Moon> are unfavorably situated, then the term of the degree in the Ascendant or at MC will fix the houserulership, usually that <star> whose ruler is in an appropriate configuration with the Ascendant.

Let these controlling points be considered as proven by us. 

% -- Day Births ----------------------------------------------
\subsection{\textit{[Day Births]}}
The first control: the \Sun\xspace in \Leo, the \Moon\xspace in \Cancer; the luminary in appropriate configuration with the Ascendant or MC will have the control, and the ruler of its term will be houseruler. If both are in the terms of the same star, that star will unquestionably be judged the houseruler.

The second control: if the \Sun\xspace is in the Ascendant, the \Moon\xspace in <the XII Place of the> Bad Daimon, the \Sun\xspace will have control. 

If the \Sun\xspace is in <the XI Place of the> Good Daimon, and the \Moon\xspace is at MC, the \Sun\xspace will have control. 

If the \Sun\xspace is in the Descendant while the \Moon\xspace is in the sign just following the Descendant, \textbf{/133K/} the \Sun\xspace will have control. 

If the \Moon\xspace is in the Descendant while the \Sun\xspace is in the sign just following the Descendant, the \Sun\xspace will have control. 

If the \Sun\xspace just precedes MC while the \Moon\xspace
is in the Ascendant, the \Moon\xspace will have control. 

If the \Sun\xspace again just precedes MC while the \Moon\xspace just follows the Ascendant, the \Moon\xspace will have control. 

If the \Sun\xspace again just precedes MC while the \Moon\xspace \textbf{/126P/} is also at MC, the \Moon\xspace will have control. 

If the \Sun\xspace just precedes MC and the \Moon\xspace just follows MC, the \Moon\xspace will have control. 

If the \Moon\xspace precedes MC while the \Sun\xspace is at IC, the \Sun\xspace will have control. 

If the \Moon\xspace precedes MC while the \Sun\xspace just follows IC, the \Sun\xspace will have control. 

If the \Sun\xspace precedes the overhead angle while the \Moon\xspace follows IC, the \Moon\xspace will have control. 

If the \Sun\xspace precedes the overhead angle while the \Moon\xspace is at IC, the \Moon\xspace will have control. 

If both luminaries precede MC, the Ascendant will have control and the ruler if its terms will be considered the houseruler. 

If the \Moon\xspace follows MC while the \Sun\xspace is in <the IX Place of> the God, the luminary which first sends its rays exactly to the Ascendant’s degree-position will have control. 

If the \Sun\xspace and the \Moon\xspace just precede the Ascendant in the XII Place, MC will have control and the ruler of its terms will be houseruler.

As can be seen, if the nativity is during the day, the luminaries are not dominant if they are above the earth. The Ascendant will have control and the ruler of its degrees will be the houseruler. For night births, if the luminaries precede IC, MC will have control\footnote{In other words, if it is a day birth and both lights are badly placed above the horizon, then use the Ascendant as the control and its term ruler as the houseruler. For night births, if the lights are in the 3rd (precede the IC) then use the Midheaven as the control and its term ruler as the houseruler.}.

% -- Night Births --------------------------------------------
\subsection{\textit{[Night Births]}}

If the \Sun\xspace just follows IC while the \Moon\xspace just
precedes MC, the luminary which first sends its rays exactly to the Ascendant will have control. 

If the \Sun\xspace and the \Moon\xspace are in the Descendant, the term of the <preceding> new moon will have control and the ruler of its degrees will be the houseruler. Similarly if both are in the Ascendant, at MC, or at IC, the term of the new moon will have control and the ruler of its terms will be the houseruler. 

If the luminaries are in the same sign (or in different signs) and in the terms of the same star, infallibly that star will be the houseruler.

If the \Sun\xspace is found to be in its own depression <\Libra>, it will not be the apheta, unless it is exactly in
the Ascendant (to the degree). The same is true for the \Moon\xspace in \Scorpio\xspace <its depression>. 

If the \Moon\xspace is found to be new and to be under the rays of the \Sun, it will not be the apheta, unless it too is exactly in the Ascendant.

If the \Moon\xspace is nearing full \textbf{/134K}/ and passes out of this phase within the term in the Ascendant, it will be both the apheta and the anaereta, if it passes out of the full-moon phase on that same day. It will be necessary to examine the number of degrees between this day and the full moon; having found this number, <you can> forecast the number of years. \textbf{/127P/}

For example: Ascendant, \Moon\xspace in \Aries 22°. On the same
day it passed out of the full-moon phase at 27° of the same sign. The distance from its position then and the full moon was 5°, which totals 4 years. The native lived that many years.

Death will occur particularly if a malefic applies its rays and if it is in aspect or opposition to the sign.

If a benefic is in the same relation, there will be infirmity and disease instead of death. The rest of the \Moon’s phases during its connection with <the \Sun> are destructive.

It is necessary to consider the control to be certain if the \Sun\xspace or the \Moon\xspace is in aspect with the ruler of the terms, and if it is at an angle or in operative degrees. If it is found to be turned away, the nativity is judged to lack a houseruler. 

If the ruler of the \Sun’s or the \Moon’s sign and the ruler of the terms exchange terms, then too will the houserulership be without a controller. 

It will be necessary to determine if the star that seems to be houseruler is in the Descendant, for if it is, that nativity as well will lack a houseruler.

\newpage
\subsection{\textit{[Summary of Conditions for a Control]}}
\vspace{1em}
\begin{mdframed}[backgroundcolor=cyan!5]
\begin{longtable}[tc]{P{0.15\linewidth} P{0.15\linewidth} P{0.6\linewidth}}
\Sun\, in & \Moon\, in & \textbf{Qualifies as the Control} \\
\toprule
\Leo & \Cancer & whichever is best configured with the Asc or MC \\
\midrule
\Libra & \Scorpio\, \tiny{or USB} &  the light can only be Control if in partile aspect to Asc \\
\midrule
Asc & 12th & \Sun \\
11th & 10th & \Sun \\
7th & 8th & \Sun \\
8th & 7th & \Sun \\
\midrule
9th & \makecell{1,2,5,4,\\10,11*} & \Moon \\
\midrule
4 or 5* & 9th & \Sun \\
9th & 9th & Ascendant \\
12th & 12th & Midheaven \\
3rd & 3rd & Midheaven \\
6th & 6th & Midheaven \tiny{(implied, not stated)} \\
7th & 7th & New Moon before birth$^\dag$ \\
1st & 1st &  `` \\
4th & 4th &  `` \\
10th & 10th &  `` \\
\bottomrule
\end{longtable}
\vspace{-1em}
\tiny{
* Or use the light most that will be first to aspect (or is most closely in aspect(?)) with the Ascendant.

\vspace{-0.5em}
\noindent $\dag$ This could be the syzygy and not simply the New Moon before birth.}

\normalsize
A Control is certain if its term lord is not averse to it or it is in an angle or in operative degrees (also see \S3.3 \textsl{The Vital Sector} for what constitutes a valid Control).
\end{mdframed}

\newpage
\subsection{\textit{[Summary of Conditions for a Houseruler]}}
\vspace{1em}
\begin{mdframed}[backgroundcolor=cyan!5]
If the \Sun\, and \Moon\, have the same term ruler it becomes the houseruler; otherwise, the term ruler of the Control is the houseruler.

\noindent There will be no houseruler if:
\vspace{-1em}
\begin{itemize}
\item the houseruler is averse to its own domicile ruler
\item the houseruler is in the Descendant
\item the domicile ruler of the Control and the houseruler are in the terms of the same planet
\end{itemize}

If the \Moon\, is a Full Moon in the same terms as the Ascendant and will separate from its opposition to the \Sun\, before leaving those terms then it will be both the Control and the killing planet.

If the ruler of the Control and ruler of the houseruler are in mutual reception by term there will be a houseruler but no Control.

(also see \S3.3 \textsl{The Vital Sector} for further information on what makes an active Control and houseruler.)
\end{mdframed}

\newpage